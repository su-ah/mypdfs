\subsection{Manifolds, Properly}

\begin{defn}
    a \textbf{manifold} (mfld) is a set $M$ w max atlas $A$, s.t.
	$A$ is 2nd ctbl and Hausdorff.
\end{defn}
nows a good time as any to remind you* that charts need not be cts,
since we havent defnd a topology on $M$ yet.
go look at the defn again.

a quick note on notation - we write $(x^{0}:x^{1}:\cdots:x^{n})$
for the equivalence class of $(x^{0},x^{1},\dots,x^{n})\in\bR^{n+1}\sm\set{0}$.

\begin{xmp}[source=Primary Source Material]
	\vspace{-0.3in}
    \begin{enumerate}
		\item $S^{n}$ by stereo proj
		\item $\bR P^{n}$, real projective $n$-space
			$= \set{\trm{1-dim subspaces of }\bR^{n}}$.
			ok hes just spamming RPn defs now.

			as quot of $S^{n}$, we can choose a chart such as the proj map
			on the open upper hemisphere, or equivalently:
			\begin{equation*}
				(x^{0}:\cdots:x^{n}) \mto
				\left(\frac{x^{1}}{x^{0}},\dots,\frac{x^{n}}{x^{0}}\right)
				\qquad
				(x^{0}:\cdots:x^{n})\in\set{(x^{0}:\cdots:x^{n}):x^{0}\neq0}
			\end{equation*}
			note the maps are equiv in the sense that their transition fn is
			smooth. note we can also do the same for any coordinate.

			thus, we get a chart for each, i.e. $n+1$ charts, which cover
			$\bR P^{n}$.
    \end{enumerate}
\end{xmp}


