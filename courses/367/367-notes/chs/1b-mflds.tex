\subsection{Manifolds, Properly}

\begin{defn}
    a \textbf{manifold} (mfld) is a set $M$ w max atlas $A$, s.t.
	$A$ is 2nd ctbl and Hausdorff.
\end{defn}
nows a good time as any to remind you* that charts need not be cts,
since we havent defnd a topology on $M$ yet.
go look at the defn again.

a quick note on notation - we write $(x^{0}:x^{1}:\cdots:x^{n})$
for the equivalence class of $(x^{0},x^{1},\dots,x^{n})\in\bR^{n+1}\sm\set{0}$.

\begin{xmp}[source=Primary Source Material]
	\vspace{-0.3in}
    \begin{enumerate}
		\item $S^{n}$ by stereo proj
		\item $\bR P^{n}$, real projective $n$-space
			$= \set{\trm{1-dim subspaces of }\bR^{n}}$.
			ok hes just spamming RPn defs now.

			as quot of $S^{n}$, we can choose a chart such as the proj map
			on the open upper hemisphere, or equivalently:
			\begin{equation*}
				(x^{0}:\cdots:x^{n}) \mto
				\left(\frac{x^{1}}{x^{0}},\dots,\frac{x^{n}}{x^{0}}\right)
				\qquad
				(x^{0}:\cdots:x^{n})\in\set{(x^{0}:\cdots:x^{n}):x^{0}\neq0}
			\end{equation*}
			note the maps are equiv in the sense that their transition fn is
			smooth. note we can also do the same for any coordinate.

			thus, we get a chart for each, i.e. $n+1$ charts, which cover
			$\bR P^{n}$.
    \end{enumerate}
\end{xmp}

\newpage
\lecdate{Lec 4 - Jan 15 (Week 2)}
snow day looll

\lecdate{Lec 5 - Jan 20 (Week 3)}

\begin{defn}
	a subset $U\subseteq M$ is \textbf{open} if for all charts
	$\vphi_{\alpha}:U_{\alpha}\gto\bR^{n}$, the image
	$\vphi_{\alpha}(U\cap U_{\alpha})$ is open in $\bR^{n}$.
\end{defn} \

\begin{prop}
    $U\subseteq M$ open iff $\vphi_{\beta}(U\cap U_{\beta})$ open for
	some coll'n $(U_{\beta},\vphi_{\beta})$ cvring $U$.
\end{prop} \

\begin{pf}[source=Primary Source Material]
    fix $\set{(U_{\beta},\vphi_{\beta})}$.
	let $\vphi_{\alpha}$ be another chart in the (max'l) atlas.
	then:
	\begin{gather*}
	    U\cap U_{\alpha} \ = \ \bigcup_{\beta}U\cap U_{\alpha}\cap U_{\beta} \\
		\vphi_{\alpha}(U\cap U_{\alpha}\cap U_{\beta})
		\ = \ (\vphi_{\alpha}\circ\vphi_{\beta}^{-1})\circ\vphi_{\beta}
		(U\cap U_{\alpha}\cap U_{\beta}) \\
		\ = \ (\vphi_{\alpha}\circ\vphi_{\beta}^{-1})
		\left(\vphi_{\beta}(U\cap U_{\beta})\cap
		\vphi_{\beta}(U_{\alpha}\cap U_{\beta})\right)
	\end{gather*}
	since $\vphi_{\beta}(U\cap U_{\alpha}\cap U_{\beta})$ is open,
	$\vphi_{\alpha}(U\cap U_{\alpha}\cap U_{\beta})$ is, and
	$\vphi_{\alpha}(U\cap U_{\alpha})$ is a union of open sets.
	(i think?)
\end{pf}

fact: if $A$ atlas on $M$, $U\subseteq M$ open, then:
\begin{equation*}
	A_{U} \ = \
	\set{(U\cap U_{\alpha}, \vphi_{\alpha}\big\rvert_{U\cap U_{\alpha}})}
\end{equation*}
is an atlas.
moreover, if $A$ max/2nd ctbl/T2, so is $A_{U}$.
in particular, any open subset of a mfld is a mfld.

we also borrow certain terms from topology, namely
nbhds, closed sets, connected.

note we define a nbhd of a pt as a (!) set containing an open set containing
the pt. importantly, the nbhd need not be open itself.

\begin{lm}[type=Remark]
    $U$ open iff $U=\bigcup U_{\alpha}$ for charts
	$(U_{\alpha},\vphi_{\alpha})$ in a max atlas of $M$
\end{lm}
in other words, a max atlas is indeed a basis for the mfld topology.

\begin{pf}[source=Primary Source Material]
	take restrictions... i think? idk, had to dip
\end{pf}

\subsection{Orientability}
\lecdate{Lec 6 - Jan 22 (Week 3)}
finally... this time... this time for sure...

usual non-orientability example: mobius strip.
if we were to place tangent vectors, perhaps completing each to a tangent
space, then we would - in theory - want them to satisfy the ``right-hand rule",
or equivalently, that $\det(v_{1} \ v_{2} \ v_{3}) > 0$.

he drew hermit crabs on a mobius strip omg

\begin{defn}
    a lin transformation is \textbf{orientation-preserving} if $\det>0$.

	a transition map $\tau$ btwn 2 charts is \textbf{orientation-preserving} if
	$\det D\tau>0$ everywhere.
	in this case, we say the charts are \textbf{orientation-compatible}.

	an atlas is \textbf{oriented} if all charts are orientation-compatible,
	a \textbf{maximal oriented atlas} is an oriented atlas which is maximal
	(and not vice versa), and
	a mfld is \textbf{orientable} if it has an oriented atlas.
\end{defn}

\newpage
\begin{exr}[source=Primary Source Material]
    given a max orient atlas and two charts orient compat w the atlas,
	show the charts are orient compat w each other.
\end{exr} \

\begin{xmp}[source=Primary Source Material]
    sphere is orientable.
	note our usual atlas of double stereo proj is \textit{not} oriented,
	but can be made oriented:
	\begin{equation*}
		(\vphi_{N},(-x_{1},x_{2},\dots,x_{n})\circ\vphi_{S})
	\end{equation*}
	claim: this is oriented.
	pf: do a computation. alternatively, for $n\geq2$:

	note $U_{N}\cap U_{S}=S^{n}\sm\set{N,S}$ is conn.
	therefore, $\det(D\tau)\neq0$ must be the same on the entire img.
\end{xmp} \

\begin{lm}
    if $X$ conn and $A$ some set, then
	any ``locally constant" $f:X\gto A$ is constant.

	$f$ loc constant if every $p\in X$ has nbhd $U$ s.t. $f(U)=f(p)$.
\end{lm} \

\begin{pf}[source=Primary Source Material]
    $f^{-1}(a)=\bigcup_{p\in f^{-1}(a)}U_{p}$ open.
	then $X=\bigsqcup_{A}f^{-1}(a)$, so only one can be non-empty.
\end{pf} \

\begin{exr}[source=Primary Source Material]
    $\bR P^{2}$ is not orientable.
	proof by hermit crabs :3 (exercise)
\end{exr} \

\begin{defn}
    given orient mfld $(M,A)$, the \textbf{opposite orientation}
	is given by the atlas:
	\begin{equation*}
	    \tilde{A}:=\set{(F(U),F\circ\vphi):(U,\vphi)\in A} \qquad
		F(x) = (-x_{1},x_{2},\dots,x_{n})
	\end{equation*}
	where $F$ is a ``flip" of the first coord.
\end{defn} \

\begin{prop}
    sps $M$ orient mfld.
	then every conn chart (ie $U$ conn) compat w $\alpha$ is
	orient compat w \textit{either} $A$ or $\tilde{A}$.
\end{prop} \

\begin{pf}[source=Primary Source Material]
	let $A=\set{(U_{\alpha},\vphi_{\alpha})}$. then:
	\begin{equation*}
	    \Sigma_{p}:=
		\sgn(\det D(\vphi_{\alpha}\circ\vphi^{-1})\rvert_{\vphi(p)})
	\end{equation*}
	is indep of $\alpha$.
	we claim $\Sigma_{p}$ also indep of $p$.

	indeed, $\Sigma_{p}:U\gto\set{\pm}$ is loc const, since
	$\Sigma_{p}$ const on $U_{\alpha}\cap U$ $\fa\alpha$.
	by earlier lemma, its constant.
\end{pf} \

\begin{crll}
    if $M$ orient conn, there are exactly 2 max orient atlases.
\end{crll}



