\subsection{Preliminaries and Basics}

\lecdate{Lec 1 - Jan 6 (Week 1)}

we know what manifolds are.
recall implicit thm:

\begin{thm}[title=Implicit Function Thm]
    given eqn $f(x_{1},\dots,x_{n+1})=0$ for smooth $f$ and a soln
	$p\in\bR^{n+1}$ s.t. $\nabla f\neq0$ at $p$, then
	$f(x_{1},\dots,x_{n+1})$ is the soln set near $p$.
	
	furthermore, we can represent solns as
	$(x_{1},\dots,x_{n},g(x_{1},\dots,x_{n}))$, where $g$ also smooth.
\end{thm}

note $f(x_{1},\dots,x_{n+1})=0$ is locally the graph of a fn.
if $0$ a reg pt of $f$, then we can cvr $\set{x:f(x)=0}$ by graphs,
i.e. charts. oh example spam sure

\begin{thm}[title=Whitney Embedding Thm]
    every $n$-mfld has an embedding in $\bR^{2n}$.
\end{thm}
we wont get to this thm in this course, but its neat. or sth

\begin{xmp}[source=Primary Source Material]
	let $M$ be the set of all rots of a ball.
	this is a \textit{configuration space} - pts represent ways to configure
	another object/space.

	how do we put coords on (a piece of) this space?
	one way: identify w $\SO(3)$.
	another way: where $N$ goes has 2 degs of freedom.
	where the vector $e_{1}=(1,0,0)$ goes @ $N$ is 1 deg of freedom.
	thus, $M$ is 3-dim; this also gives a way of defining a chart near the id.
\end{xmp}

\newpage
\begin{xmp}[source=Primary Source Material]
	consider $M$ as the config space of a \textit{linkage}
	(picture a graph, but physical, where edges can rotate around
	vertices - saw a vid abt this i think).
	in this ex, sps it ``looks like" a quadrilateral, and wlog one edge is
	fixed in space; all edges coplanar.
	what dim is $M$?

	each vec is in $\bR^{2}$, so we start w 8 variables.
	but since its a closed shape, $n_{1}+n_{2}+n_{3}+n_{4}=0$, so -2 dims.
	furthermore, $\norm{n_{i}}$ fixed, so -4 dim.
	since $n_{1}$ fixed, -1 dim; this leaves 1 dim, and indeed,
	$M$ is 2-dim'l.
	it can in fact be parameterized by two adj angles $\theta,\vphi$.
\end{xmp} \

\begin{lm}[type=Fact]
    ``closed" (ie cpt) surfaces are easy to enumerate (classify).
	\vspace{-.15in}
	\begin{itemize}
	    \item orientable: sphere or conn sum of torii
		\item non-orientable: $\bR P^{2}$, klein bottle, conn sum of those
			w torii (handles)
	\end{itemize}
\end{lm}

ok now for our proper defns.
\begin{defn}
    a \textbf{coordinate chart} is an inj $\vphi:U\gto\bR^{n}$ w open img
	for some $U\subseteq M$.

	two charts $\vphi:U\gto\bR^{n}$ and $\psi:V\gto\bR^{n}$ are
	\textbf{compatible} if the transition function given by:
	\begin{equation*}
	    \psi\circ\vphi^{-1}:\vphi(U\cap V)\gto\psi(U\cap V)
	\end{equation*}
	is a diffeo.
	we also enforce that if $U\cap V=\eset$, then they are compatible.
\end{defn}

\newpage
\lecdate{Lec 2 - Jan 8 (Week 1)}

idea: charts give a coord system on $U\subseteq M$.
in particular, coordinates are given by $\vphi(p)=(x^{1},\dots,x^{n})$.
then, transition maps represent coord \textit{changes}.

this guy is doing weird inverse wizardry. ok whatever man.

\begin{defn}
    a chart is compatible with an atlas (we know what this is) if it is
	compatible w every chart in the atlas.
\end{defn} \

\begin{lm}
    if two charts are compatible w the same atlas, they are compatible
	w each other.
\end{lm} \

\begin{thm}[type=Definition/Theorem]
    given an atlas $A$ on $M$, theres a unique maximal atlas $\tilde{A}$
	which consists of every chart compatible w $A$.
	in particular, every chart compatible w $\tilde{A}$ is alr in $\tilde{A}$.
\end{thm} \

\begin{pf}[source=Primary Source Material]
    pf that $\tilde{A}$ is an atlas:
	it cvrs $M$ since $A\subseteq\tilde{A}$, and is pairwise compatible by
	the lemma.

	pf that all charts compat w $\tilde{A}$ is in $\tilde{A}$:
	compat w $\tilde{A}$ implies compat w $A$ which means its in $\tilde{A}$.
\end{pf}

\lecdate{Lec 3 - Jan 13 (Week 2)}

still tryna define a mfld. maybe ill actually start takign notes. hmmmm

current draft:
a mfld is a set $M$ w a maximal atlas $A$.
note that given any atlas, we can uniquely complete to a max atlas.
furthermore, 2 charts gen the same atlas iff each chart in one is cptbl w
each chart in the other.

possible problems: $M$ is ``too big".
eg, $M=\bR$ w singleton charts, long line, line w two origins.
in particular, the line w two origins is not T2 at the origins.

we can now give the final defn for a mfld.


