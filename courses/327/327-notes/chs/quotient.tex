\subsection{Quotient Spaces}
\lecdate{Lec 12 (Makeup) - Jun 17 (Week 7)}

motivating example: the pacman game board is homeo to a torus.

\begin{defn}
    let $X,Y$ be spaces, $p:X\sto Y$ be surj.
    we say $p$ is a \textbf{quotient map} [qmap] if for any $U\subseteq Y$,
    $U$ open iff $p^{-1}(U)$ open.

    for any set $Y$ and surj $p:X\sto Y$, we define the
    \textbf{quotient topology} as subsets $U\subseteq Y$ with $p^{-1}(U)$ open.
\end{defn}
exercise: check this is a topology

how do we characterize cts functions $f:X/\sim\sto Y$?

\begin{prop}
    let $X,Y$ be spaces, $\sim$ equiv rel on $X$, and $p$ the qmap.
    if $g:X\sto Y$ is a function that is constant on equiv classes, then $g$
    induces a map $f:X/\sim\sto Y$ s.t. $g=f\circ p$.

    in particular, $f$ cts iff $g$ cts, and $f$ qmap iff $g$ qmap.
\end{prop}

\begin{pf}[source=Primary Source Material]
    define $f$ by $f([x])=g(x)$.
    since $g$ constant on classes, this is well-defined; clearly $g=f\circ p$.

    clearly if $f$ cts, $g$ is also cts.
    if $g$ cts, then for any open $U\subseteq Y$, $g^{-1}(U)$ open.
    but $g^{-1}(U)=p^{-1}(f^{-1}(U))$ so $f^{-1}(U)$ open by defn.

    the fact about qmaps follows from $p$ being a qmap.
\end{pf} \

\begin{crll}
    sps $g:X\sto Y$ surj cts.
    define $\sim$ on $X$ as in the canonical decomposition of $g$.
    then $g$ induces a map $f:X/\sim\sto Y$ and $f$ homeo iff $g$ qmap.
\end{crll}

\begin{pf}[source=Primary Source Material]
    clearly $g$ constant on equiv classes, so $f$ follows from the thm.

    if $f$ homeo, then $f$ qmap, so $g$ qmap;
    if $g$ qmap, then $f$ qmap, so $f$ homeo since its bij.
\end{pf}



