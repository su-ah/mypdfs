\documentclass{article}
\usepackage{preamble}
\usepackage{env}

\def\perthousand\relax
\def\micro\relax

% available environments:
% theorem: thm
% definition: defn
% proof: pf
% corollary: crll
% lemma: lm
% question: qu
% solution: soln
% example: xmp
% exercise: exr
%
% options: title=<title>   {all}
%          source=<source> {pf, qu, soln, xmp, exr}
%               Note: if content is taken directly from the main resource,
%                     cite the main resource as ``Primary source material"


% define these variables!
\def\coursecode{MAT402H5}
\def\coursename{Classical Geometries} % use \relax for non-course stuff
\def\studytype{2} % 1: Personal Self-Study Notes / 2: Course Lecture Notes / 3: Revised Notes / 4: Exercise Solution Sheet
\def\author{\me}
\def\createdate{January 10, 2025}
\def\updatedate{\today}
\def\source{Class Lectures} % name, ed. of textbook, or `Class Lectures` for class notes
\def\sourceauthor{Prof. Emmy Murphy} % for class notes, put lecturer
% \def\leftmark{} % set text in header; should only be necessary in assignments etc.
% \pagenumbering{arabic} % force revert numbering to default; should only be necessary in assignments etc.

\makeatletter
% settings for toc alignment
%
% Configuration
% -------------
% Horizonal alignment in \numberline:
%   l: left-aligned
%   c: centered
%   r: right-aligned
% \nl@align@: Default setting
% \nl@align@<levelname>: Setting for specific level

\def\nl@align@{l}% default
\def\nl@align@section{r}

\makeatother

\begin{document}

\cover
\toc
\blurb

% start here

\section{Introduction}
\subsection{The Three Classical Geometries (and some history)}
\lecdate{Lec 1 - Jan 10 (Week 1)}

Generally, there are three types of well-known (classical) geometry, dating back to the greeks:
\begin{enumerate}
    \item Euclidean geometry, the oldest of the three, synonymous with flat plane geometry.
    \item Elliptic/Spherical geometry, the study of geometry on a sphere (or similar curved
        surfaces). One major text is \textit{The book of unknown arcs of a sphere}, by Al-Jayyani,
        dating to the 11th century.
    \item Hyperbolic geometry is the newest of the three, due to Lobachevsky (and tangentially
        prior, Gauss). uhh sth sth
\end{enumerate}
Of course, Euclid's \textit{Elements} is the canonical foundation of mathematics. It studies
plane geometry in the flat, or \textit{zero curvature} case. Although Euclid's work is exceptional,
Euclidean geometry actually dates back much further by at
least a century. In a sense, geometry is older than history itself.

Some form of spherical geometry has also been known since pre-history.
Autolycus nad Theodosius particularly wrote books specifically on spherical geometry in BCE,
but unlike \textit{Elements}, were more obscure in the West.
Al-Jayyani's work is notable specifically as a text on spherical \textit{trigonometry}, giving
means to computing explicit arclengths. Flat plane trig arose approx 100y prior in the Islamic
Golden Age.

In the European Renaissance, spherical geometry had (at least) two massive touchstones: perspective
drawing and global cartography. Concerning the development of mathematics, spherical had less
impact compared to other Golden Age and Renaissance works (algebra, trig, calculus, etc.).

Euler studied spherical extensively in the mid 1700s. Elliptic geometry specifically refers to a
version of spherical which conforms to Hilbert's synthetic geometry (Antipode problem). This
separation only made sense in the 19th century, around when hyperbolic came around.

Hyperbolic, by contrast, doesn't really have pre-Enlightenment study.
This is due to the fact that, unlike flat planes and spheres, there isn't any real-world analogue
of the hyperbolic plane.

As a result, it is \textit{purely abstraction}, one of the first to be studied.
Today, it is considered a catalyst for a larger revolution, where mathematicians were free to
study ``pure abstractions" with no referent to the real world. Instead, an abstraction could be
seen as ``legitimate" simply by its context among other abstractions.

Hyperbolic continues to be widely studied, and has more depth/challenges than the other two.
However, it is not \textit{inherently} more complex than the geometry of Euclid. \npgh

Though discovered at different times, these geometries are beset understood in context of each
other. A general theme in math is that of trichotomy: a (real) number is either positive, negative,
or zero. Similarly, there are exactly three (homogeneous) geometries in two dimensions: flat
(zero curvature), spherical (positive curvature), and hyperbolic (negative curvature).

The discovery of hyperbolic (and the relationship between the three) heralded many later discoveries
in the 20th century. The \textit{Uniformization Theorem} of 1907 relates complex analysis to the
three. From this and relationships to surface topology, the field of geometric topology was born.

\newpage
\subsection{Some Intuition/Examples}

Recall the Euclidean space, the space we're familiar with.
We know that the sum of interior angles of a triangle is 180 degrees.
We also know that the area of a circle of radius $ r $ is given as $ r^{2}c $, for some constant
$ c $ (in this case, $ \pi $).

However, on, say, a sphere, this may not be the case. Consider a triangle made of two perpendicular
meridians together with the equator. This would be a triangle whose interior angles sum to 270
degrees. In fact, on a sphere, the interior angles \textit{always} sum to a value greater than 180.
Similarly, a triangle has area equal to $ (A+B+C - 180)c $, where $ A,B,C $ are the angles of the
triangle.

We can also consider the area of a disc (rather than a circle), where the disk is the set of points
all within some distance of a given point. The area of such a disc would be equal to
$ (1-\cos(r))c $, which is \textit{smaller} than Euclidean space, and for small values of $ r $ is
roughly equivalent to $ \pi r^{2} - \frac{\pi}{6}r^{4} $.

In hyperbolic space, triangles have interior angle sum \textit{less} than 180 degrees.
Furthermore, the area is $ (180 - A - B - C)c $, where $ A,B,C $ are the interior angles.
The area of a disc in hyperbolic space also turns out to be (approximately)
$ (r^{2} + \frac{r^{4}}{6})c $ when $ r $ is small, which is \textit{larger} than Euclidean.

In non-Euclidean space, notice that the area of a triangle can be obtained from its interior
angles. In other words, two triangles with the same interior angles must always have the same area.
Euclidean space is distinct in this sense, as this does not hold true - thus, these spaces do not
quite have the same notion of similar triangles.

Notice that in Euclidean space, area is always proportional to $ r^{2} $. In spherical space, the
area is upper bounded, and in hyperbolic space, it is \textit{exponential} in area.

\lecdate{Lec 2 - Jan 14 (Week 2)}
okay i think im not gonna do the same kind of writing lol,
but in essence, euclid's elements are p pivotal

The five postulates of Euclid's Elements:
\begin{enumerate}
    \item[P1:] Two points determine a line.
    \item[P2:] Any line segment can be extended infinitely.
    \item[P3:] We can construct a circle with any given radius about any given center.
    \item[P4:] All right angles are ``equal".
    \begin{itemize}
        \item By ``equal", Euclid really means what we call congruence. (What is congruence?)
        \item In a way, nonsensical from a modern perspective.
    \end{itemize}
    \item[P5:] Suppose we have three lines with the third intersecting both the first and second
        lines. If the sum of the ``interior" angles is less than 180 degrees (originally phrased as
        two right angles), then the first two lines intersect. [diagram to follow]
\end{enumerate}
Note that P5 is \textit{not local} - it holds true if one of the angles is a right
angle, and the other is even 1/1000th less than a right angle. It is a statement of the idea which
proposes that ``If [something] happens here, then [something else] happens somewhere else
(potentially far away)."

P5 feels distinct compared to the other postulates; it seems like it should (maybe) be a result
rather than an assumption. That is, P5 seems \textit{provable} from P1-4. Throughout history, many
tried but none succeeded. Nowadays, we know why: other (non-Euclidean) geometries don't follow P5.

\newpage
\section{Absolute Euclidean geometry}

\begin{defn}
    \textbf{Absolute geometry} is plane geometry which is agnostic toward P5.
\end{defn}
i kind of zoned out due to eepy. oops =P also these r propositions. i missed one. lol.

\begin{lm}[title=Prop. 1]
    prop 1 which i missed. u can always form an equilateral triangle? sth like that
\end{lm}

\begin{lm}[title=Prop. 2]
    Given $ \oline{AB} $ and a point $ P $, there is a line segment $ \oline{PQ} $ at $ P $
    such that $ \oline{AB} \cong \oline{PQ} $.
\end{lm}
diagram...

To note is the usage of subtraction of lengths and magnitudes, as well as the focus of thinking of
values as lengths.

\begin{lm}[title=Prop. 3]
    Given two line segments, there exists a line segment with the length of the difference in their
    length.
\end{lm}

\begin{lm}[title=Prop. 4]
    Proposition 4(?) is about SAS congruence.
\end{lm}
Euclid's proof uses ``superposition" - when one triangle is placed on top of the other,
there is no other choice. Many/most modern (high school) geometry texts take SAS as an assumption
(following Hilbert's Axiomatization, 1899). This is not needed. Something else is, as Euclid
is vague. (thoughts of prof.)

\begin{lm}[title=Prop. 8]
    Prop 8 gives the SSS triangle congruence.
\end{lm}
This has similar issues as the above.

\begin{lm}[title=Prop. 11]
    Given a line $ L $, a point $ A $ on $ L $, we can construct the line through $ A $
    perpendicular to $ L $.
\end{lm}
[diagram] To prove: the angle is a right angle. (uses props 9, 10)

\begin{lm}[title=Prop. 12]
    Given a line $ L $ and a point $ A $ \textit{not} on $ L $, we can construct the line
    through $ A $ which is perpendicular to $ L $.
\end{lm}
[diagram] The point $ P $ is taken to be the midpoint of $ \oline{BC} $. (midpoints are from an
earlier prop.)

\begin{lm}[title=Prop. 16]
    Given $ \Delta ABC $, the exterior angle (say, at $ C $) is larger than either opposite angle.
\end{lm}
[diagram]
Let $ E $ be the midpoint of $ \oline{AC} $.
Draw $ \oline{BE} $ (ray), and stop at $ F $ where $ \oline{EF} \cong \oline{BE} $.
Draw $ \oline{CF} $. Then $ \Delta EAB \cong \Delta ECF $ (by SAS, and that opposite angles are
congruent).
But $ ECF < ECR $, since $ ECF $ is contained in $ ECR $. This shows that $ A $ is less than the
exterior angle; angle $ B $ is essentially analogous.

Aside: what about spherical? Taking a triangle between two meridians and the equator, it is easy
to see that the proposition does not hold!

In some ways, spherical is the odd one out. Specifically, it has some topology issues which make it
distinct. For instance, relevant here:
\begin{itemize}
    \item A ``line" is a circle on the sphere. In particular, the center of the circle is the
        center of the sphere. Curved lines do not have the same center.
    \item Lines - and in fact space itself - wraps back on itself.
    \item Lines intersect at \textit{two} points.
\end{itemize}
The point is that Euclid's ``hidden assumptions" regarding separation and betweeness rules out
spherical geometry. (Can this be fixed?)

\lecdate{Lec 4 - Jan 21 (Week 3)}

At proposition 29, we shift into Euclidean geometry:
\begin{lm}[title=Prop. 29]
    The interior angles of a triangle sum to $ 180 \degree $.
\end{lm}

\begin{lm}[title=Prop. 33]
    Rectangles exist, and opposite sides are congruent.
\end{lm}

\begin{lm}[title=Prop. 34]
    In parallelagrams, opposite sides and angles are congruent.
\end{lm}

...

\begin{lm}[title=Prop. 37/38]
    Area of a triangle only depends on base and height.
\end{lm}

\begin{lm}[title=Prop. 41]
    The area of a triangle is half the area of the corresponding parallelogram.
\end{lm}

\begin{lm}[title=Prop. 47]
    $ a^{2}+b^{2} = c^{2} $. Yep.
\end{lm}
Of particular importance are propositions 33 and 47.

33 is important, because it serves as the basis of \textit{coordinates}.

Let $ \bb{E}^{2} $ (or $ \bb{E} $) be the Euclidean plane.
Let $ \bb{R}^{2} = \set{(x,y): x,y \in \bb{R}} $.
Although we think of them as essentially the same thing (and in many other different ways), there
is historically a subtle difference.

\end{document}
