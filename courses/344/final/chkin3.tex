\documentclass{article}
\usepackage{preamble}
\usepackage{envedit}
% \usepackage{configure}

% available environments:
% theorem: thm
% definition: defn
% proof: pf
% corollary: crll
% lemma: lm
% question: qu
% solution: soln
% example: xmp
% exercise: exr
%
% options: title=<title>   {all}
%          source=<source> {pf, prb, soln, xmp, exr}  Note: if content is taken directly from the main resource, cite the main resource as ``Primary source material"


% define these variables!
\def\coursecode{MAT344}
\def\coursename{\relax} % use \relax for non-course stuff
\def\studytype{} % 1: Personal Self-Study Notes / 2: Course Lecture Notes / 3: Revised Notes
\def\author{Emerald (Emmy) Gu}
\def\createdate{}
\def\updatedate{\today}
\def\source{} % name, ed. of textbook, or `Class Lectures` for class notes
\def\sourceauthor{} % for class notes, put lecturer
\def\leftmark{Check-In 3} % set text in header; should only be necessary in assignments etc.
\pagenumbering{arabic} % force revert numbering to default; should only be necessary in assignments etc.

\begin{document}

% start here

\begin{mylm}[type=Conjecture 1]
Let $ G_{1}, G_{2} $ be simple, undirected, unweighted graphs. Denote the adjacency matrices of $ G_{1}, G_{2} $ as $ A_{1}, A_{2} $ respectively. \vsp
Suppose $ G_{1} $ and $ G_{2} $ are isomorphic. Then, $ A_{1} $ is similar to $ A_{2} $.
\end{mylm}
This is my very first conjecture which came to be as a natural response to an early and immediate concern of mine.
Since the rows/columns of the adjacency matrix of a graph each represent an individual vertex in $ G $, then does the order in which you index the vertices on $ G $ affect the adjacency matrix? \vsp
The answer is a very clear yes, in the sense that the matrix may \textit{appear} different.
But since we mainly care about eigenvalues, we examine those - and we see that they, in fact, do not change.
This conjecture is the conclusion of a brief investigation and generalization in that direction. \npgh

The relation between isomorphic graphs and similar (adjacency) matrices is one that feels very natural to me - graph isomorphism and matrix similarity are both ways of ``categorizing objects which are almost alike" in some sense.
Matrix similarity also conveniently tells us that two matrices will share the same eigenvalues, which is the property we happen to be after.
This is the main reason I expect this conjecture to be true, though admittedly, if it were false, then I have drastically more work to do. \npgh

I suspect this conjecture would not be too hard to prove due to a key observation.
Graph isomorphism can be interpreted as a relabelling of vertices, which translates to simply switching two rows (and their corresponding columns) for their adjacency matrices.
This, in turn, translates to multiplying by an elementary matrix which is its own inverse. This precisely satisfies the conditions for matrix similarity, thus giving us our conjecture right away. \npgh

For counterexamples, the best candidate would be two isomorphic graphs whose isomorphism is difficult to treat as a ``relabelling of indices".
I suspect the most likely candidates for these would be pairs of isomorphic graphs with very few possible isomorphisms, or two graphs with significant ``dissimilarity".

\newpage
\begin{mylm}[type=Conjecture 2]
Let $ G $ be a simple, undirected, unweighted graph. Denote the adjacency matrix of $ G $ as $ A_{G} $. \vsp
Then, $ G $ is a regular graph if all eigenvalues of $ A_{G} $ are integers.
\end{mylm}

This is one of two conjectures which arose from investigation of a specific graph.
The graph in question is the complete graph on 4 vertices, $ K_{4} $, and all possible subgraphs of $ K_{4} $ up to isomorphism.
Listing out all the eigenvalues for these graphs revealed this pattern. \vsp
This conjecture originally began as an ``if and only if" statement.
However, some of my early work involved the examination of specific types of graphs for many values of $ n $, such as all cycle graphs on $ n $ vertices, with $ n $ ranging from 2 to 20 (inclusive).
Many of these graphs do not have strictly integer eigenvalues, and they are all indeed 2-regular, thus disproving the forward direction.
However, seeing as the pattern does indeed hold for all subgraphs of $ K_{4} $, I do believe it remains true under some additional condition such as divisibility. \npgh

This conjecture to me seems like it will be somewhat difficult to prove, but can be done perhaps with enough ``machinery", so to speak.
I imagine the proof has two main parts: relating the adjacency matrix of a regular graph to its characteristic polynomial, and relating that polynomial to its roots.
In particular, it likely involves significant algebra which I myself might not be familiar with. \npgh

Since I suspect that for a $ d $-regular graph on $ n $ vertices, the divisibility of $ n $ by $ d $ likely plays an important factor, then it might be worth examining graphs of varying such categories.
For instance, a 3-regular graph on 7 vertices and a 4-regular graph on 10 vertices may yield useful insights.

\newpage
\begin{mylm}[type=Conjecture 3]
Let $ G $ be a simple, undirected, unweighted graph on $ n $ vertices, for some $ n \in \bb{N} $. Denote the adjacency matrix of $ G $ as $ A_{G} $. \vsp
Let $ \lambda $ be any eigenvalue for $ A_{G} $. Then $ \abs{\lambda} \leq n - 1 $. 
Furthermore, if $ G $ is \textbf{not} the complete graph on $ n $ vertices, then $ \abs{\lambda} < n - 1 $.
\end{mylm}

This is the second of two conjectures rising from the investigation of $ K_{4} $ and its subgraphs.
Since my software which I originally used to calculate eigenvalues gave exact values, many eigenvalues were being represented as complex numbers, despite being real-valued.
Having forgotten that some real numbers only have complex representations, I found this very frustrating.
However, during my investigation of $ K_{4} $ and its subgraphs, I decided to list the approximations of the eigenvalues, and I realized they all satisfied this property. \npgh

I am fairly confident that this conjecture is true, since the degree of a vertex is significantly related to the number of 1s in the adjacency matrix. \npgh

\end{document}
