\subsection{Review, Metric, and Normed Vector Spaces}

\lecdate{Lec 1 - Jan 6 (Week 1)}

we start w some review, particularly of $\bR$.
recall the least upper bound property:
for any $S\subseteq \bR$, $M$ is an upper bd for $S$ if for all $x\in S$
we have $x\leq M$.

\textbf{fact:} any nonempty $S\subseteq\bR$ bdd above has a least upper bd.

\begin{thm}[title=Archimedean Property]
    If $a<b\in\bR$ are distinct, then $\exists \, q\in\bQ$ such that
	$a<q<b$.
\end{thm}

\begin{pf}[source=Primary Source Material]
    sps wlog $0<a<b$. let $M\in\bN$ such that:
	\begin{equation*}
	    M>\frac{1}{b-a} \ \implies \ M(b-a)>1
	\end{equation*}
	let $N\in\bN$ be the largest s.t. $N\leq Ma$.
	then $q=\frac{N+1}{M}$ satisfies $a<q<b$. indeed:
	\begin{equation*}
	    N+1>Ma\ \implies \ a<\frac{N+1}{M} \qquad
		Mb>Ma+1\geq N+1 \ \implies \ b>\frac{N+1}{M}
	\end{equation*}
\end{pf}

genuinely hes just reviewing 157. why.
at least hes moved on to metric spaces at [checks watch] 9:47.
its now 10:00 and hes defining a nvs.

\lecdate{Lec 2 - Jan 8 (Week 1)}

$\ell ^{p}$ spaces. we know these. note $p\in[1,\infty]$.
\begin{equation*}
    \ell^p=\set{(a_{n}):\sum_{n}\abs{a_{n}}^{p}}<\infty,p<\infty \qquad
	\ell^\infty=\set{(a_{n}):\trm{sup}_{n}\abs{a_{n}}<\infty}
\end{equation*}
we know these norms. check $\ell^p$ is a vector space:
\begin{gather*}
	\abs{a_{n}+b_{n}}^{p} \ \leq \ (\abs{a_{n}}+\abs{b_{n}})^{p} \ \leq \
	(2\max(\abs{a_{n}},b_{n}))^{p}
	\ \leq \ 2^{p}(\abs{a_{n}}^{p}+\abs{b_{n}}^{p}) \\
	\implies \ \sum\abs{a_{n}+b_{n}}^{p}<\infty
\end{gather*}

now we claim $p$-norm is a norm.
most important is triangle inequality.

\begin{lm}[title=Young's Inequality]
    if $x,y\geq0$, then:
	\begin{equation*}
	    xy\leq \frac{x^{p}}{p}+\frac{y^{q}}{q}
	\end{equation*}
	for conjugate exponents $p,q\in(1,\infty)$.
\end{lm} \

\begin{exr}[type=Proof.,num=\relax]
    exercise; consequence of the fact that $s \mto \ell ^{s}$ is cvx.
	apparently.
\end{exr} \

\begin{thm}[title=Holder's Inequality]
    sps $p,q\in[1,\infty]$ s.t. $1/p+1/q=1$ (conjugate exponents).
	let $a\in\ell^p$, $b\in\ell ^{q}$. then:
	\begin{equation*}
	    \sum\abs{a_{n}b_{n}} \leq \norm{a}_{p}\norm{b}_{q}
	\end{equation*}
\end{thm}

\begin{pf}[source=Primary Source Material]
    case $p=1,q=\infty$. then:
	\begin{equation*}
	    \sum \abs{a_{n}b_{n}}\leq \sum\abs{a_{n}}\trm{sup}_{n}\abs{b_{n}}
		= (\trm{sup}_{n}\abs{b_{n}})\sum\abs{a_{n}}
		=\norm{a}_{1}\norm{b}_{\infty}
	\end{equation*}
	case $p,q\in(1,\infty)$: wlog, assume $\norm{a}_{p}=\norm{b}_{q}=1$.
	apply young's:
	\begin{gather*}
	    \abs{a_{n}b_{n}}=\abs{a_{n}}\abs{b_{n}}
		\leq\frac{\abs{a_{n}}^{p}}{p}+\frac{\abs{b_{n}}^{q}}{q} \\
		\sum\abs{a_{n}b_{n}}\leq p^{-1}\sum\abs{a_{n}}^{p}
		+q^{-1}\sum\abs{b_{n}}^{q}
		=\frac{\norm{a}^{p}_{p}}{p}+\frac{\norm{b}^{q}_{q}}{q}
		=p^{-1}+q^{-1}=1
	\end{gather*}
\end{pf}

\begin{thm}[title=Minkowski's Inequality]
    let $p\in[1,\infty]$, $a,b\in\ell^p$.
	then $\norm{a+b}_{p}\leq\norm{a}_{p}+\norm{b}_{p}$.
\end{thm} \

\begin{crll}
    $\ell^p$ is a nvs.
\end{crll} \

\begin{pf}[source=Primary Source Material]
	$p=1,\infty$ easy, so sps $p\in(1,\infty)$. then:
	\begin{equation*}
	    \sum\abs{a_{n}+b_{n}}^{p}=\sum\abs{a_{n}+b_{n}}\abs{a_{n}+b_{n}}^{p-1}
		\leq\sum\abs{a_{n}}\abs{a_{n}+b_{n}}^{p-1}
		+\sum\abs{b_{n}}\abs{a_{n}+b_{n}}^{p-1}
	\end{equation*}
	note:
	\begin{equation*}
	    p^{-1}+q^{-1}=1 \ \implies \ q^{-1}=1-p^{-1}=\frac{p-1}{p}
	\end{equation*}
	apply holder's to first sum:
	\begin{equation*}
	    \sum\abs{a_{n}}\abs{a_{n}+b_{n}}^{p-1}
		\leq\norm{a}_{p}\left( \sum\abs{a_{n}+b_{n}}^{(p-1)q} \right)^{1/q}
		=\norm{a}_{p}\left( \sum\abs{a_{n}+b_{n}}^{p} \right)^{1-p^{-1}}
	\end{equation*}
	applying to both sums gives:
	\begin{gather*}
		\sum\abs{a_{n}+b_{n}}^{p}
		\ \leq \ (\norm{a}_{p}+\norm{b}_{p})
		\left( \sum\abs{a_{n}+b_{n}}^{p} \right)^{1-p^{-1}}
		\ = \ \left( \norm{a}_{p}+\norm{b}_{p} \right)\norm{a+b}^{p-1}_{p} \\
		\norm{a+b}_{p}^{p}\leq(\norm{a}_{p}+\norm{b}_{p})\norm{a+b}_{p}^{p-1}
		\ \implies \ \norm{a+b}_{p}\leq\norm{a}_{p}+\norm{b}_{p}
	\end{gather*}
\end{pf} \

\begin{exr}[source=Primary Source Material]
    prove that if $p<q$ then $\ell^p\subsetneq\ell ^{q}$.
	hint: consider $\sum1/n^{s}$ for some good $s$.
\end{exr} \

\begin{xmp}[source=Primary Source Material]
    $C[0,1]$, the nvs of cts $f:[0,1]\gto\bR$.
	$p$-norm for $p\in[1,\infty]$ entirely analogous.
\end{xmp} \

\begin{exr}[source=Primary Source Material,title=HW 1.4]
    repeat holder, minkowski pfs to show:
	\begin{equation*}
	    \norm{fg}_{1}\leq\norm{f}_{p}\norm{g}_{q} \qquad
		\norm{f+g}_{p}\leq\norm{f}_{p}+\norm{g}_{p}
	\end{equation*}
\end{exr}

``sequences and convergence" are we just doing topology again.
well now we're doing continuity, specifically seq continuity.
so yes, we're just doing topology again.

hint (idea?) for hw3, basically versions of 1-x but seq $\sto$ more curve.

\lecdate{Lec 3 - Jan 13 (Week 2)}

some equivs btwn continuity, bdries, i forgot what else.
evidently im not rly paying attn

oh equiv of metrics/norms.
note that if $p<q$, then $\ell^p\subsetneq\ell ^{q}$ so
$\norm{\cdot}_{p}$ and $\norm{\cdot}_{q}$ not equiv.


