\subsection{Metric Space Topology}

\lecdate{Lec 4 - Jan 15 (Week 2)}
tdy was a snow day so going off posted notes but i still think
nothings happened.

\begin{xmp}[source=Primary Source Material]
    $C([0,1])$ is complete under sup norm.

	let $f_{n}$ be cauchy. for all $x\in[0,1]$, we have:
	\begin{equation*}
	    \abs{f_{n}(x)-f_{m}(x)}\leq\sup_{x}\abs{f_{n}(x)-f_{m}(x)}
		=\norm{f_{n}-f_{m}}_{\infty}
	\end{equation*}
	thus, $(f_{n}(x))$ is cauchy for fixed $x$, so $f_{n}(x)\sto f(x)$ as
	$n\sto\infty$ for fixed $x$.
	want to show that $f\in C([0,1])$ and $\norm{f-f_{n}}_{\infty}\sto\infty$
	as $n\sto\infty$.

	let $\ep>0$. then there exists $N>0$ s.t. for all $n,m\geq N$,
	$\norm{f_{n}-f_{m}}_{\infty}<\ep$.
	since $f_{n}$ is cauchy, for all $x$, we have for $n\geq N$:
	\begin{equation*}
		\abs{f(x)-f_{n}(x)}=\lim_{m\sto\infty}\abs{f_{m}(x)-f_{n}(x)} \leq
		\lim_{m\sto\infty}\left(\sup_{x}\abs{f_{m}(x)-f_{n}(x)}\right) <\ep
	\end{equation*}
	thus for all $n\geq N$, $\sup_{x}\abs{f(x)-f_{n}(x)}<\ep$, so
	$f_{n}$ cvgs uniformly to $f$.

	now we show $f$ cts. let $\ep>0$ and $x\in[0,1]$.
	then there exists $N$ such that \\
	$\sup_{y}\abs{f(y)-f_{N}(y)}<\ep$.
	since $f_{N}$ cts, $\ex\delta>0$ s.t. $\fa y\in[0,1]$ w $\abs{x-y}<\delta$,
	we have $\abs{f_{N}(x)-f_{N}(y)}<\ep$. thus:
	\begin{equation*}
	    \abs{f(x)-f(y)} \ \leq \
		\abs{f(x)-f_{N}(x)}+\abs{f_{N}(x)-f_{N}(y)}+\abs{f_{N}(y)-f(y)}
		\ \leq \ 3\ep
	\end{equation*}
	this concludes the proof.
\end{xmp}

lol ok thats it.

\lecdate{Lec 5 - Jan 20 (Week 3)}

still more cpt stuff ...

\begin{defn}
    we say $f:M\gto N$ is \textbf{uniformly cts} if $\fa\ep>0$,
	$\ex\delta>0$ s.t. $\fa x,y\in M$, w/
	\begin{equation*}
	    d_{M}(x,y)<\delta \ \implies \ d_{N}(f(x),f(y))<\ep
	\end{equation*}
	notice the distinction btwn this and regular continuity.
\end{defn} \

\begin{thm}
    if $f:M\gto N$ cts and $M$ cpt, then $f$ uni. cts.
\end{thm} \

\begin{pf}[source=Primary Source Material]
    pick a seq $(x_{n},y_{n})\in M^{2}$ s.t.
	\begin{equation*}
	    d_{M}(x_{n},y_{n})\sto0
	\end{equation*}
	since $M^{2}$ cpt, $\ex(x_{n_{k}},y_{n_{k}})\sto(x,y)$. then:
	\begin{equation*}
	    d(x,y) \ \leq \ d(x,x_{n_{k}})+d(x_{n_{k}},y_{n_{k}})+d(y_{n_{k}},y)
		\ \implies \ d(x,y)=0
	\end{equation*}
	so $x=y$. therefore:
	\begin{equation*}
	    d(f(x_{n_{k}}),f(y_{n_{k}})) \ \leq \
		d(f(x_{n_{k}}),f(x)) + d(f(y),f(y_{n_{k}})) \ \implies \
		d(f(x_{n_{k}}),f(y_{n_{k}})) = 0
	\end{equation*}
	since $f$ cts.
\end{pf}

oh hey connectedness

\lecdate{Lec 6 - Jan 22 (Week 3)}
more connectedness ig

\newpage
\begin{thm}
    $\bR$ is conn
\end{thm}

\begin{pf}[source=Primary Source Material]
    let $U\subseteq\bR$ be non-empty clopen. wts $U=\bR$.

	let $p\in U$ and defn:
	\begin{equation*}
	    X:=\set{x\in U:(p,x)\subseteq U}
	\end{equation*}
	$X$ non-empty since $U$ open, $p\in U$. claim: $\sup X=\infty$.
	\begin{block}
	    sps otw, let $s:=\sup X<\infty$. we show $s \in X$.

		by defn, $\ex x_{n}\sto s$.
		then $\fa n, (p,x_{n})\subseteq U$, so:
		\begin{equation*}
			(p,s)=\bigcup_{n}(p,x_{n})\subseteq U
		\end{equation*}
		since $x_{n}\in U$ and $U$ closed, $s \in U$ so $s \in X$.
		since $U$ open, $s + \ep\in X$ for all $\ep>0$.
	\end{block}
	thus, we must have $(p,\infty)\subseteq U$.
	by a symmetric argument, $(-\infty,p)\subseteq U$. we have
	$ (-\infty,p)\cup\set{p}\cup(p,\infty)\subseteq U $, so $U=\bR$.
\end{pf}

``for culture".

one place conn is used: pdes, known as \textit{method of continuity}.

generally, want to find soln to eq:
\begin{equation*}
    Lu:=\sum a_{ij}\frac{\p}{\p x_{i}}\frac{\p}{\p x_{j}}u = f
\end{equation*}
given $f:\bR^{n}\gto\bR$, find $u$ s.t. $Lu=f$.
for(?) $a=(a_{ij})$ is elliptic, $\lambda^{-1}\leq \trm{?}\leq\lambda$
for some $\lambda>0$.
when $a_{ij}=\delta_{ij}$, ie $a=\trm{id}$, this gives poissons eq'n.

for $t\in[0,1]$, define:
\begin{gather*}
    L_{t}=t\sum	_{i}\frac{\p^{2}}{\p x_{i}^{2}}
	+(1-t)\sum_{i,j}a_{ij}\frac{p}{\p x_{i}}\frac{\p}{\p x_{j}} \\
	I=\set{t\in[0,1]:L_{t}u=f \trm{ has soln}}
\end{gather*}
wts $I=[0,1]$. let ?? $\in I$, try to show ??? clopen.

anyway.

\begin{defn}
    let $S\subseteq M$. $p\in M$ is a \textbf{cluster pt} of $S$ if
	$\fa r>0$, $\abs{B_{p}(r)\cap S}=\infty$.
\end{defn}

\begin{thm}
    tfae
	\begin{enumerate}
	    \item $\ex(x_{n})\in S$ of distinct pts cvg to $p$
		\item $\fa r>0$, $\abs{B_{p}(r)\cap S}=\infty$
		\item $\fa r>0$, $\abs{B_{p}(r)\cap S}\geq2$
		\item $\fa r>0$, $B_{p}(r)\cap S$ contains a pt $\neq p$
	\end{enumerate}
\end{thm}

\begin{pf}[source=Primary Source Material]
    clearly $1 \ \implies \ 2 \ \implies \ 3 \ \implies \ 4$.
	we show $4 \ \implies \ 1$.

	let $x_{1}\in B_{p}(1)\cap S$ s.t. $x_{1}\neq p$.
	then $r_{1}<d(x_{1},p)$.
	let $x_{2}\in B_{p}(r_{1}\cap S)$ s.t. $x_{2}\neq p$.
	note $x_{2}\neq x_{1}$ since $r_{1}<d(x_{1},p)$.
	take $r_{2}<\min(d(x_{2},p),1/2)$, induct.
\end{pf}

\begin{defn}
    a metric space is \textbf{perfect} if every pt is a cluster pt.
\end{defn}

\begin{thm}
    complete non-empty perf metric sp is unctbl.
	(crll: $\bR$ is unctbl)
\end{thm} \

\begin{pf}[source=Primary Source Material]
    sps $M=\set{x_{1},x_{2},\dots}$ ctbl.
	we construct a seq of closed sets of the form:
	\begin{equation*}
		Y_{i}:=\bar{B_{y_{i}}}(r_{i})
	\end{equation*}
	s.t. $\fa n$:
	\begin{enumerate}
	    \item $x_{n}\notin Y_{n}$
		\item $Y_{n+1}\subseteq Y_{n}$
		\item $r_{n}\leq1/n$
	\end{enumerate}
	assuming the construction, $(y_{n})_{n}$ is cauchy. sps $y_{n}\sto y$.
	by 2, since we have that $y_{m}\in Y_{n}$ $\fa m\geq n$, then
	$y\in Y_{n}$ $\fa n$.
	but $x_{n}\notin Y_{n}$ so $y\neq x_{n}$ $\fa n$, contradiction.

	we now construct such a seq of sets.
	let $y_{1}\in B(x_{1},1)$ s.t. $y_{1}\neq x_{1}$.
	next, define $r_{1}<\min(d(y,x_{1}),1)$.
	then $Y_{1}=\bar{B_{y_{1}}}(r_{1})$ satisfies $x_{1}\notin Y_{1}$.

	for any $k$, choose $y_{k}\in B(y_{k-1},r_{k-1})$ s.t. $y_{k}\neq x_{k}$.
	choose $r_{k}>0$ s.t.:
	\begin{equation*}
	    r_{k}<d(y_{k},x_{k}) \qquad
		r_{k}\leq1/k \qquad
		B(y_{k},r_{k})\subseteq B(y_{k-1},r_{k-1})
	\end{equation*}
	then $x_{k}\notin\bar{B}(y_{k},r_{k})$.
\end{pf} \

\begin{xmp}[title=Cantor Set]
	see below
\end{xmp}
as a fun ex of a perf space sth sth containing no interval,
we construct the std middle thirds cantor set.

oh he drew some pictures
\begin{equation*}
    A_{n}:=[0,1]\sm
	\left(
	\bigcup_{j=0}^{n-1}\left(\frac{1+3j}{3^{n}},\frac{2+3j}{3^{n}}\right)
	\right)
	\qquad
	C:=\bigcap_{n}A_{n}
\end{equation*}
properties of $C$:
\begin{enumerate}
    \item closed + cpt
	\item ``totally disconn"
	\item non-empty
	\item perf but contains no interval
	\item unctbl
\end{enumerate}
``totally disconn" means $\fa r>0, p\in C$, $\ex$ set $U$ clopen in $C$ s.t.
$U\subseteq B(p,r)\cap C$.

n.b.: ``proper" defn is that only singletons are conn.

\textbf{pf that $\mbf{C}$ non-empty:} consider endpts.

\textbf{pf that $\bm{C}$ unctbl:}
closed bdd subset of $[0,1]$, so complete + perf thus unctbl.

\textbf{pf that $\bm{C}$ contains no interval:}
sps $(\alpha,\beta)\subseteq C$.
then if $1/3^{n}<\abs{\beta-\alpha}$, $C_{n}$ is intervals of len $1/3^{n}$,
so $(\alpha,\beta)\subsetneq C_{n}$.
since $C\subseteq C_{n}$, $(\alpha,\beta)\subsetneq C$.

\textbf{pf that $\bm{C}$ totally disconn:}
let $r>0,p\in C$. choose $n$ s.t. $1/3^{n}<r$.
then $p\in C_{n}$ implies $\ex$ interval $I$ of len $1/3^{n}$ in $C_{n}$
s.t. $p\in I$. wts $U=I\cap C$ satisfies:
\begin{itemize}
    \item $U\subseteq B(p,r)\cap C$
	\item $U$ clopen in $C$
\end{itemize}
since $U=I\cap C$, $I$ closed interval, $U$ is thus closed in $C$.
write $I=[a,b]$ and choose $\ep<1/3^{n}$. then:
\begin{equation*}
	I\cap C \ = \ [a,b]\cap C \ = \ (a-\ep,b+\ep)\cap C \ = \ U
\end{equation*}
so $U$ open in $C$.

\textbf{pf that $\bm{C}$ perf:}
let $p\in C$ and $r>0$. wts $\abs{B(p,r)\cap C}\geq2$.
if $n$ s.t. $1/3^{n}<r$, then $p\in I$ is a closed interval of len $1/3^{n}$
in $C_{n}$.
[take endpts]


\lecdate{Lec 7 - Jan 27 (Week 4)}

defn: cvring cpt: topo defn of cpt.
thm: (seq) cpt iff cvring cpt iff complete + totally bdd.
note he phrased the 2nd one as a set $\subseteq M$ being cpt iff \textit{closed} + totally bdd,
with $M$ complete, rather than just complete + totally bdd.

\begin{defn}
    given metric sp $M$, a space $M'$ is a \textbf{completion} of $M$ if:
	\begin{enumerate}[i)]
		\item $\ex\iota:M\gto M'$ isometry onto $i(M)$
		\item $M'$ complete
		\item $\iota(M)$ dense
	\end{enumerate}
\end{defn}
note: he uses $\hat{M}$ but i dont feel like doing that so.

\begin{thm}
    every metric sp has a completion
\end{thm} \

\begin{pf}[source=Primary Source Material]
    sketch of ``real" proof.
	define the set of cauchy seqs in $M$ as:
	\begin{equation*}
	    \cl{C} \ := \ \set{(x_{n}):x_{n}\in M, (x_{n}) \trm{ is cauchy}}
	\end{equation*}
	defn:
	\begin{equation*}
		a\sim b \ \iff \ \lim d(a_{n},b_{n})=0 \qquad
		M':=\cl{C}/\sim \qquad
		D([a],[b]) \ := \ \lim d(a_{n},b_{n})
	\end{equation*}
	we check:
	\begin{enumerate}[i)]
		\item lim exists and is well-defn'd

			pf: skipped, exercise
		\item is a metric

			pf: triangle [basically] gets inherited, nothing to do
		\item $M'$ complete

			pf: uhh missing?
		\item inclusion is an isometry

			pf: inclusion given by constant seqs, isometry follows
		\item img of inclusion is dense

			pf: also missing
	\end{enumerate}
\end{pf}



