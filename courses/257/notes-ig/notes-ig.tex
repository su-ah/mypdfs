\documentclass{article}
\usepackage{preamble}
\usepackage{env}
\usepackage{configure}

% available environments:
% theorem: thm
% definition: defn
% proof: pf
% corollary: crll
% lemma: lm
% question: qu
% solution: soln
% example: xmp
% exercise: exr
%
% options: title=<title>   {all}
%          source=<source> {pf, qu, soln, xmp, exr}  Note: if content is taken directly from the main resource, cite the main resource as ``Primary source material"


% define these variables!
\def\coursecode{MAT257}
\def\coursename{Analysis III} % use \relax for non-course stuff
\def\studytype{2} % 1: Personal Self-Study Notes / 2: Course Lecture Notes / 3: Revised Notes / 4: Exercise Solution Sheet
\def\author{\me}
\def\createdate{September 4, 2024}
\def\updatedate{\today}
\def\source{Class Lectures} % name, ed. of textbook, or `Class Lectures` for class notes
\def\sourceauthor{Prof. Ehsaan Hossain} % for class notes, put lecturer
% \def\leftmark{} % set text in header; should only be necessary in assignments etc.
% \pagenumbering{arabic} % force revert numbering to default; should only be necessary in assignments etc.

\begin{document}

\cover
\toc
\blurb

\section{Introduction}
\subsection{Motivation and Preliminaries}

Lecture 3 - Sept 09

This course is a continuation of first-year calculus/analysis - so what do our worksheets have to dowith this?

The idea is that when we generalize concepts from first-year, we shift from single-variable to multi-variable. We want to examine how these generalizations work (or don't)! For example:

\begin{align*}
    \bb{R} & \rightarrow \bb{R}^{n} \\
    \abs{x} & \rightarrow \norm{x}_{2} \\
    \abs{x - y} & \rightarrow \norm{x - y}_{2}, \textrm{d}(x, y) \\
    (a, b) & \rightarrow B(p, r) \\
    [a, b] & \rightarrow \bar{B}(p, r) \\
    f: \bb{R} \rightarrow \bb{R} & \rightarrow f: \bb{R}^{n} \rightarrow \bb{R}^{m} \\
    \bb{Q} & \rightarrow \bb{Q}^{n} \\
    \lim_{x \rightarrow a} & \rightarrow \lim_{\hat{x} \rightarrow \hat{a}} \textrm{(change abs values to norms!)}
\end{align*}

\subsection{Congregating our results}

Lec 11 (Week 6) - Oct 7

\begin{defn}
    A function $ f: X \rightarrow Y $ is continuous if:
    \begin{equation*}
        \forall \, x_{0} \in X, \forall \, \ep > 0, \exists \, \delta > 0 \trm{ s.t. }
        d(x, x_{0}) < \delta \implies d(f(x), f(x_{0})) < \ep
    \end{equation*}
    A function is \textit{uniformly} continuous if:
    \begin{equation*}
        \forall \, \ep > 0, \exists \, \delta > 0 \trm{ s.t. } \forall \, x, y \in X,
        d(x, y) < \delta \implies d(f(x), f(y)) < \ep
    \end{equation*}
    Here, the same $ \delta $ works at \textit{every} point.
\end{defn}

Intuition for uniform continuity: a graph has somewhat bounded variation.

Proposition: uniformly cts implies cts.
Converse counterexample: $ f(x) = \frac{1}{x} $.

\newpage
\begin{thm}[title=Free Upgrade Theorem]
    If $ f : X \rightarrow Y $ is continuous and $ X $ is compact (or clustering),
    then $ f $ is uniformly continuous.
\end{thm}

\begin{pf}[source=Primary Source Material]
    Lebesgue's proof (open covers): \vsp
    %
    Let $ \ep > 0 $. Want $ \delta > 0 $ such that:
    \begin{equation*}
        \abs{x - y} < \delta \implies \abs{f(x) - f(y)} < \ep
    \end{equation*}
    By continuity, for all $ x \in [0, 1] $, we have $ \delta = \delta_{x} > 0 $ such that:
    \begin{equation*}
        \abs{x - y} < \delta_{x} \implies \abs{f(x) - f(y)} < \ep
    \end{equation*}
    for all $ y \in [0, 1] $.
    Idea: why not let $ \delta = \inf\set{\delta_{x} : x \in [0, 1]} $?
    Well, what if $ \delta = 0 $?
    But: $ [0, 1] $ is compact! So:
    \begin{equation*}
        [0, 1] \subseteq \bigcup_{x \in X} (x-\delta_{x}, x + \delta_{x})
        \implies [0, 1] \subseteq \bigcup_{k} B(x_{k}, \delta_{x_{k}})
    \end{equation*}
    has a finite subcover!
    Now, we claim that $ \delta = \min \set{\delta_{x_{1}}, \dots, \delta_{x_{k}}} $ works. \vsp
    %
    Suppose $ x, y $ are in the same $ U_{x_{j}} $, then done.
    Therefore, suppose $ x \in U_{x_{i}}, y \in U_{x_{j}} $. Then...?
\end{pf}

\begin{pf}[source=Primary Source Material]
    Bolzano-Weierstrass proof (convergent subsequence): \vsp
    %
    Let $ \ep > 0 $. Want to find $ \delta > 0 $ such that:
    \begin{equation*}
        d(x, y) < \delta \implies d(f(x), f(y)) < \ep
    \end{equation*}
    Suppose the contradiction.
    Then, $ \forall \, \delta > 0, \exists \, x, y \in X $ where $ d(x, y) < \delta $ but
    $ d(f(x), f(y)) \geq \ep $.
    We have $ x_{n}, y_{n} \in X $ such that
    $ d(x_{n}, y_{n}) < \frac{1}{n} $, but $ d(f(x_{n}) f(y_{n})) \geq \ep $.
    By clustering, $ \exists \, x_{n_{k}} \rightarrow p, y_{n_{k}} \rightarrow q $.
    WLOG, we can assume these indices are the same.
    Note:
    \begin{equation*}
        f(x_{n_{k}}) \rightarrow f(p) \qquad f(y_{n_{k}}) \rightarrow f(q)
    \end{equation*}
    By def of convergence, for all $ \alpha > 0 $, we can find $ K \geq 1 $ such that:
    \begin{equation*}
        d(x_{n_{k}}, p) < \frac{\alpha}{3} \qquad
        d(y_{n_{k}}, q) < \frac{\alpha}{3} \qquad
        d(x_{n_{k}}, y_{n_{k}}) < \frac{\alpha}{3}
    \end{equation*}
    So, we have that $ d(p, q) < \alpha \implies p = q $ for all $ k \geq K $. Now:
    \begin{gather*}
        f(x_{n_{k}}) \rightarrow f(p) \implies d(f(x_{n_{k}}), f_{p}) < \frac{\ep}{2} \\
        f(y_{n_{k}}) \rightarrow f(p) \implies d(f(y_{n_{k}}), f(p)) < \frac{\ep}{2} \\
        d(f(x_{n_{k}}), f(y_{n_{k}})) \leq d(f(x_{n_{k}}), f_{p}) + d(f(y_{n_{k}}), f(p)) < \ep
    \end{gather*}
\end{pf}

\begin{thm}
    The following are equivalent:
    \begin{itemize}
        \item Compact
        \item Clustering (Sequentially compact)
        \item Complete and totally bounded
    \end{itemize}
\end{thm}

\begin{pf}[source=Primary Source Material]
    $ 1 \implies 2 $.
    Suppose $ X $ is compact. Suppose $ (x_{n}) $ is a sequence with no cluster point.
    Let $ p \in X $. Since $ p $ is not a cluster point, then $ \exists \, \ep_{p} > 0 $ such that
    \begin{equation*}
        d(x_{n}, p) < \ep_{p}
    \end{equation*}
    for only finitely many $ n $. That is:
    \begin{equation*}
        I_{p} = \set{n \in \bb{N} : d(x_{n}, p) < \ep_{p}}
    \end{equation*}
    is finite. \vsp
    %
    Let $ U_{p} = B(p, \ep_{p}) $. Then $ \set{U_{p}}_{p \in X} $ is an open cover.
    By compactness, we have $ p_{1}, \dots, p_{n} $ giving a finite cover. \vsp
    %
    But this is a contradiction, because by Pigeonhole Principle,
    one of the $ U_{p_{i}} $ would have to contain infinitely many $ x_{n} $. \npgh

    $ 2 \implies 1 $. we know this proof :)
\end{pf}

\end{document}
