\documentclass{article}
\usepackage{preamble}
\usepackage{env}

% define these variables!
\def\coursecode{MAT257}
\def\leftmark{Week 14 - Integration} % set text in header; should only be necessary in assignments etc.
\pagenumbering{arabic} % force revert numbering to default; should only be necessary in assignments etc.

\DeclareMathOperator{\vol}{vol}

\makeatletter
% settings for toc alignment
%
% Configuration
% -------------
% Horizonal alignment in \numberline:
%   l: left-aligned
%   c: centered
%   r: right-aligned
% \nl@align@: Default setting
% \nl@align@<levelname>: Setting for specific level

\def\nl@align@{l}% default
\def\nl@align@section{r}

\makeatother

\begin{document}
\setcounter{subsection}{12}
\setcounter{exr}{12}

\begin{exr}
    Let $ E \subseteq \bb{R}^{n} $ be a box. Let $ P, Q $ be partitions of $ E $ such that $ Q $
    refines $ P $ ($ P_{i} \subseteq Q_{i} $ for each $ i $). \vsp
    %
    Show that $ U(f,P) \geq U(f,Q) $ and $ L(f,P) \leq L(f,Q) $.
\end{exr}

\begin{pf}
    Since $ P, Q $ both partition E, then:
    \begin{equation*}
        \sum_{j}\vol(P(j)) \ = \ \sum_{k}\vol(Q(k))
    \end{equation*}
    Furthermore, for any $ A \subseteq B \subseteq E $, we have that $ \trm{sup}_{A}(f) \leq
    \trm{sup}_{B}(f) $. Notice that for each $ Q(k) $, there exists a unique $ P(j) $ such that
    $ Q(k) \subseteq P(j) $. \vsp
    %
    Thus, reindex each $ Q(k) $ as $ Q(j, k_{j}) $, where $ j $ is the unique tuple such that
    $ Q(k) \subseteq P(j) $. Then, it follows that:
    \begin{align*}
        U(f, P) = \sum_{j}M_{j}\vol(P(j)) & \ = \sum_{j}M_{j} \left( \sum_{k_{j}}
        \vol(Q(j, k_{j})) \right) \\
                                          & \ = \sum_{j}\sum_{k}M_{j}\vol(Q(j,k_{j})) \\
                                          & \ \geq \sum_{j}\sum_{k}M_{k_{j}}\vol(Q(j,k_{j})) \\
                                          & \ = \sum_{k}M_{k}\vol(Q(k)) \\
                                          & \ = U(f, Q)
    \end{align*}
    Showing that $ L(f,P) \leq L(f, Q) $ is entirely analogous.
\end{pf}

\end{document}
