\documentclass{article}
\usepackage{preamble}
\usepackage{env}
\usepackage{configure}

\def\leftmark{Week 6 - Equivalence} % set text in header; should only be necessary in assignments etc.
\pagenumbering{arabic} % force revert numbering to default; should only be necessary in assignments etc.

\begin{document}

\setcounter{subsection}{6}
\setcounter{exr}{38}

\begin{exr}
    Prove that $ \norm{\cdot} \sim \norm{\cdot}' $ if and only if
    $ \norm{\cdot} \approx \norm{\cdot}' $.
\end{exr}

\begin{pf}[source=bestie :D]
    Suppose $ \norm{\cdot} \sim \norm{\cdot}' $. Consider the identity map defined as:
    \begin{gather*}
        I : (X, \norm{\cdot}) \rightarrow (X, \norm{\cdot}') \\
        I^{-1} : (X, \norm{\cdot}') \rightarrow (X, \norm{\cdot})
    \end{gather*}
    Since $ \norm{\cdot} \sim \norm{\cdot}' $, it follows that $ I, I^{-1} $ are topologically
    continous, and thus continuous. Since they are also trivially linear, they are therefore
    bounded. Then, there exist $ m, M \geq 0 $ such that:
    \begin{gather*}
        \norm{I(x)}' = \norm{x}' \leq m \norm{x} \\
        \norm{I^{-1}(x)} = \norm{x} \leq M \norm{x}'
    \end{gather*}
    Note that if either $ m $ or $ M $ is equal to $ 0 $, then both norms are the $ 0 $ function.
    Otherwise, we see that:
    \begin{equation*}
        \frac{1}{M}\norm{x} \leq \norm{x}' \leq m\norm{x}
    \end{equation*}
    So we have that $ \norm{\cdot} \approx \norm{\cdot}' $ as needed.
\end{pf}

\begin{pf}[source=Alan]
    For the converse, suppose $\Vert \cdot \Vert \approx \Vert \cdot \Vert^\prime$.
    This means there exists $\alpha > 0$ and $\beta > 0$ such that for all $x \in X$:
    $$
    \alpha \Vert x \Vert \leq \Vert x \Vert^\prime \leq \beta \Vert x \Vert
    $$

    Now, let $x \in X$. and for all $\ep > 0$, consider $B_{\Vert \cdot \Vert}(x, \ep)$. 
    Then note, for all $y \in B_{\Vert \cdot \Vert^\prime}(x, \alpha \ep)$, we have that:
    $$
    \Vert y - x \Vert \leq \frac{ \Vert y - x \Vert^\prime }{\alpha}
    < \frac{\alpha \ep}{\alpha} < \ep
    $$
    Therefore, $B_{\Vert \cdot \Vert^\prime}(x, \alpha \ep) \subseteq 
    B_{\Vert \cdot \Vert}(x, \ep)$. \vsp
    %
    Similiarily, for all $x \in X$ and for all $\ep > 0$,
    consider $B_{\Vert \cdot \Vert^\prime}(x, \ep)$. 
    Then note, for all $y \in B_{\Vert \cdot \Vert}(x, \frac{\ep}{\beta})$, we have that:
    $$
    \Vert y - x \Vert^\prime \leq  \beta \Vert y - x \Vert < \frac{\beta\ep}{\beta} < \ep
    $$
    Therefore, $B_{\Vert \cdot \Vert}(x, \frac{\ep}{\beta}) \subseteq B_{\Vert \cdot \Vert^\prime}(x, \ep)$.
    So then by Tutorial 2, we see $\Vert \cdot \Vert \sim \Vert \cdot \Vert^\prime$.
\end{pf}

\end{document}
