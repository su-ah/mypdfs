% q20
\newpage
\label{q20}
\begin{qu}[num=20]
    Let $ x $ be a sequence in $ (\ell^{\infty}, \norm{\cdot}_{\infty}) $.
    Prove that $ D_{x} $ is compact if and only if $ x_{n} \rightarrow 0 $.
\end{qu}

\begin{soln}
    First, suppose that $ x_{n} \rightarrow 0 $.
    We show that $ D_{x} $ is complete and totally bounded. \vsp
    %
    Notice that $ D_{x} $ is closed. To see this, let $ (p_{n}) $ be a sequence
    in $ D_{x} $ such that $ p_{n} \rightarrow p $ for some $ p $. We show that
    $ p \in D_{x} $. Indeed, we have that each coordinate gives us a sequence
    $ (\pi_{i}(p_{n}))_{n} $ in the interval $ [-x_{i}, x_{i}] $ (WLOG,
    $ x_{i} \geq 0 $). Then:
    \begin{equation*}
        p_{n}\rightarrow p \qquad \pi_{i}(p_{n}) \rightarrow \pi_{i}(p)
    \end{equation*}
    Since $ [-x_{i}, x_{i}] $ is closed, then $ \pi_{i}(p) \in [-x_{i}, x_{i}] $,
    and thus $ p \in D_{x} $ as needed. \vsp
    %
    Since $ D_{x} $ is a closed subset of a complete space, it is complete.
    To see that it is totally bounded, let $ \ep > 0 $, and set $ s = \norm{x} $.
    Now, consider the real interval $ (-s, s) $. Clearly, it is totally bounded,
    and so there exists a finite set of points such that:
    \begin{equation*}
        C = \set{c_{1}, c_{2}, \dots, c_{m}} \qquad
        (-s, s) \subseteq \bigcup_{c \in C} B(c, \ep)
        \vspace{-0.2in} % why is there sm space here
    \end{equation*}
    Consider the subset $ \cl{C} \subseteq \ell^{\infty} $ defined as:
    \begin{equation*}
        \cl{S} = \set{k_{i} : k_{i} = (c_{i}, c_{i}, \dots), i \in \bb{N},
        i \leq m}
    \end{equation*}
    Then, for any $ y \in D_{x} $, since $ \norm{y} \leq \norm{x} $, it follows
    that $ y \in B(k_{i}, \ep) $ for some $ k_{i} \in S $.
    Therefore, $ D_{x} $ is totally bounded, and therefore is compact as needed.
    \vsp
    %
    Now, suppose that $ x_{n} \cnot\rightarrow 0 $.
    Since $ x_{n} \cnot\rightarrow 0 $, then there exists some $ \ep > 0 $ such
    that $ \abs{x_{n_{i}}} \geq \ep $ for infinitely many $ n_{i} $.
    Define a sequence $ s_{j} \in D_{x} $ as:
    \begin{equation*}
        s_{j_{k}} = \begin{cases} \ep & j = k \\ 0 & j \neq k \end{cases}
    \end{equation*}
    Then, we see that $ (s_{j}) $ is a sequence in $ D_{x} $ which does not have
    a convergent subsequence, and so $ D_{x} $ is not sequentially compact
    as needed.
\end{soln}


% q21.1
\newpage
\label{q21}
\begin{qu}[num=21.1]
    Let $ (X, d) $ be a metric space, $ K \subseteq X $ compact.
    Prove that if $ C $ is a closed set disjoint from $ K $, then $ \dist(K, C)
    > 0 $.
\end{qu}

\begin{soln}
    Suppose otherwise. Then, for all $ \ep > 0 $, there exists $ a \in K, b \in
    C $ such that $ d(a,b) < \ep $. Define sequences $ \set{a_{n}} \subseteq K,
    \set{b_{n}} \subseteq C $ such that for all $ n $:
    \begin{equation*}
        d(a_{n},b_{n}) < \frac{1}{n}
    \end{equation*}
    Since $ K, C $ closed, then $ a_{n} \sto a \in K, b_{n} \sto b \in C $ for
    some $ a, b $. Then, we have that for any $ \ep > 0 $, there exists some
    $ m > 0 $ such that:
    \begin{equation*}
        d(a_{m}, b) < \frac{1}{m} < \ep
    \end{equation*}
    But this implies that $ a_{n}\sto b $, which is a contradiction.
\end{soln}

% q21.2
\begin{qu}[num=21.2]
    Suppose $ K $ is a closed set such that $ \dist(K, C) > 0 $ for all closed
    sets $ C $ disjoint from $ K $. Does it follow that $ K $ is compact?
\end{qu}

\begin{soln}
    No; consider $ K = [1, \infty) $. Let $ C $ be any disjoint closed set, and
    denote $ c = \sup(C) $. We have $ c \in C $, and since $ C $ is disjoint,
    then $ c < 1 $, so $ d(K, C) > 0 $. However, we clearly see that $ K $ is not
    compact as it is unbounded.
\end{soln}


% q22
\newpage
\label{q22}
\begin{qu}[num=22]
    Let $ c_{0} $ denote the set of sequences converging to zero.
    Prove that $ c_{0}^{*} \equiv \ell^{1} $.
\end{qu}

\begin{soln}
    We construct an isometry $ f:\ell^{1}\gto c_{0}^{*} $.
    For any $ c \in c_{0} $, we define $ f $ as:
    \begin{equation*}
        f(s) \ = \ \sum_{n=1}^{\infty}s_{n}c_{n} =: g_{s}(c)
    \end{equation*}
    First, we show that $ f(s) \in c_{0}^{*} $. Note that the series converges as
    $ s \in \ell^{1} $ and $ c \in c_{0} $. Linearity follows trivially. To see
    that $ f $ is bounded, notice:
    \begin{equation*}
        \abs{g_{s}(c)} \ = \ \abs{\sum_{n=1}^{\infty}s_{n}c_{n}} \ \leq \
        \sum_{n=1}^{\infty}\abs{s_{n}}\abs{c_{n}} \ \leq \ \sup_{n}(c_{n})
        \sum_{n=1}^{\infty}\abs{s_{n}} \ = \ \norm{s}\norm{c}
    \end{equation*}
    Therefore, we have that:
    \begin{equation*}
        \norm{g_{s}} \ = \ \sup\set{\abs{g_{s}(c)}:c\in c_{0},\norm{c}\leq1}
        \ \leq \ \norm{c}
    \end{equation*}
    It follows that $ f $ is bounded. Next, we show that $ f $ is an isometry.
    Fix $ s \in \ell^{1} $ and $ N \in \bN $. Define a sequence $ c \in c_{0} $
    as follows:
    \begin{equation*}
        c_{n} \ = \
        \begin{cases}
            \frac{s_{n}}{\abs{s_{n}}} & s_{n}\neq0 \trm{ and } n \leq N \\
            0 & s_{n}=0 \trm{ or } n > N
        \end{cases}
    \end{equation*}
    Notice that $ \norm{x} \leq 1 $. Then, as $ N \sto \infty $, we see that:
    \begin{equation*}
        g_{s}(c) \ = \ \sum_{n=1}^{\infty}s_{n}c_{n} \ = \ \sum_{n=1}^{\infty}
        \abs{s_{n}} \ = \ \norm{s}
    \end{equation*}
    Since this is true of any $ s $, we have that $ f $ is an isometry as needed.
    Finally, we show $ f $ is surjective. Indeed, fix $ h \in c_{0}^{*} $. Take a
    basis of $ c_{0} $:
    \begin{equation*}
        \beta \ = \ \set{e_{n}} \qquad
        (e_{n})_{i} \ = \ \delta_{ni}
    \end{equation*}
    In other words, $ \beta $ is the basis of indicator sequences. Then, since
    $ h $ linear, we have:
    \begin{equation*}
        h(c) \ = \ \sum_{n=1}^{\infty}c_{n}h(e_{n})
    \end{equation*}
    Define a sequence $ s \in \ell^{1} $ as $ s_{n} \ = \ h(e_{n}) $.
    Then, for any $ c \in c_{0} $, it follows that:
    \begin{equation*}
        f(s) \ = \ \sum_{n=1}^{\infty}s_{n}c_{n} \ = \ \sum_{n=1}^{\infty}
        h(e_{n})c_{n} \ = \ h(c)
    \end{equation*}
    Note that by our previous work, this also implies that $ s \in \ell^{1} $.
    Thus, we have that $ f $ is a bijective isomorphism, and so we conclude that
    $ c_{0}^{*} \equiv \ell^{1} $ as needed.
\end{soln}


% q23.1
\newpage
\label{q23}
\begin{qu}[num=23.1]
    Find an explicit formula for $ \Phi : S^{2} \setminus \set{(0, 0, 1)} \gto
    \bb{R}^{2} $, given as the stereographic projection of the unit $ 2 $-sphere
    onto $ \bb{R}^{2} $.
\end{qu}

\begin{soln}
    First, consider the problem reduced by one dimension.
    We want to find $ \Phi(P) $ satisfying:

    \centering
    \scalebox{.95}{\incfig{stereoproject2d}}
    \flushleft

    Here, we can use properties of similar triangles to derive that
    $ \dfrac{\Phi(P)}{1} = \dfrac{x}{1-y} $.
    To extend this up a dimension to our original problem, we see that:
    
    \centering
    \scalebox{.75}{\incfig{stereoproject3d}} \\
    \small Still unsure how these diagrams are getting to school... Bad joke?
    \flushleft
    
    \normalsize
    In other words, we can project the point $ P $ onto the $ xz $-plane and
    $ yz $-plane, then projecting those points onto the $ xy $-plane in order to
    get the $ x $-component and $ y $-component respectively of $ \Phi(P) $. As a
    formula, we get:
    \begin{equation*}
        \Phi(x, y, z) = \left( \frac{x}{1-z}, \frac{y}{1-z} \right)
    \end{equation*}
\end{soln}

% q23.2
\begin{qu}[num=23.2]
    Deduce that $ \Phi $ is continuous.
\end{qu}

\begin{soln}
    We can write $ \Phi $ as:
    \begin{equation*}
        \Phi(x, y, z) = (\Phi_{1}(x,y,z), \Phi_{2}(x,y,z))
    \end{equation*}
    Then, since each $ \Phi_{i} $ is clearly continuous, it follows that $ \Phi $
    is continuous as well.
\end{soln}

% q23.3
\vspace{-0.1in}
\begin{qu}[num=23.3]
    Given $ p = (s, t) \in \bb{R}^{2} $, find an explicit formula for
    $ \Phi^{-1}(p) $.
\end{qu}

\begin{soln}
    We want to solve for $ x, y, $ and $ z $ such that:
    \begin{equation*}
        s = \frac{x}{1-z} \qquad t = \frac{y}{1-z} \qquad x^{2}+y^{2}+z^{2}=1
    \end{equation*}
    First, notice we that $ x = (1-z)s, y = (1-z)t $.
    Then, rearranging the equation of the unit sphere, we get that:
    \begin{align*}
        & x^{2} + y^{2} = 1 - z^{2} \\
        \implies \ & x^{2} + y^{2} = (1-z)(1+z) \\
        \implies \ & (1-z)^{2}s^{2} + (1-z)^{2}t^{2} = (1-z)(1+z) \\
        \implies \ & s^{2} + t^{2} = \frac{1+z}{1-z}
    \end{align*}
    Denoting $ N = s^{2} + t^{2} $, and rearranging, we get that:
    \begin{equation*}
        1 + z = N(1 - z) \ \implies \ z = \frac{N - 1}{N + 1}
    \end{equation*}
    Finally, substituting this value of $ z $ into our other equations, we have:
    This gives us the final formula for the inverse as:
    \begin{equation*}
        \Phi^{-1}(s, t) = \left( \frac{2s}{s^{2} + t^{2} + 1},
        \frac{2t}{s^{2} + t^{2} + 1}, \frac{s^{2} + t^{2} - 1}{s^{2} + t^{2} + 1}
        \right)
    \end{equation*}
\end{soln}

% q23.4
\begin{qu}[num=23.4]
    Deduce that $ \Phi $ is a homeomorphism.
\end{qu}

\begin{soln}
    Similarly to above, we can see that each component of $ \Phi^{-1} $ is
    continuous, and thus $ \Phi^{-1} $ is continuous. \vsp
    %
    Since $ \Phi $ and $ \Phi^{-1} $ are both continuous, then $ \Phi $ is indeed
    a homeomorphism.
\end{soln}


% q24
\newpage
\label{q24}
\begin{qu}[num=24]
    Let $ X $ be a normed vector space. Prove the following are equivalent:
    \begin{enumerate}
        \item $ X $ is finite dimensional.
        \item The unit ball $ \bar{B}(0, 1) $ is compact.
        \item $ X $ is locally compact; that is, every point is contained in some
            open set with a compact closure.
   \end{enumerate}
\end{qu}

\begin{soln}
    $ (1 \implies 2) $
    Since $ X $ is finite-dimensional, then $ X $ is homeomorphic to
    $ \bb{R}^{n} $. This immediately tells us the closed unit ball is compact.
    \npgh

    $ (2 \implies 1) $
    We prove by contrapositive. Suppose $ X $ is infinite-dimensional. Consider
    the sequence given by $ \set{e_{i}}_{i=1}^{\infty} $, where $ e_{i} $ is the
    unit vector with a 1 in the $ i $-th coordinate, and $ 0 $ otherwise.
    Clearly each vector is in the closed unit ball, however this sequence has no
    convergent subsequence and so $ \bar{B}(0, 1) $ cannot be compact as needed.
    \npgh

    $ (2 \implies 3) $
    For any point $ p $, we can define the isometry:
    \begin{equation*}
        \Delta(x) = x + p
    \end{equation*}
    This is trivially an isometry, so we can take the unit ball centered at
    $ p $. This is isometric to the open unit ball, whose closure is compact, and
    so the closure of the open unit ball centered at $ p $ is also compact as
    needed. \npgh

    $ (3 \implies 2) $
    Suppose $ X $ is locally compact. Consider the unit ball. Then, there exists
    some open set $ O $ containing $ 0 $ and the closure is compact. Take a
    sufficiently small $ B(0, \ep)  $ such that the closure is contained within
    $ O $. Since it is a closed subset of a compact set, it must be compact, and
    since it is also homeomorphic to the unit ball, it is therefore also compact
    as needed.
\end{soln}


%% q25
%% TODO
%\newpage
%\begin{qu}[num=25]
%    Let $ \vphi: M_{n}(\bR) \gto M_{n}(\bR) $ be the function given by
%    $ \vphi(A) = A^{2} $. Find a linear approximation $ L_{A} $ to $ \vphi $ at
%    $ A \in M_{n}(\bR) $. Give an explicit formula for $ L_{A}(B) $ as a function
%    of $ B $, a proof that $ L_{A} $ is a bounded linear mapping, and a proof
%    that $ L_{A} $ is a linear approximation to $ \vphi $ at $ A $.
%\end{qu}
%
%
%% q26.1
%% TODO; has hint
%\newpage
%\begin{qu}[num=26.1]
%    Let $ X $ be a finite-dimensional normed vector space, $ U $ an open convex
%    subset, and $ f: U \gto \bR^{m} $ totally differentiable. \vsp
%    %
%    Suppose there exists some $ C \geq 0 $ such that $ \norm{f'(p)}_{\trm{op}}
%    \leq C $ for all $ p \in U $. Prove that:
%    \begin{equation*}
%        \norm{f(p) - f(q)}_{2} \leq C\norm{p-q}
%    \end{equation*}
%    for all $ p, q \in U $. Conclude that $ f $ is uniformly continuous.
%\end{qu}
%
%% q26.2
%% TODO
%\begin{qu}[num=26.2]
%    Prove that $ f'(p) = 0 $ for all $ p \in U $ if and only if $ f $ is a
%    constant function.
%\end{qu}
%
%% q26.3
%% TODO
%\begin{qu}[num=26.3]
%    Assume $ U = X $ and suppose $ f $ is twice differentiable.
%    Show that $ f'' = 0 $ if and only if $ f $ is affine-linear; that is, there
%    exists a bounded linear mapping $ M: X \gto Y $ and a vector $ b \in Y $
%    such that:
%    \begin{equation*}
%        f(p) = M(p) + b
%    \end{equation*}
%    for all $ p \in X $.
%\end{qu}


% q27.1
\newpage
\label{q27}
\begin{qu}[num=27.1]
    Let $ f : \bb{R}^{2} \gto \bb{R} $ be continuously differentiable.
    Show that the map $ \bb{R} \gto \bb{R} $ given by $ t \mto f(x, t) $ is
    integrable.
\end{qu}

\begin{soln}
    Since $ f $ is continuous, the map is also continuous and thus integrable.
\end{soln}

% q27.2
\begin{qu}[title=Feynman's Trick,num=27.2]
    Define the map $ \vphi: \bb{R}\gto \bb{R} $ by:
    \begin{equation*}
        \vphi(x) = \int_{a}^{b} f(x,t) \di t
    \end{equation*}
    Show that $ \vphi $ is differentiable, with derivative given by:
    \begin{equation*}
        \frac{\d\!\vphi}{\d x}(x_{0}) = \int_{a}^{b}\frac{\p f}{\p x} (x_{0}, t)
        \di t
    \end{equation*}
    This is known as differentiation under the integral sign, or sometimes
    Feynman's trick.
\end{qu}

\begin{soln}
    By definition, we have that:
    \begin{equation*}
        \frac{\d\!\vphi}{\d x} = \lim_{h\gto0}\frac{\int_{a}^{b}f(x+h,t)\di t
        - \int_{a}^{b}f(x,t)\di t}{h} \ = \
        \lim_{h\gto0} \frac{\int_{a}^{b}f(x+h,t)-f(x,t)\di t}{h}
    \end{equation*}
    This limit exists by FTC. Then, notice that:
    \begin{align*}
        \abs{\frac{\d\!\vphi}{\d x} - \int_{a}^{b}\frac{\p f}{\p x}(x, t)\di t}
        \ & = \ \abs{\frac{\int_{a}^{b}f(x+h,t)-f(x,t)\di t}{h} - \int_{a}^{b}
            \p_{x} f(x,t)\di t} \vsp
        & = \ \abs{\int_{a}^{b}\frac{f(x+h,t)-f(x,t)-h\p_{x}f(x,t)}{h}\di t} \vsp
        & \leq \ \int_{a}^{b}\abs{\frac{f(x+h,t)-f(x,t)}{h} - \p_{x}f(x,t)} \di t
    \end{align*}
    As $ h \gto 0 $, we have that the term in the absolute value is less than
    $ \ep $ for any $ \ep > 0 $, by definition of partial derivative. Therefore,
    we conclude that the two values are equal as needed.
\end{soln}

% q27.3
\begin{qu}[num=27.3]
    Use Feynman's trick to solve the single-variable integral:
    \begin{equation*}
        \int_{0}^{\infty} e^{-t^{2}} \di t
    \end{equation*}
\end{qu}

\begin{soln}
    We want to solve for $ I $, where $ I $ is the definite integral given by:
    \begin{equation*}
        I = \int_{0}^{\infty}e^{-t^{2}}\di t
    \end{equation*}
    By the hint, consider the function $ f:\bb{R}^{2}\gto\bb{R} $ given by:
    \begin{equation*}
        f(x,t) = \frac{e^{-x^{2}(1+t^{2})}}{1+t^{2}}
    \end{equation*}
    Note that as $ x \gto \infty $, $ f(x) \gto 0 $.
    Define the function $ \vphi:\bb{R}\gto\bb{R} $ as:
    \begin{equation*}
        \vphi(x) = \int_{0}^{\infty}f(x,t)\di t
    \end{equation*}
    Using Feynman's trick, we see that:
    \begin{align*}
        \frac{\d\!\vphi}{\d x} \ & = \ \int_{0}^{\infty} \frac{\p}{\p x}
        \frac{e^{-x^{2}(1+t^{2})}} {1+t^{2}} \di t \vsp
        & = \ \int_{0}^{\infty}\frac{-2x(1+t^{2})e^{-x^{2}(1+t^{2})}}{1+t^{2}}
        \di t \vsp
        & = \ -2e^{-x^{2}}\int_{0}^{\infty} xe^{-x^{2}t^{2}} \di t \vsp
        & = \ -2e^{-x^{2}}\int_{0}^{\infty}e^{-u^{2}}\di u \vsp
        & = \ -2e^{-x^{2}}I
    \end{align*}
    where we do a $ u $-substitution with $ u = xt $.
    Integrating back, we see that:
    \begin{gather*}
        \int_{0}^{\infty} \frac{\d\!\vphi}{\d x}\di x \ = \ \int_{0}^{\infty}
        -2e^{-x^{2}}I\di x \ \implies \ I^{2} \ = \ \frac{\vphi(0)}{2} \vsp
        \vphi(0) \ = \ \int_{0}^{\infty}f(0,t)\di t \ = \ \int_{0}^{\infty}
        \frac{1}{1+t^{2}} \di t \ = \ \arctan(x) \ \Big|_{0}^{\infty} \ = \
        \frac{\pi}{2}
    \end{gather*}
    Therefore, we have that:
    \begin{equation*}
        I^{2} \ = \ \frac{\pi}{4} \ \implies \ I \ = \ \frac{\sqrt{\pi}}{2}
    \end{equation*}
\end{soln}


%% q28.1
%% TODO
%\newpage
%\begin{qu}[num=28.1]
%    Let $ U \subseteq X $ be open and $ f: U \rightarrow Y $ be
%    twice continuously differentiable. \vsp
%    %
%    Suppose $ f : \bb{R}^{2} \rightarrow \bb{R} $ is given by $ f(x, y) = x^{2}
%    -xy+y^{2} $.
%    Find, with proof, an explicit formula for the linear mapping $ f''(2, -1) $.
%    Write down the matrix which represents this mapping with respect to a
%    suitable ``standard" basis.
%\end{qu}
%
%% q28.2a
%\begin{qu}[num=28.2(a)]
%    Now, suppose $ f : U \rightarrow \bb{R} $ where $ U \subseteq \bb{R}^{n} $ is
%    open. \vsp
%    %
%    Show that $ f $ is twice continuously differentable if and only if all second
%    partial derivatives exist and are continuous.
%\end{qu}
%
%% q28.2b
%\begin{qu}[num=28.2(b)]
%    Let $ f $ be twice continuously differentiable. Let $ v $ in $ \bb{R}^{n} $
%    and let $ D_{v}f : U \rightarrow \bb{R} $ be the directional derivative of
%    $ f $ along $ v $. Show that $ D_{v}f $ is continuously differentiable.
%\end{qu}
%
%% q28.2cd
%\begin{qu}[title=Clairaut's Theorem,num=28.2(cd)]
%    Show that the directional derivatives commute:
%    \begin{equation*}
%        D_{v}(D_{w}f) = D_{w}(D_{v}f) \quad \trm{ for all } i, j \in \set{1,
%        \dots, n} 
%    \end{equation*}
%    Then, deduce Clairaut's Theorem:
%    \begin{equation*}
%        \frac{\p^{2}f}{\p x_{i}\p x_{j}} = \frac{\p^{2}f}{\p x_{j}\p x_{i}} \quad
%        \trm{for all } v, w \in \bb{R}^{n}
%    \end{equation*}
%\end{qu}


% q29.1a
\newpage
\label{q29}
\begin{qu}[num=29.1(a)]
    Consider the tangent plane to the surface $ z = x + xy^{2} - y^{3} $ at
    $ p = (2,1,3) $. \vsp
    %
    First, fix $ x = 2 $ and set $ y = 1 + t $. Write $ z $ as a function of
    $ t $ and find $ z'(0) $. This is the slope in the $ y $ direction.
\end{qu}

\begin{soln}
    We substitute:
    \begin{equation*}
        z = 2 + 2(1+t)^{t}-(1+t)^{3} = 3+t-t^{2}-t^{3}
    \end{equation*}
    This gives us that $ z'(0) = 1 $, so we have the vector $ (1, 0, 1) $.
\end{soln}

% q29.1b
\begin{qu}[num=29.1(b)]
    Repeat the above to find the slope in the $ x $ direction.
\end{qu}

\begin{soln}
    We substitute:
    \begin{equation*}
        z = 2+t+2+t-1 = 3+2t
    \end{equation*}
    This gives us that $ z'(0) = 2 $, so we have the vector $ (0, 1, 2) $.
\end{soln}

% q29.1c
\newpage
\begin{qu}[num=29.1(c)]
    Given the two slopes, we now have two vectors which span a plane.
    Shift the plane to be tangent at $ p $.
    Write the plane in the form $ Ax+By+Cz=D $, where $ A, B, C, D \in \bb{R} $.
\end{qu}

\begin{soln}
    Given the previous two vectors, we find the normal vector using the cross
    product:
    \begin{equation*}
        n = (0, 1, 2) \times (1, 0, 1) = (2, 1, -1)
    \end{equation*}
    We set $ n = (A, B, C) $ and solve for $ D $ with $ p = (x,y,z) $:
    \begin{equation*}
        2(2)+1(1)-1(3)=2
    \end{equation*}
    So the equation of tangent plane at $ p $ is $ 2x+y-z=2 $.
\end{soln}

% q29.2
\newpage
\begin{qu}[num=29.2]
    Find with proof the equations of all planes in $ \bb{R}^{3} $ which are:
    \begin{itemize}
        \item tangent to the surface $ z = x+xy^{2}-y^{3} $
        \item parallel to the vector given by $ \vec v = (3, 1, 1) $
        \item pass through the point $ (-1, -2, 3) $
    \end{itemize}
\end{qu}

\begin{soln}
    First, we take the partial derivatives of $ z $ to get our basis vectors:
    \begin{equation*}
        \frac{\p z}{\p x} = 1+y^{2} \implies (1, 0, 1+y^{2}) \qquad
        \frac{\p z}{\p z} = 2xy-3y^{2} \implies (0, 1, 2xy-3y^{2})
    \end{equation*}
    We take the cross product to find the normal vector:
    \begin{equation*}
        (1,0,1+y^{2})\times(0,1,2xy-3y^{2})=(-1-y^{2},3y^{2}-2xy,1)
    \end{equation*}
    Next, we take the dot product with $ (3,1,1) $ to get a restriction on being
    parallel:
    \begin{equation*}
        -3-3y^{2}+3y^{2}-2xy+1=0 \ \implies \ -2-2xy=0 \ \implies \ xy=-1
    \end{equation*}
    Thus, any points at which such a plane can exist must satisfy $ xy=-1 $.
    Finally, we plug in $ (-1,-2,3) $ so that the plane will pass through the
    point:
    \begin{equation*}
        1+y^{2}+4xy-6y^{2}+3=-5y^{2} \ \implies \ D = -5y^{2}
    \end{equation*}
    So the general form of the plane looks like $ (-1-y^{2})x+(3y^{2}+2)y+z=
    -5y^{2} $. Lastly, to find all the valid points, since the planes must be
    tangent, we plug in a general point and solve:
    \begin{gather*}
        (-1-y^{2})x+(3y^{2}+2)y+z \ = \ -x-xy^{2}+3y^{3}+2y+x+xy^{2}-y^{3}
        \ = \ -5y^{2} \\
        \ \implies \ y(2y^{2}+5y+2) \ = \ 0
    \end{gather*}
    Since we must have $ xy=-1 $, then we know $ y\neq0 $. Using the quadratic
    equation, we find that the two possible values of $ y $ are
    $ y=\frac{-1}{2},-2 $. This gives us the two corresponding points and planes:
    \begin{gather*}
        \frac{-5}{4}x+\frac{11}{4}y+z=\frac{-5}{4}, \quad
        p=\left(2,\frac{-1}{2},\frac{21}{8}\right) \\
        -5x+14y+z=-20, \quad p=\left(\frac{1}{2},-2,\frac{21}{2}\right)
    \end{gather*}
\end{soln}

% q29.3
\newpage
\begin{qu}[num=29.3]
    Let $ f : \bb{R}^{2} \rightarrow \bb{R} $ be continuously differentiable, and
    define a new function $ g:\bb{R}^{2}\rightarrow\bb{R} $ by:
    \begin{equation*}
        g(x,y) = f(f(xy,x),f(y,xy))
    \end{equation*}
    Suppose that:
    \begin{itemize}
        \item $ f(3,1) = 5 $ and $ \nabla f(3,1) = (1, 2) $
        \item $ f(3,3) = 2 $ and $ \nabla f(3,3) = (0, -1) $
        \item $ g(1,3) = 6 $ and $ \nabla g(1,3) = (3, 4) $
    \end{itemize}
    Find $ \nabla f(5, 2) $.
\end{qu}

\begin{soln}
    Let $ a(x,y)=f(xy,x),b(x,y)=f(y,xy),h(x,y)=(a(x,y),b(x,y)) $. Then:
    \begin{equation*}
        g(x,y)=f(h(x,y))
    \end{equation*}
    We also notice that $ g(1,3)=f(h(1,3))=f(a(1,3),b(1,3))=f(5,2) $. We have:
    \begin{gather*}
        \nabla g(x,y) \ = \ \nabla(f\circ h)(x,y) \ = \
        (\nabla f\circ h(x,y))Jh(x,y) \vsp
        \ \implies \ Jh(x,y) \ = \ (\nabla a(x,y), \nabla b(x,y)) \vsp
        \ \implies \ Jh(1,3) \ = \ \begin{pmatrix}
            1 & 2 \\ 0 & -1
        \end{pmatrix} \vsp
        \ \implies \ \nabla g(1,3) \ = \ \nabla f(5,2)\cdot Jh \vsp
        (3,4) \ = \ (m,n)\begin{pmatrix}
            1 & 2 \\ 0 & -1
        \end{pmatrix} \vsp
        \ \implies \ (3,4)=(m,2m-n) \ \implies \ m = 3, n = 2
    \end{gather*}
    So the value of $ \nabla f(5,2) = (3,2) $ as needed.
\end{soln}
