\subsection{Surface Area of Spheres in Higher Dimensions}
The goal of this project is to find the surface area of spheres in
higher dimensions. Note that since translations do not alter the surface area,
we can always place our shape favourably.

We begin with the unit sphere. Recall that $ S^{2} $ can be parametrized by the
map:
\begin{equation*}
    f(\theta, \phi) \ = \ (\cos\theta\sin\phi, \sin\theta\sin\phi, \cos\phi)
\end{equation*}
where $ \theta $ is the usual angle in $ \bR^{2} $, and $ \phi $ measures the
``downward" angle formed with the $ z $-axis. We'll mimic this process as we
``add more dimensions". For instance, to generalize from $ S^{2} $ to $ S^{3} $,
we'll add another angle $ \rho $ which measures the ``falling" angle of a point
made with the new 4th axis, analogously to $ \phi $. Similarly to above, we see
that we should also thus have $ x_{4} = \cos\rho $.

Fix any $ x \in S^{3} $. Aside from $ x_{4} $, what do its other coordinates look
like? The set of points on $ S^{3} $ with $ x_{4} = \cos\rho $ for some fixed
$ \rho $ forms a copy of $ S^{2} $ on its own. This is perhaps best visualized
when $ \rho = \pi $, with $ S^{2} $ being the ambient sphere.

Note that this induced sphere is not necessarily a unit sphere; it has radius
given by $ \sin\rho $, which can be visualized as representing a point's distance
from the axis from which $ \rho $ is measured (for $ S^{2} $, for instance, this
is the $ z $-axis as previously mentioned).

Thus, the set of points on our induced sphere would have a coefficient by the
sphere's radius; in other words, each of the other coordinates would be
multiplied by $ \sin\rho $. Note that this is distinct from the ambient sphere's
radius, which we will represent with $ r $. Thus, we have a parametrization of
$ S^{3} $ as:
\begin{equation*}
    f(\theta,\phi,\rho) \ = \
    (\cos\theta\sin\phi\sin\rho,\sin\theta\sin\phi\sin\rho,
    \cos\phi\sin\rho,\cos\rho)
\end{equation*}
This pattern generalizes nicely; to make our lives slightly easier, however, we
will swap the first two coordinates. This allows for a neater pattern which does
not affect the overall parametrization or surface area. We can now tackle the
problem of finding the surface area of a general $ S^{n} $.

\newpage
Consider $ S^{n-1} \subseteq \bR^{n} $. Although $ S^{n-1} $ typically refers to
the unit sphere, we will consider any radius $ r $. We'll define our
parametrization by defining each of its components:
\begin{equation*}
    f = (f_{1}, \dots, f_{n}) \qquad
    f_{i}(\phi_{1},\dots,\phi_{n-1}) \ = \
    r\cos(\phi_{i-1})\prod_{k=i}^{n-1}\sin(\phi_{k})
\end{equation*}
Here, each $ \phi_{i} $ represents an angle, and we'll set $ \phi_{0} = 0 $ for
well-definedness. We know how to find the volume of a parametrized manifold;
before we begin this rather lengthy calculation, however, we will need a few
identities.
\begin{lm}
    Fix any natural $ n $. Then:
    \begin{align*}
        & \sum_{k=1}^{n}\left(\cos(\phi_{k-1})\prod_{j=k}^{n-1}\sin(\phi_{j})
        \right)^{2} \vsp 
        \ = \ & \prod_{j=1}^{n-1}\sin^{2}(\phi_{j}) \ + \ \cos^{2}(\phi_{1})
        \prod_{j=2}^{n-1}\sin^{2}(\phi_{j}) \ + \ \cos^{2}(\phi_{2})
        \prod_{j=3}^{n-1}\sin^{2}(\phi_{j}) \ + \ \dots \vsp
        \ + \ & \cos^{2}(\phi_{i-2})\sin^{2}(\phi_{i-1}) \ + \
        cos^{2}(\phi_{i-1}) \vsp
        \ = \ & (\sin^{2}(\phi_{1}) + \cos^{2}(\phi_{1}))
        \prod_{j=2}^{n-1}\sin^{2}(\phi_{j}) \ + \ \cos^{2}(\phi_{2})
        \prod_{j=3}^{n-1}\sin^{2}(\phi_{j}) \ + \ \dots \vsp
        \ + \ & \cos^{2}(\phi_{i-2})\sin^{2}(\phi_{i-1}) \ + \
        cos^{2}(\phi_{i-1}) \vsp
        \ = \ & (\sin^{2}(\phi_{2}) + \cos^{2}(\phi_{2}))
        \prod_{j=3}^{n-1}\sin^{2}(\phi_{j}) \ + \ \dots
        \ + \ \cos^{2}(\phi_{i-2})\sin^{2}(\phi_{i-1}) \ + \
        cos^{2}(\phi_{i-1}) \\
        \vdots \ & \vsp
        \ = \ & \sin^{2}(\phi_{i-2})\sin^{2}(\phi_{i-1}) \ + \
        \cos^{2}(\phi_{i-2})\sin^{2}(\phi_{i-1}) \ + \ \cos^{2}(\phi_{i-1}) \vsp
        \ = \ & \sin^{2}(\phi_{i-1}) \ + \ \cos^{2}(\phi_{i-1}) \vsp
        \ = \ & 1
    \end{align*}
    We also have:
    \begin{align*}
        \int\sin^{n}x\di x \ = \ & \int\sin^{n-1}x\cdot\sin x\di x \vsp
        \ = \ & -\sin^{n-1}x\cos x \ + \
        \int\cos x(n-1)\sin^{n-2}x \cos x\di x \vsp
        \ = \ & -\sin^{n-1}x\cos x \ + \
        (n-1)\int(1-\sin^{2}x) \sin^{n-2}x\di x \vsp
        \ = \ & -\sin^{n-1}x\cos x \ + \
        (n-1)\int\sin^{n-2}x\di x \ - \
        (n-1)\int\sin^{n}x\di x
    \end{align*}
    Rearranging, we get:
    \begin{equation*}
        \int\sin^{n}x\di x \ = \
        \frac{n-1}{n}\int\sin^{n-2}x\di x - \sin^{n-1}x\cos x
    \end{equation*}
    In particular:
    \begin{equation*}
        \int_{0}^{\pi}\sin^{n}x\di x \ = \
        \frac{n-1}{n}\int_{0}^{\pi}\sin^{n-2}x\di x
    \end{equation*}
    Although they seem random, these will come in handy later on.
\end{lm}
Now, we can begin computing. As a warning, there will be lots of symbols.
We know that the Jacobian of $ f $ is given as:
\begin{equation*}
    Jf \ = \
    \begin{bmatrix}
        \dfrac{\p f_{1}}{\p \phi_{1}} & \dfrac{\p f_{1}}{\p \phi_{2}} & \dots &
        \dfrac{\p f_{1}}{\p \phi_{n-1}} \\[1.125em]
        \dfrac{\p f_{2}}{\p \phi_{1}} & \dfrac{\p f_{2}}{\p \phi_{2}} & \dots &
        \dfrac{\p f_{2}}{\p \phi_{n-1}} \\[1.125em]
        \dfrac{\p f_{3}}{\p \phi_{1}} & \dfrac{\p f_{3}}{\p \phi_{2}} & \dots &
        \dfrac{\p f_{3}}{\p \phi_{n-1}} \\[1em]
        \vdots & \vdots & \ddots & \vdots \\[1em]
        \dfrac{\p f_{n}}{\p \phi_{1}} & \dfrac{\p f_{n}}{\p \phi_{2}} & \dots &
        \dfrac{\p f_{n}}{\p \phi_{n-1}} \\[1.125em]
    \end{bmatrix}
\end{equation*}
The next step is to multiply the matrix with its transpose, which would be rather
messy. Instead, we can define the matrix by defining each entry:
\begin{equation*}
    ((Jf)^{T}(Jf))_{ij} \ = \ \sum_{k=1}^{n}\frac{\p f_{k}}{\p\phi_{i}}
    \frac{\p f_{k}}{\p\phi_{j}}
\end{equation*}
Next, I claim that this matrix is in fact diagonal. Of course, to show this, we
need to know what each partial derivative of $ f_{i} $ actually is. Using our
definition of $ f_{i} $, we have:
\begin{equation*}
    \frac{\p f_{i}}{\p\phi_{j}} \ = \
    \begin{cases}
        0 & j<i-1 \vsp
        -r\displaystyle\prod_{k=j}^{n-1}\sin(\phi_{k}) & j=i-1 \vsp
        r\cos(\phi_{i-1})\cos(\phi_{j})\displaystyle
        \prod_{\substack{k=i\\k\neq j}}^{n-1} \sin(\phi_{k}) & j>i-1
    \end{cases}
\end{equation*}
Now, we verify that the matrix is diagonal. Indeed, suppose WLOG that $ i < j $.
Then:
\begin{align*}
    & ((Jf)^{T}(Jf))_{ij} \ = \ \sum_{k=1}^{n}\frac{\p f_{k}}{\p\phi_{i}}
        \frac{\p f_{k}}{\p\phi_{i}} \ = \ \sum_{k=1}^{i+1}\frac{\p f_{k}}
        {\p\phi_{i}}\frac{\p f_{k}}{\p\phi_{i}} \vsp
    \ = \ & \left(\sum_{k=1}^{i}\left(r\cos(\phi_{k-1})\cos(\phi_{i})
        \prod_{\substack{\ell=k\\ \ell\neq i}}^{n-1}\sin(\phi_{\ell})\right)
        \left(r\cos(\phi_{k-1})\cos(\phi_{j})\prod_{\substack{\ell=k\\
        \ell\neq j}} ^{n-1}\sin(\phi_{\ell})\right)\right) \vsp
    \ - \ & \left(r\cos(\phi_{k-1})\cos(\phi_{j}) \prod_{\substack{\ell=i+1\\
        \ell\neq j}} ^{n-1}\sin(\phi_{\ell})\right)\left(r\prod_{\ell=i}^{n-1}
        \sin(\phi_{\ell})\right) \vsp
    \ = \ & \left(r^{2}\cos(\phi_{i})\cos(\phi_{j})\sin(\phi_{i})\sin(\phi_{j})
        \prod_{\substack{\ell=i+1\\ \ell\neq j}}^{n-1}\sin^{2}(\phi_{\ell})
        \right)\left(\sum_{k=1}^{i}\cos^{2}(\phi_{k-1}) \prod_{\ell=k}^{i-1}
        \sin^{2}(\phi_{\ell})-1\right) \vsp
    \ = \ & \left(r^{2}\cos(\phi_{i})\cos(\phi_{j})\sin(\phi_{i})\sin(\phi_{j})
        \prod_{\substack{\ell=i+1\\ \ell\neq j}}^{n-1}\sin^{2}(\phi_{\ell})
        \right)\left(1-1\right) \vsp
    \ = \ & 0
\end{align*}
So it is indeed diagonal - this makes taking the determinant much easier. That
being said, we still need to know what the diagonal entries actually are. We see
that:
\begin{align*}
    & ((Jf)^{T}(Jf))_{ii} \ = \
        \sum_{k=1}^{i+1}\left(\frac{\p f_{k}}{\p\phi_{i}}\right)^{2} \vsp
    \ = \ & \sum_{k=1}^{i}\left(r\cos(\phi_{k-1})\cos(\phi_{i})
        \prod_{\substack{\ell=k\\ \ell\neq i}}^{n-1}\sin(\phi_{\ell})\right)^{2}
        \ + \ \left(r\prod_{\ell=i}^{n-1}\sin(\phi_{\ell})\right)^{2} \vsp
    \ = \ & \left(r\cos(\phi_{i})\prod_{\ell=i+1}^{n-1}\sin^{2}(\phi_{\ell})
        \right)^{2}\sum_{k=1}^{i}\cos^{2}(\phi_{k-1}) \prod_{\ell=k}^{i-1}
        \sin^{2}(\phi_{\ell}) \ + \ \left(r\prod_{\ell=i}^{n-1}
        \sin(\phi_{\ell})\right)^{2} \vsp
    \ = \ & \left(r\cos(\phi_{i})\prod_{\ell=i+1}^{n-1}\sin^{2}(\phi_{\ell})
        \right)^{2} \ + \ \left(r\prod_{\ell=i}^{n-1}
        \sin(\phi_{\ell})\right)^{2} \vsp
    \ = \ & r^{2}\prod_{\ell=i+1}^{n-1}\sin^{2}(\phi_{\ell})
\end{align*}
Taking the square root and the product over each $ i $ gives us what we want to
integrate:
\begin{equation*}
    V(Jf) \ = \ r^{n-1}\prod_{i=1}^{n-1}\prod_{j=i+1}^{n-1}\sin(\phi_{j}) \ = \
    r^{n-1}\prod_{i=2}^{n-1}\sin^{i-1}(\phi_{i}) \ = \
    r^{n-1}\sin^{n-2}\phi_{n-1}\cdots\sin\phi_{2}
\end{equation*}
Integrating $ V(Jf) $ would thus give us the surface area of $ S^{n-1} $.
We write it down:
\begin{gather*}
    \underbrace{
    \int_{0}^{\pi}\cdots\int_{0}^{\pi}\int_{0}^{2\pi}}_{n-1 \trm{ integrals}}
    V(Jf)\di\phi_{1}\di\phi_{2}\di\phi_{3}\cdots\di x_{n-1} \\
    \ = \ 2\pi r^{n-1}\int_{0}^{\pi}\cdots\int_{0}^{\pi}\sin^{n-2}\phi_{n-1}
    \cdots\sin\phi_{2}\di\phi_{2}\cdots\di\phi_{n-1} \\
    \ = \ 2\pi r^{n-1}\prod_{i=2}^{n-1}\left(\int_{0}^{\pi}\sin^{i-1}\phi_{i}
    \di\phi_{i}\right)
\end{gather*}
Plugging in a few values for $ n $ reveals a recursive sequence:
\begin{gather*}
    S_{(1)} \ = \ \int_{0}^{\pi}\sin x\di x \ = \ 2 \\
    S_{(2)} \ = \ \int_{0}^{\pi}\sin^{2} x\di x \ = \ \frac{1}{2}\cdot\pi \\
    S_{(3)} \ = \ \int_{0}^{\pi}\sin^{3} x\di x \ = \ \frac{2}{3}\int_{0}^{\pi}
    \sin x\di x \ = \ \frac{2}{3}\cdot2 \\
    S_{(4)} \ = \ \int_{0}^{\pi}\sin^{4} x\di x \ = \ \frac{3}{4}\int_{0}^{\pi}
    \sin^{2} x\di x \ = \ \frac{3}{4}\cdot\frac{1}{2}\cdot\pi \\
    \vdots \\
    S_{(i)} \ = \ \frac{i-1}{i}\cdot S_{(i-2)}
\end{gather*}
This comes from our prior lemma. Thus, we can define the surface area of a
general sphere $ S^{n-1} $ of radius $ r $ as:
\begin{equation*}
    \vol(S^{n-1}) \ = \ 2\pi r\prod_{i=2}^{n-1}S_{(i-1)}
\end{equation*}
\newpage
Regrettably, this definition relies on the use of a recursive sequence; there is
indeed a derivation which is more theoretical rather than computational and
yields a fully closed form. However, the closed form also requires knowledge of
certain properties of the Gamma function, and thus has not been included in this
project.

