\documentclass{article}
\usepackage{preamble}
\usepackage{env}
\usepackage{configure}

% define these variables!
\def\coursecode{MAT257}
\def\leftmark{Week 11 - Quadratic Forms} % set text in header; should only be necessary in assignments etc.
\pagenumbering{arabic} % force revert numbering to default; should only be necessary in assignments etc.

\makeatletter
% settings for toc alignment
%
% Configuration
% -------------
% Horizonal alignment in \numberline:
%   l: left-aligned
%   c: centered
%   r: right-aligned
% \nl@align@: Default setting
% \nl@align@<levelname>: Setting for specific level

\def\nl@align@{l}% default
\def\nl@align@section{r}

\makeatother

\begin{document}
\setcounter{subsection}{9}
\setcounter{exr}{5}

\begin{exr}
    Let $ Q(x, y) = x^{2} + 5xy + 3y^{2} $. Find the corresponding symmetric matrix $ A $.
\end{exr}

\begin{pf}
    We know that $ A $ must be symmetric, so we can write:
    \begin{equation*}
        A = \begin{bmatrix}
            a & b \\ b & c
        \end{bmatrix}
    \end{equation*}
    where we solve for $ a, b, $ and $ c $.
    Note that:
    \begin{equation*}
        \vec{x}^{T}A\vec{x} \ = \
        \begin{bmatrix}
            x & y
        \end{bmatrix}
        \begin{bmatrix}
            a & b \\ b & c
        \end{bmatrix}
        \begin{bmatrix}
            x \\ y
        \end{bmatrix} \ = \
        ax^{2} + 2bxy + cy^{2}
    \end{equation*}
    Therefore, we set $ a = 1, b = \frac{5}{2}, $ and $ c = 3 $ to get the corresponding matrix.
    Note that this agrees with the derivation found in Exercise 9.5.
\end{pf}

\end{document}
