\documentclass{article}
\usepackage{preamble}
\usepackage{env}
\usepackage{configure}

% available environments:
% theorem: thm
% definition: defn
% proof: pf
% corollary: crll
% lemma: lm
% question: qu
% solution: soln
% example: xmp
% exercise: exr
%
% options: title=<title>   {all}
%          source=<source> {pf, qu, soln, xmp, exr}  Note: if content is taken directly from the main resource, cite the main resource as ``Primary source material"


% define these variables!
\def\coursecode{}
\def\coursename{} % use \relax for non-course stuff
\def\studytype{} % 1: Personal Self-Study Notes / 2: Course Lecture Notes / 3: Revised Notes / 4: Exercise Solution Sheet
\def\author{\me}
\def\createdate{}
\def\updatedate{\today}
\def\source{} % name, ed. of textbook, or `Class Lectures` for class notes
\def\sourceauthor{} % for class notes, put lecturer
\def\leftmark{Definition Project - Compactness} % set text in header; should only be necessary in assignments etc.
\pagenumbering{arabic} % force revert numbering to default; should only be necessary in assignments etc.

\begin{document}

\begin{center}
    {\Large \textbf{On the Subject of Compactness I}}

    \medskip
    {\large E. Gu, K. Kang, F. Shaik}
\end{center}

\begin{abstract}
    This paper aims to introduce the concept of compactness, and provide some elementary motivation.
    We will also see how compact sets can be used to simply construct other compact sets, and
    we end with a method to find compact sets in $ \bb{R}^{n} $.
\end{abstract}

\section{Introduction}
\setcounter{subsection}{1}

What is compactness? What does it mean for a set to be compact?
Sure, we have a definition for it, but where does it come from? Why is it defined this way?
Answering these questions will allow us to build a better intuition about compactness,
meaning we can more easily understand how it may or may not relate to other properties.
So, let's look at the definition, and keep it in the back of our minds as we explore what it
might encapsulate.

\begin{defn}
    Let $ (X, d) $ be a metric space. \vsp
    %
    Consider any collection of open subsets $ \set{U_{i}}_{i \in I} $ where each
    $ U_{i} \subseteq X $, for some indexing set $ I $.
    We say that $ U = \bigcup_{i \in I} U_{i} $ is an \underline{open cover} of $ X $ if
    $ X \subseteq U $. \vsp
    %
    We say that $ X $ is \textbf{compact} if for every open cover $ U $ of $ X $, there exists some
    natural number $ n \in \bb{N} $ such that $ X \subseteq \bigcup_{k=1}^{n} U_{i_{k}} $.
    We call such a cover a \underline{finite subcover}.
\end{defn}

Okay, so where does this come from?
Usually, when we make a definition, it's because it specifies a useful property or generalizes
something. So what could compactness possibly be generalizing? \vsp
%
One way to think about it is that it somehow generalizes the notion of \textit{finiteness} of sets.
Certainly, if for every open cover, you can find a finite subcover, then wouldn't we expect it
to behave similarly in some way?
The name itself even gives clues - usually, when something is compact, we mean that it's smaller,
or more compressed. Of course, if we \textit{really} want to find out, then we can always take a
look at some examples!

\section{Some Basic Examples}

Let's start simple, in $ \bb{R} $, just to get a handle on things.
We can always move up to $ \bb{R}^{n} $ if the one-dimensional space seems to easy.
Some simple sets we have pretty good intuition for are intervals; let's examine some of those.

\begin{xmp}
    Consider the interval $ (0, \infty) $. We want to check if this is compact. \vsp
    %
    Indeed, this interval is clearly not compact, as we can take the following open cover:
    \begin{equation*}
        U = \bigcup_{i=1}^{\infty} (0, i)
    \end{equation*}
    We can see that it covers our interval, but it cannot have a finite subcover.
    If some finite subcover existed, then there would be a maximum open
    interval $ (0, M) $ which clearly does not contain $ M $.
\end{xmp}

Okay, so $ (0, \infty) $ doesn't work. But this also goes off to infinity - that doesn't make sense
for a term called ``compact" anyway. Let's try putting a bound on it.

\begin{xmp}
    Consider the interval $ (0, 1) $. We want to check if this is compact. \vsp
    %
    Similar to the previous example, this is also not compact.
    We can take the open cover given by:
    \begin{equation*}
        U = \bigcup_{i=1}^{\infty} \left( \frac{1}{i}, 1 - \frac{1}{i} \right)
    \end{equation*}
    Similarly to the last example, this cover does not have a finite subcover.
    So indeed, the interval $ (0, 1) $ is not compact.
\end{xmp}

So even if we bound the interval, it still isn't compact.
But in this case, the issue is much more clear: the reason we can construct a cover like this
is because our interval is open.
So, let's add in the endpoints to make the interval closed, and try this.

\begin{xmp}
    Consider the interval $ [0, 1] $. We want to check if this is compact. \vsp
    %
    Indeed, this interval is compact.
    To see this, let $ U = \bigcup_{i \in I} U_{i} $ be any open cover.
    Note that there exists some $ U_{0} $ such that $ 0 \in U_{0} $.
    Thus, $ [0, 0] $ has a finite subcover. \vsp
    %
    Now, consider $ [0, x] $ such that it has a finite subcover for some $ x \in (0, 1) $.
    Since there exists some $ U_{x} $ such that $ x \in U_{x} $, then there exists some
    $ \delta > 0 $ such that $ [x, x + \delta] \subseteq U_{x} $.
    Since $ [0, x] \subseteq \bigcup_{k=1}^{n} U_{i_{k}} $ has a finite cover, then
    $ [0, x] \cup [x, x + \delta] \subseteq \bigcup_{k=1}^{n} U_{i_{k}} \cup U_{x} $,
    so $ [0, x + \delta] $ also has a finite cover. \vsp
    %
    Finally, note that the above holds generally true for all $ x \in (0, 1) $, by a process
    similar to induction.
    Also, since $ \delta > 0 $, then $ [0, x] \subsetneq [0, x + \delta] $, thus the above
    is sufficient to show that $ [0, 1] $ has a finite subcover as needed.
\end{xmp}

Neat! So we were able to find a compact subset of $ \bb{R} $.
Closure seemed to play a role in making the set compact,
but can we drop the bounded requirement?

\begin{xmp}
    Consider the interval $ [0, \infty) $. We want to check if this is compact. \vsp
    %
    Just like the first example, this is not a compact set. Consider the cover given by:
    \begin{equation*}
        U = \bigcup_{i=1}^{\infty} (-1, i)
    \end{equation*}
    Indeed, $ U $ covers our interval - it actually covers a little bit more than that.
    But clearly, we still can't find a finite subcover, so it is not compact.
\end{xmp}

Okay, so even if we have a closed set, it still needs to be bounded.
That seems to make sense. We might also note that any finite set is trivially compact, but
we won't make a big deal out of it here. It does, however, serve as a good sanity check - if
finite sets are automatically compact, then maybe compact really is a
form of generalizing the notion of finiteness like we said.
In $ \bb{R} $, these compact sets seem to be closed intervals, as long as they don't run off
to infinity. But what can we say about $ \bb{R}^{n} $?

\section{Bigger Sets}

Here, we'll examine the basic set operations of union, intersection, and cartesian product.
We'll look at how they interact, and how they can help us generalize to $ \bb{R}^{n} $. \npgh

First, let's take a look at unions.
Indeed, it is fairly trivial to see that the union and intersection of two compact sets is compact:

\begin{lm}
    Let $ (X, d) $ be a metric space, and $ A, B \subseteq X $ compact subsets.
    Then, $ A \cup B $ is compact.
\end{lm}

\begin{pf}
    Let $ U = \bigcup_{i \in I} U_{i} $ be an open cover.
    Since $ A, B $ are compact, then they have finite subcovers:
    \begin{equation*}
        A \subseteq \bigcup_{j=1}^{m} U_{i_{j}} \quad
        B \subseteq \bigcup_{k=1}^{n} U_{i_{k}}
    \end{equation*}
    So clearly, we have that:
    \begin{equation*}
        A \cup B \subseteq \bigcup_{j=1}^{m+n} U_{i_{j}}
    \end{equation*}
    So $ A \cup B $ is compact as needed.
\end{pf}

\begin{lm}
    Let $ (X, d) $ be a metric space, and $ A, B \subseteq X $ compact subsets.
    Then, $ A \cap B $ is compact.
\end{lm}

\begin{pf}
    Let $ U = \bigcup_{i \in I} U_{i} $ be an open cover of $ A \cap B $.
    Note that if $ U $ does not cover $ A $ nor $ B $, then we can extend the cover
    by adding open sets which cover $ A \setminus B $ (WLOG). \vsp
    %
    Clearly then, since $ A $ is compact, then there is a finite subcover which covers $ A $,
    and so must also cover $ A \cap B $. \vsp
    %
    Note that if we did add open sets to the cover in order to fully cover $ A \setminus B $,
    these would not appear in the finite subcover of $ A \cap B $, since no point in $ A \cap B $
    would be in an open set covering $ A \setminus B $.
\end{pf}

The cartesian product of two compact sets is also compact, although it's not so simple of a proof.
\newpage

\begin{lm}
    Let $ (X, d) $ be a metric space, and $ A, B \subseteq X $ compact subsets.
    Then, $ A \times B $ is compact.
\end{lm}

\begin{pf}
    the error is still being fixed for this one - hang tight!
\end{pf}

Now that we know that the three basic set operations preserve compactness, we can quickly make
some general observations. For example:
\begin{itemize}
    \item Any (non-empty) open set in $ \bb{R} $ is not compact.
    \item Any closed set in $ \bb{R} $ is compact.
    \item Any (non-empty) open rectangle in $ \bb{R}^{n} $ is not compact.
    \item Any closed rectangle in $ \bb{R}^{n} $ is compact.
\end{itemize}

These are just a few basic examples, but we already see that we can in fact say a lot from here.
We could probably determine whether or not just about any set in $ \bb{R}^{n} $ is compact, using
only these basic properties - specifically, the fact that these set operations preserve compactness.
But is there an easier, more universal way to identify them?

\section{Categorizing Compact Sets}

We noticed earlier in our example of $ \bb{R} $ that the compact intervals were those which were
closed and bounded. Using our properties we looked at in the last section, we can expect similar
behaviour in the more general $ \bb{R}^{n} $. And indeed, this section is dedicated precisely to
that result!

\begin{thm}[title=Heine-Borel Theorem]
    Let $ S \subseteq \bb{R}^{n} $ be a subset of $ \bb{R}^{n} $.
    Then, $ S $ is compact if and only if it is closed and bounded.
\end{thm}

\begin{pf}
    We begin by proving that closed and bounded implies compact. This is done in two steps.
    First, we show that any countable cover must have a finite subcover.
    Then, we show that any uncountable cover has a countable subcover,
    thereby reducing to the first step. \vsp
    %
    Suppose for the sake of contradiction that $ S $ is not compact.
    Let $ U = \set{U_{i}}_{i=1}^{\infty} $ be a countable open cover,
    such that there is no finite subcover.
    More precisely, this means that for any $ n \in \bb{N} $:
    \begin{equation*}
        S \nsubseteq \bigcup_{j = 1}^{n} \set{U_{i_{j}}} 
    \end{equation*}
    So any union of finitely many open sets from the cover do not completely cover $ S $.
    This allows us to define the following sequences:
    \begin{equation*}
        \set{p_{k}}_{k=1}^{\infty} \quad , \quad p_{k} \in
        S \setminus (U_{1} \cup U_{2} \cup \dots \cup U_{k})
    \end{equation*}
    Since $ S $ is a bounded set, then our sequence is bounded.
    Therefore, by the Bolzano-Weierstrass Theorem, there exists a subsequence
    $ \set{p_{k_{j}}}_{j \geq 1} $ such that $ p_{k_{j}} \rightarrow p $ for some $ p \in S $. \vsp
    %
    Since $ p_{k_{j}} \rightarrow p $, then $ p $ is a limit point of $ S $.
    Since $ S $ is closed, then $ p \in S $.
    Therefore, there exists some $ U_{p} $ such that $ p \in U_{p} $.
    Since $ U_{p} $ is an open set, then there must exist $ \delta > 0 $ such that
    we have that $ B(p, \delta) \subseteq U_{p} $. \vsp
    %
    Since $ p_{k_{j}} \rightarrow p $, then there exists some $ N $ such that
    for all $ j \geq N $, $ p_{k_{j}} \in U_{p} $.
    In particular, there exists some $ p_{k_{N}} $ such that:
    \begin{equation*}
        p_{k_{N}} \in U_{p} \quad , \quad p_{k_{N}} \in
        S \setminus (U_{1} \cup U_{2} \cup \dots \cup U_{p} \cup \dots \cup U_{k_{N}})
    \end{equation*}
    Clearly, this is a contradiction, as we have that
    $ p_{k_{N}} \in U_{p} $ and $ p_{k_{N}} \notin U_{p} $.
    Therefore, it must be true that $ S $ has a finite subcover as required. \vsp
    %
    Now, suppose that $ U = \set{U_{i}}_{i \in I} $ is an uncountable open cover of $ S $.
    As mentioned earlier, it suffices to show that there exists a countable subcover,
    thereby reducing it to the first step. \vsp
    %
    Recall that $ S $ is bounded. Then, by the Bolzano-Weierstrass Theorem,
    any sequence in $ S $ has a subsequence $ (x_{n_{i}}) $ which
    converges to some $ x \in S $. This gives us a cluster point $ x $ for the sequence $ (x_{n}) $.
    Since this is true of all sequences in $ S $, then $ S $ is a clustering subset of $ X $. \vsp
    %
    Since $ S $ is a clustering subset, then by the Lebesgue Number Lemma, there exists some
    $ \delta > 0 $ such that $ \delta $ is the magic number of our open cover $ U $. \vsp
    %
    Consider the set $ S' = S \cap \bb{Q}^{n} $. Since $ \bb{Q}^{n} $ is a countable set,
    then this is a countable set of points in $ S $.
    Therefore, it makes sense to define the set $ U_{s} $ as:
    \begin{equation*}
        U_{s} = \bigcup_{i=1}^{\infty} \set{B \left( s_{i}, \frac{\delta}{2} \right) } \quad
        s_{i} \in S'
    \end{equation*}
    This is indeed a cover, as $ \bb{Q}^{n} $ is dense in $ \bb{R}^{n} $.
    Furthermore, since each open ball has diameter less than $ \delta $,
    then there exists $ U_{s_{i}} $ such that:
    \begin{equation*}
        B \left( s_{i}, \frac{\delta}{2} \right) \subseteq U_{s_{i}}
    \end{equation*}
    Therefore, $ U_{s} $ is a countable subcover of $ S $, so by the first step,
    $ S $ must be compact, as required.
\end{pf}

\begin{pf}
    Now, we show that if $ S $ is compact, then it is closed and bounded.
    We'll do this by contradiction, and take each case separately. \vsp
    %
    Suppose $ S $ is not closed. Then, there exists a limit point $ p \in \overline{S} $ such that
    $ p $ is not in $ S $ itself. \vsp
    %
    Consider the open cover defined as:
    \begin{equation*}
        \cl{U} = \bigcup_{i=1}^{\infty} U_{i} \qquad
        U_{i} = \left(\overline{B} \left( p, \frac{1}{i} \right) \right)^{c}
    \end{equation*}
    We can see that $ \cl{U} = \bb{R} \setminus \set{p} $, so $ S \subseteq \cl{U} $.
    Suppose by contradiction that $ S $ is compact; then, there exists a finite collection
    such that:
    \begin{equation*}
        S \subseteq \bigcup_{k = 1}^{j} U_{i_{k}}
    \end{equation*}
    Since this is finite, then we can consider the highest index $ m $.
    Since $ p $ is a limit point, then:
    \begin{equation*}
        U_{i_{m+1}} \cap S = B \left( p, \frac{1}{m+1} \right) \cap S \neq \varnothing
    \end{equation*}
    So there exists some $ x \in U_{i_{m+1}} \cap S $ such that $ x $ is not in the finite
    collection, but $ x \in S $, so this is a contradiction. Therefore, $ S $ is not compact. \npgh

    Now, suppose $ S $ is not bounded. Let $ x \in K $ be any point. Then, consider the open cover:
    \begin{equation*}
        \cl{U} = \bigcup_{i = 1}^{\infty} B(x, i)
    \end{equation*}
    Clearly, $ S \subseteq \cl{U} $.
    Now, suppose that $ S $ is compact. Then, there exists a finite subcover of $ \cl{U} $.
    Since it's finite, denote the maximum of these indices as $ m $. \vsp
    %
    Since $ S $ is unbounded, then there exist $ x_{1}, x_{2} \in S $ such that:
    \begin{gather*}
        d(x_{1}, x_{2}) > 2m \\
        \implies \ d(x_{1}, x) + d(x, x_{2}) \geq d(x_{1}, x_{2}) > 2m
    \end{gather*}
    So we must have that one of $ x_{1} $ and $ x_{2} $ has distance more than $ m $ from $ x $.
    WLOG, suppose this point is $ x_{1} $.
    Then, $ x_{1} \notin B(x, m) $ so is not in the finite subcover, but is a point in $ S $.
    This is a contradiction, so $ S $ must not be compact as needed.
\end{pf}

This theorem allows us to \textit{instantly} tell if a set is compact or not.
No need to search for compact sets such that their intersection is the set of interest, or find
compact sets whose union is the set of interest, etc.
Instead, this theorem tells us precisely which sets are compact. \vsp
%
Of course, there still is a catch - otherwise, it would be too good to be true!
Let's look at a counterexample:

\begin{xmp}
    Consider the space $ (C[0,1], \norm{\cdot}_{\infty}) $.
    We show that the closed unit ball is not compact. \vsp
    %
    Indeed, we can take a sequence $ (x_{n})_{n\geq1} $ as $ x_{n} = x^{n} $.
    Clearly on $ [0, 1] $, the supremum of each term is $ 1 $.
    However, we can see that this sequence does not have a convergent subsequence. \vsp
    %
    e
\end{xmp}

\section{Generalizing}

So far, we have examined compact sets in $ \bb{R}^{n} $, and were able to make several conclusions
about them. But what about more generally? What can we say about compact sets in other
metric spaces? (We (will) have a follow-up paper that discusses this!) \npgh

Furthermore, we only looked at some simple set operations.
What if we take them further? Do infinite unions, intersections, cartesian products still
preserve compactness? We also didn't look at possible subsets of a compact set.
For example, is the set of limit points of a compact set compact? Boundary points?
Questions such as these are only a few ways to further examine compact sets in the wild.

\end{document}
