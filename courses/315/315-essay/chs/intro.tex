\section{Introduction}
\subsection{Motivation}

Often in math do we come across and work with polynomials.
They're simple, easy to understand, and widely applicable across many areas of study.
For instance, we've seen polynomials a number of times within our course:

\begin{itemize}
    \item Pythagorean triples are integer solutions to
        a polynomial of two degrees in three variables:
        \begin{equation*}
            x^{2}+y^{2}-z^{2} \ = \ 0
        \end{equation*}
    \item The problem of the sum of two squares is similar:
        \begin{equation*}
            a^{2}+b^{2}-n \ = \ 0
        \end{equation*}
    \item Fermat's Last Theorem famously generalizes the Pyhthagorean triples:
        \begin{equation*}
            x^{n}+y^{n}-z^{n} \ = \ 0
        \end{equation*}
\end{itemize}

When we discuss these topics, we're really discussing the solutions to these
polynomials.
Thus, we can ask: in general, when do we have solutions to any given polynomial,
and if they exist - what are they?

\subsection{Problem Statement}

Because we typically care about integer solutions, one strategy is to find
\textit{all} solutions, real or complex, then select integer solutions.
We'll take this approach for solutions to polynomials.

Before we can worry about finding solutions, we need to know they exist.
Thankfully:
\begin{thm}[title=Fundamental Theorem of Algebra]
    A polynomial of degree $n$ has $n$ complex roots, with multiplicity.
\end{thm}

From high school, we know there's a quadratic equation to find roots of a
quadratic polynomial:
\begin{equation*}
    x \ = \ \frac{-b\pm\sqrt{b^{2}-4ac}}{2a}
\end{equation*}

There's also a cubic and even a quartic formula for degree 4 polynomials,
which are omitted for their length.
However, we claim that there are no more.

\begin{thm}[type=Claim]
    There is no closed-form equation for the roots of a degree 5 polynomial with
    rational coefficients, using only the following operations:
    \begin{equation*}
        + \qquad - \qquad \times \qquad \div \qquad \sqrt[n]{\cdot}
    \end{equation*}
    that is, addition, subtraction, multiplication, division,
    and $n$-th roots (also known as \textit{radicals}) for any integer $n$.
\end{thm}

This is a very general statement, and is difficult to precisely describe
mathematically.
However, note that the existence of a closed-form equation of the roots of a
given degree polynomial indicates that the roots of \textit{any} polynomial of
the same degree can be written in precisely that form.
Thus, for it to be false in the case of the quintic, it suffices to find even
a single quintic whose roots cannot be expressed using the above operations.
In other words, we aim to prove the slightly stronger theorem:

\begin{thm}[title=Abel-Ruffini Theorem]
    There exists a polynomial of degree 5 whose roots cannot be written
    in radicals.
\end{thm}

Throughout this essay, we will examine some of the history of this problem,
and examine where the modern proof comes from.


