\section{Introduction}
\subsection{Motivation}

Often in math do we come across and work with polynomials.
They're simple, easy to understand, and widely applicable across many areas of study.
For instance, we've seen polynomials a number of times within our course:

\begin{itemize}
    \item Linear Diophantine equations are polynomials of one degree in two variables:
        \begin{equation*}
            ax+by-c \ = \ 0
        \end{equation*}
    \item Pythagorean triples are integer solutions to
        a polynomial of two degrees in three variables:
        \begin{equation*}
            x^{2}+y^{2}-z^{2} \ = \ 0
        \end{equation*}
    \item The problem of the sum of two squares is similar:
        \begin{equation*}
            a^{2}+b^{2}-n \ = \ 0
        \end{equation*}
    \item Fermat's Last Theorem famously generalizes the Pyhthagorean triples:
        \begin{equation*}
            x^{n}+y^{n}-z^{n} \ = \ 0
        \end{equation*}
    \item Fermat's \textit{Little} Theorem, and more generally Euler's Theorem,
        is a statement about polynomials of a particular degree, modulo some value
        (ok, this is pushing it):
        \begin{equation*}
            g^{p-1} \ \equiv \ 1 \bmod p \qquad g^{\vphi(n)} \ \equiv \ 1 \bmod n
        \end{equation*}
\end{itemize}

When we discuss these topics, we're really discussing these polynomials.
But more specifically, we care about the \textit{solutions} to these polynomials, moreso than
the polynomials themselves.
Thus, it is only natural to ask - when, in general, do we have solutions to any given polynomial,
and if they exist - what are they?
\textcolor{green}{some mention of solns in $\bZ$ vs $\bR$ here?}


\subsection{Problem Statement}





