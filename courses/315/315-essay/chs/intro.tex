\section{Introduction}
\subsection{Motivation}

Often in math do we come across and work with polynomials.
They're simple, easy to understand, and widely applicable across many areas of study.
For instance, we've seen polynomials a number of times within our course:

\begin{itemize}
    \item Linear Diophantine equations are polynomials of one degree in two variables:
        \begin{equation*}
            ax+by-c \ = \ 0
        \end{equation*}
    \item Pythagorean triples are integer solutions to
        a polynomial of two degrees in three variables:
        \begin{equation*}
            x^{2}+y^{2}-z^{2} \ = \ 0
        \end{equation*}
    \item The problem of the sum of two squares is similar:
        \begin{equation*}
            a^{2}+b^{2}-n \ = \ 0
        \end{equation*}
    \item Fermat's Last Theorem famously generalizes the Pyhthagorean triples:
        \begin{equation*}
            x^{n}+y^{n}-z^{n} \ = \ 0
        \end{equation*}
    \item Fermat's \textit{Little} Theorem, and more generally Euler's Theorem,
        is a statement about polynomials of a particular degree, modulo some value
        (ok, this is pushing it):
        \begin{equation*}
            g^{p-1} \ \equiv \ 1 \bmod p \qquad g^{\vphi(n)} \ \equiv \ 1 \bmod n
        \end{equation*}
\end{itemize}

When we discuss these topics, we're really discussing these polynomials.
But more specifically, we care about the \textit{solutions} to these polynomials, moreso than
the polynomials themselves.
Thus, it is only natural to ask - when, in general, do we have solutions to any given polynomial,
and if they exist - what are they?
\textcolor{green}{some mention of solns in $\bZ$ vs $\bR$ here?}


\subsection{Problem Statement}

Throughout this course (MAT315) and in number theory as a whole,
we typically care about integer solutions the most.
One way to find integer solutions is to find \textit{all} the solutions in
general, then reduce/pick out integer solutions.
We'll take this approach for solutions to polynomials.

Before we can worry about finding solutions, we need to know they exist.
Thankfully:
\begin{thm}[title=Fundamental Theorem of Algebra]
    A polynomial of degree $n$ has $n$ complex roots, with multiplicity.
\end{thm}

From high school, we know there's a quadratic equation to find roots of a
quadratic polynomial:
\begin{equation*}
    x \ = \ \frac{-b\pm\sqrt{b^{2}-4ac}}{2a}
\end{equation*}

Slightly lesser known is the longer cubic formula, which is usually broken into
parts when written:
\begin{equation*}
    \textcolor{green}{TODO}
\end{equation*}

There's even a quartic formula for degree 4 polynomials, which is too long and
involved to be included in this essay.
However, we claim that this does not continue any further.

\newpage
\begin{thm}[type=Claim]
    We claim that there is no closed-form formula or expression for the roots
    of $p(x)$, only using the operations of
    \begin{equation*}
        + \qquad - \qquad \times \qquad \div \qquad
        (\cdot)^{n} \qquad \sqrt[n]{\cdot}
    \end{equation*}
    that is, addition, subtraction, multiplication, division,
    $n$-th powers, and $n$-th roots for any integer $n$.
\end{thm}

Throughout this essay, we will examine the history of this problem, learn what
worked, what didn't, and why, and see how to generalize this to higher degree
polynomials.

\textcolor{green}{i think that section needs some massaging still}



