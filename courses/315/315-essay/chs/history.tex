\section{History}

\subsection{Closed-form expressions}

The general method which was used to derive the quartic equation is largely similar to the derivation
for the cubic equation; the discoverer, Ferrari in 1540, was in fact a student of Cardano at the
time, and adapted the latter's argument to construct the quartic.
We examine the method published by Cardano, which is due to del Ferro and Tartaglia.

Given a polynomial $a_{n}x^{n}+a_{n-1}x^{n-1}+\dots +a_{1}x+a_{0}=0$, we can always divide by
$a_{n}$ to make the polynomial monic; this makes it easier to work with.
Thus, all polynomials from here forward will be monic without loss of generality.

Consider a quadratic $x^{2}+ax+b=(x-r)(x-s)$.
Observe that:
\begin{equation*}
    (x-r)(x-s) \ = \ x^{2}-(r+s)x+rs \ = \ x^{2}+ax+b
\end{equation*}
This means that for any quadratic $x^{2}+ax+b$ with roots $r$ and $s$,
we always have that $a=-(r+s)$ and $b=rs$.
This property generalizes to higher degree polynomials; in the case of the cubic, we have:
\begin{equation*}
    x^{3}+ax^{2}+bx+c = (x-r_{1})(x-r_{2})(x-r_{3}) \ \implies \
    \begin{cases}
        a = -(r_{1}+r_{2}+r_{3}) \\
        b = r_{1}r_{2} + r_{2}r_{3} + r_{1}r_{3} \\
        c = -r_{1}r_{2}r_{3}
    \end{cases}
\end{equation*}
These equations of the coefficients in terms of the roots are known as \textbf{Vieta's Formulas}.
Named after their discoverer Fran\c cois Vi\`ete in the 16th century for positive roots,
the method may have been known as early as the 12th century.

Let $P(x)=x^{3}+ax^{2}+bx+c$ be a cubic polynomial.
We use the above formulas to construct a quadratic $r(x)$ such that the roots of $P(x)$ can be
derived from the roots of $r(x)$.
To see how, first apply a change of variables $x=y-\frac{b}{3a}$:
\begin{equation*}
    P\left(y-\frac{b}{3a}\right)
    \ = \ \left(y-\frac{b}{3a}\right)^{3}+a\left(y-\frac{b}{3a}\right)^{2}
    +b\left(y-\frac{b}{3a}\right)+c
    \ = \ y^{3} + my + n
\end{equation*}
The act of performing such a change of variables to eliminate the term of second-highest degree
is known as \textbf{depressing the polynomial} - in this case, a cubic.
Let $Q(y):=y^{3}+my+n$; note that $m$ and $n$ are in terms of $a,b,$ and $c$.
Suppose a root $y=w+z$ is expressed as a sum of two numbers. Then:
\begin{gather*}
    (w+z)^{3} \ = \ w^{3}+3w^{2}z+3wz^{2}+z^{3} \ = \ w^{3}+z^{3}+3wz(w+z) \\
    \ \implies \ (w+z)^{3}-3wz(w+z)-(w^{3}+z^{3}) = 0
\end{gather*}
Since $y=w+z$, then:
\begin{equation*}
    \begin{cases}
        m \ = \ -3wz \\ n \ = \ -(w^{3}+z^{3})
    \end{cases}
    \ \implies \
    \begin{cases}
        w^{3}+z^{3} \ = \ -n \\ w^{3}z^{3} \ = \ \frac{-m^{3}}{27}
    \end{cases}
\end{equation*}
We then apply Vieta's formulas to construct a quadratic with roots $w^{3},z^{3}$:
\begin{equation*}
    R(x) \ := \ x^{2}+nx-\frac{m^{3}}{27} \ = \ 0 \qquad
    w^{3},z^{3} \ = \ \frac{-n}{2}\pm\sqrt{\frac{n^{2}}{4}+\frac{m^{3}}{27}}
\end{equation*}
Recalling that $y=w+z$ is a root of $Q(y)$, we thus obtain a root given as:
\begin{equation*}
    y \ = \ \sqrt[3]{\frac{-n}{2}+\sqrt{\frac{n^{2}}{4}+\frac{m^{3}}{27}}}
    + \sqrt[3]{\frac{-n}{2}-\sqrt{\frac{n^{2}}{4}+\frac{m^{3}}{27}}}
\end{equation*}
The other two roots can be obtained by taking different possible values of $w,z$ as the cube root,
and using the earlier constraint that $m=-3wz$.

In general, $R(x)$ is called the \textbf{resolvent} of $p(x)$, because it is a polynomial of
smaller degree whose roots we can use to obtain the roots of $p(x)$.
Since $m,n$ are solely in terms of the coefficients $a,b,c$ of $p(x)$,
the pre-existing quadratic formula thus makes finding roots for a cubic quite straightforward.

\subsection{Symmetry of roots}

In 1770 and 1771, Lagrange generalized the above work by finding a connection between
finding a resolvent of a polynomial and a \textit{discrete Fourier transform} of the roots.
While we do not need to know what a DFT is, the key insight lies with what we will refer to as
``almost symmetric expressions".

\begin{defn}
    A \textbf{symmetric} expression on $n$ variables is an expression whose value does not change
    upon any permutation of the $n$ variables.

    An \textbf{almost symmetric} expression is an expression which takes on a limited number of
    different values.
\end{defn}

For instance, consider $a^{2}b+b^{2}c+c^{2}a$.
This is not a symmetric expression, however it only has two algebraically distinct values:
\begin{equation*}
    a^{2}b+b^{2}c+c^{2}a \qquad a^{2}c+c^{2}b+b^{2}a
\end{equation*}
Any permutation of $a,b,c$ which does not fix any of the letters will yield the expression on the
left. On the other hand, any permutation which fixes exactly one of the letters will yield the
expression on the right.
Thus, $a^{2}b+b^{2}c+c^{2}a$ is an expression in 3 variables with 2 algebraically distinct values.
There is also one in 4 variables with 3 algebraically distinct values:
\begin{equation*}
    (a+b)(c+d) \qquad (a+c)(b+d) \qquad (a+d)(b+c)
\end{equation*}
In general, Lagrange's key insight was to take a depressed polynomial of degree $n$, and find an
almost symmertic expression in $n$ variables with $n-1$ algebraically distinct values.
This would allow him to construct a resolvent polynomial and a system of equations which connected
it to the original polynomial.
For instance, given a quartic $p(x)$ with roots $a,b,c,d$:
\begin{equation*}
    \begin{cases}
        (a+b)(c+d)=A \\
        (a+c)(b+d)=B \\
        (a+d)(b+c)=C \\
        a+b+c+d=0
    \end{cases}
    \qquad \gto \qquad
    r(x):=(x-A)(x-B)(x-C)=0
\end{equation*}
Since the roots of $r(x)$ can be found with the cubic equation, and by the system of equations are
also written in terms of the roots of $q(x)$, and since those can be written in terms of the
coefficients of $q(x)$, this provided a method of solving the quartic.
Note the last condition that $a+b+c+d=0$ comes from depressing the quartic.

The significance of Lagrange's work was in showing that the existence of formulas for 2nd, 3rd,
and 4th degree polynomials in fact stems from a single idea.
Although unknown at the time, Lagrange's work is a special case of a broader fact:
\begin{thm}
    Given a polynomial $p(x)$, any symmetric polynomial in terms of the roots of $p(x)$ can be
    written as a polynomial in terms of the coefficients of $p(x)$.
\end{thm}
This theorem would later be dubbed the \textbf{Fundamental Theorem of Symmetric Polynomials}.

However, Lagrange struggled to continue his work to degree 5 and beyond - he could not find an
almost symmetric expression in 5 variables with 4 algebraically distinct values.
More specifically, whenever he tried to construct a resolvent to the quintic, it would always end
up with a degree \textit{higher} than 5.
This ultimately led him to consider the existence of the quintic in the negative.



