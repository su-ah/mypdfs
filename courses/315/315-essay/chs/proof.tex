\section{Proof of the Statement}

\subsection{Prerequisite Algebra}

The modern proof of non-existence uses the language of Galois theory;
in this section, we'll highlight why.
We will be assuming some familiarity with fields and elementary group theory.

When discussing the existence of roots of a polynomial,
we want to clarify which field is being discussed.
For instance, every polynomial has $n$ roots in $\bC$ (with multiplicity),
but this is not the case in $\bR$.

The field of interest for this proof is $\bQ$.
This is because we care about solutions to polynomials, and $\bR$ has
``too much information" - it contains (almost) all transcendental numbers, and
every non-negative real number has an $n$-th root.

We use \textbf{field extensions} over $\bQ$ to examine our polynomials, because
this allows us to control the obtainable elements within our field.
In particular, we know when a root can be written with the operations we
care about.
A \textbf{splitting field} of a polynomial is a field extension over which the
polynomial factors into linear terms.

Given an extension $L/K$, its \textbf{Galois group} is the group $\Aut(L/K)$ of
automorphisms of the extension $L/K$.
This is the group of permutations over $L/K$.
The extension is a \textbf{Galois extension} if all $\vphi\in\Aut(L/K)$ fix $K$.
Equivalently, the extension is the splitting field of a (separable) polynomial.

A \textbf{solvable group} is a series of normal subgroups
\begin{equation*}
    \set{e} = G_{1} \ngrp G_{2} \ngrp \cdots \ngrp G_{n}
\end{equation*}
such that each factor group $G_{k+1}/G_{k}$ is abelian.
The motivation for this definition actually comes from Galois theory itself,
and encapsulates a property we will see in the proof itself.

\subsection{Proving the claim}

We finish this essay with the proof of the claim.

Let $p(x)$ be a degree 5 polynomial with rational coefficients;
suppose its roots can be written using radicals, and WLOG are distinct.
Consider the splitting field $\bQ[\alpha_{1},\dots,\alpha_{n}]$ of $p(x)$.
We construct a tower of field extensions as follows:

Write out each of the roots of $p$.
For visual purposes, consider $\sqrt[5]{1+3\sqrt[3]{10}}$.
Let $\beta_{1}$ be any of the ``innermost" radicals of the root, in this case
$\sqrt[3]{10}$.
Start by adjoining $\beta_{1}$ to $\bQ$.
Then, iteratively adjoin the next radical $\beta_{k}=\sqrt[n]{x}$ where
$x\in\bQ[\beta_{1},\dots,\beta_{k-1}]$.

At each step, if the extension is not Galois, then first adjoin the $n$-th roots
of unity, followed by the desired radical.
For instance, in the case of $\sqrt[3]{2}$:
\begin{equation*}
    \bQ \ \gto \ \bQ[e^{2\pi i/3}] \ \gto \ \bQ[e^{2\pi i/3}, \sqrt[3]{2}]
\end{equation*}
Thus, the extension at each step is the splitting field of a polynomial, and
therefore is Galois.
After this process, $\bQ[\beta_{1},\dots,\beta_{m}]$ contains all the roots
of $p(x)$, so we can draw the diagram shown in Figure 1.

Now, consider any three consecutive extensions $A,B,C$ as in Figure 2,
where $G, N, L$ are the Galois groups of $C/A$, $C/B$, and $B/A$ respectively.
First, we claim that $N\ngrp G$.

\begin{block}
    Let $\vphi\in G$ be an automorphism.
    Since $C/A$ is Galois, then $\vphi$ fixes elements in $A$.
    Recall $B$ is constructed by adjoining $A$ with particular roots of a polynomial $f$.
    Let $x\in B$ be a root of $f$. Notice:
    \begin{equation*}
        f(x) = \sum_{i=0}^{n}a_{i}x^{i} = 0 \ \implies \ f(\vphi(x))=\vphi(f(x))=0
    \end{equation*}
    So $\vphi$ sends roots of $f$ to roots of $f$, or in other words, $\vphi(B)=B$.

    Now, let $\psi\in N$. Consider $(\vphi\circ\psi\circ\vphi^{-1})(b)$ for some $b\in B$:
    \begin{equation*}
        \vphi^{-1}(b) \in B \ \implies \ \vphi(\psi(\vphi^{-1}(b))) = \vphi(\vphi^{-1}(b)) = b
    \end{equation*}
    In other words, we see that $\vphi\circ\psi\circ\vphi^{-1}$ fixes $B$, and thus must be an element
    of $N$.
\end{block}

Next, we claim that $L=G/N$ is a quotient group.

\begin{block}
    Since $N\ngrp G$, the group $G/N$ is well-defined.
    Note that any automorphism on $B$ can be extended to an automorphism on $C$, and
    as previously shown, automorphisms on $C$ restrict to automorphisms on $B$.
    If $\vphi_{1},\vphi_{2}\in G$ agree on $B$, then $\vphi_{1}\vphi_{2}^{-1}$ is the identity on
    $L$, so $\vphi_{1}\vphi_{2}^{-1}\in N$.
    It then follows that $L=G/N$.
\end{block}

Returning to our original diagram in Figure 1, recall that
$\bQ[\beta_{1},\dots,\beta_{m}]/\bQ$ is Galois.
Then it is the splitting field of some polynomial $f$ with coefficients in $\bQ$,
but then for all $1\leq k\leq m$, notice
$\bQ[\beta_{1},\dots,\beta_{m}]/\bQ[\beta_{1},\dots,\beta_{k}]$ is also Galois
using the same polynomial $f$.
Therefore, every extension shown in Figure 3 is Galois.

The extensions shown on the left sides being Galois requires a slight amount of additional work,
but the details are roughly similar to our prior construction.
Due to length, they have been omitted.

Consider the groups $G_{0},\dots,G_{m}$ as labeled above.
From our previous work, we can conclude that $G_{k+1}\ngrp G_{k}$ and
$G_{k}/G_{k+1}$ is abelian for each $k$.
In other words, we have the following series:
\begin{equation*}
    G_{m} \ \ngrp \ G_{m-1} \ \ngrp \ \cdots \ \ngrp \ G_{2} \ \ngrp \ G_{1}
    \ \ngrp \ G_{0}=G
\end{equation*}
where each composition factor is abelian.
Therefore, we can conclude that $G$ is solvable.
By the fourth isomorphism theorem (also known as the correspondence/lattice
theorem), it follows that $G/N$ is solvable as well.
Since $G/N$ is determined by our original $p(x)$,
we have proven that if $p(x)$ is solvable by radicals, then the
Galois group of its splitting field is necessarily solvable.

It now suffices to find a quintic for which the Galois group of its splitting
field is \textit{not} solvable. Indeed, consider:
\begin{equation*}
    p(x) \ = \ x^{5}-x-1
\end{equation*}
This polynomial has Galois group $S_{5}$, the proof for which is omitted.
We claim that $S_{5}$ is not a solvable group; indeed, a composition series
is given by:
\begin{equation*}
    \set{e} \ \ngrp \ A_{5} \ \ngrp \ S_{5}
\end{equation*}
The composition factors are thus $A_{5}$ and $\bZ_{2}$, and by the Jordan-Holder
theorem, any composition series of $S_{5}$ necessarily has $A_{5}$ as a
composition factor.
However, $A_{5}$ is not abelian, so $S_{5}$ is not solvable,
concluding our proof.

