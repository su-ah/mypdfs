\subsection{Straightedge and Compass Constructions}
\lecdate{Lec 42 - Mar 21 (Week 23)}

Back in Ancient Greece, mathematicians only used straight edges and compasses to
draw lines and curves. Using a straightedge and compass, we can find the midpoint
between any 2 points in the plane:

\centering
\scalebox{.95}{\incfig{midpoint}}
\flushleft

Similarly, given a line and a point not on it, we can draw a line parallel to
the line which passes through the point:

\centering
\scalebox{.95}{\incfig{ptparaperp}}
\flushleft

We're concerned with three particularly famous problems related to such
counstructions:
\begin{enumerate}[ ]
    \item Doubling the cube (Find $ \sqrt[3]{2} $):
        
        \centering
        \scalebox{.92}{\incfig{doublecube}}
        \flushleft
        
    \item Squaring the circle (Find $ \sqrt{\pi} $):
        
        \centering
        \scalebox{.92}{\incfig{squarecirc}}
        \flushleft
        
    \item Trisecting the angle (Find $ \cos (\theta/3) $):
        
        \centering
        \scalebox{.92}{\incfig{angletrisect}}
        \flushleft
        
\end{enumerate}
We will work in the plane $ \bR^{2} $ writing points as $ (x, y) $.
We assert a line to be of \textbf{unit length} from $ (0, 0) $ to $ (1, 0) $.
When we find points of intersection of 2 lines, circles or a line and circle,
we get new points whose coordinates can be described.
\begin{defn}
    A number is called a \textbf{constructible number} if it is possible to
    construct a line segment of the corresponding length using only a
    straightedge and compass in $ \bR^{2} $.
\end{defn}
The set of constructible numbers are closed under addition, subtraction,
multiplication, and inverses:

\centering
\scalebox{.95}{\incfig{constructibles}}
\flushleft

Thus, the set of constructible numbers is in fact a \textit{field}.
The point of intersection of two lines is also constructible, and:

\centering
\scalebox{.95}{\incfig{intersections}}
\flushleft

So what does this all mean?

Any of these basic steps can produce coordinates which are either in the same
field as the original diagram, \textit{or} in a quadratic extension.
A sequence of such basic steps that guarantee a field of degree $ 2^{k} $ over
$ \bQ $:

\centering
\scalebox{.95}{\incfig{2ktower}}
\flushleft

But this tells us something: Say we could double a cube. Then, the sides of this
new cube would be length $ \sqrt[3]{2} $, meaning some coordinates would be in
$ \bQ[\sqrt[3]{2}] $. By Eisenstein, $ x^{3}-2 $ is irreducible, so we get:

\centering
\scalebox{.95}{\incfig{3root2}}
\flushleft

So we see $ \bQ[\sqrt[3]{2}] $ \textit{cannot} be in any field created by basic
steps. Thus, we conclude that doubling the cube is in fact \textit{impossible}
with a straightedge and compass.
(brandon note: ``what crack was the guy who solved this smoking when realized
this?? i need some")

Another result is that since $ \pi $ is transcendental, so is $ \sqrt{\pi} $, so
we have that $ [\bQ[\pi]:\bQ] \ = \ \infty, $ which is even worse than before.
Okay, what about trisecting angles? Can we construct $ \cos(\theta/3) $?

Turns out, the answer is ``sometimes", but not in general. For example,
it happens that $ \cos(20\degree/3) $ satisfies an irreducible cubic, so this
can't be constructed. However, any angle which is a multiple of $ 3\degree $
\textit{can} be trisected.

\lecdate{Lec 43 - Mar 26 (Week 24)}
Pierre Waltsen solved the constructibility problems.

\begin{defn}
    Suppose $ f(x) \in \bF[x] $. An extension $ \bE/\bF $ is a
    \textbf{splitting field} for $ f $ if $ f $ ``splits completely", i.e.:
    \begin{equation*}
        f(x) = (x-\alpha_{1})(x-\alpha_{2})\cdots(x-\alpha_{n})
    \end{equation*}
    and $ f(x) $ does \textit{not} split for any subfield of $ \bE $.
\end{defn}

\begin{xmp}[source=Primary Source Material]
    For $ f(x)=x^{2}-2 \in \bQ[x] $, its splitting field is
    $ \bQ[\sqrt{2}]/\bQ $.
    Note that both $ \pm\sqrt{2} \in \bQ[\sqrt{2}]/\bQ $. \vsp
    %
    For $ g(x) = x^{2}+1 \in \bC[x] $, it already splits, so $ \bC $ is its
    splitting field.
    For $ g(x) \in \bR[x] $, its splitting field is $ \bC/\bR $. \vsp
    %
    For $ x^{3}-2 \in \bQ[x] $, its splitting field is
    $ \bQ[\sqrt[3]{2}, \sqrt{-3}] $, an extension of $ \bQ $ with degree 6. \vsp
    %
    For $ h(x)=x^{4}+4 \in \bQ[x] $, it has roots $ 1 \pm i,-1\pm i $.
    Thus, its splitting field is $ \bQ[i]/\bQ $.
\end{xmp}

\begin{thm}
    Given $ f(x) \in \bF[x] $, there exists a splitting field for $ f $.
\end{thm}

\begin{pf}[source=Primary Source Material]
    If $ \deg(f) = 1 $, then $ \bF $ is clearly the splitting field.
    Furthermore, $ \bF $ is the splitting field for any polynomial that splits
    in $ \bF $. \vsp
    %
    Assume by induction that the result holds for polynomials $ g $ where
    $ \deg(g) < n = \deg(f) $. We know we can construct $ \bE/\bF $ such that
    if $ \alpha \in \bE $ and $ f(\alpha) = 0 $, then $ f(x)=(x-\alpha)g(x) $
    for some $ g $. Furthermore, $ \deg(g) < n $, so let $ \bE $ be the splitting
    field of $ g $. \vsp
    %
    Let $ \bK = \bE[\alpha] $. Then, we see that $ f $ splits in $ \bK $.
    To take a minimal field, take the intersection of all fields in which $ f $
    split.
\end{pf}

\begin{defn}
    An extension $ \bE/\bF $ is \textbf{normal} if $ \bE $ is a splitting field
    for some $ f(x) \in \bF[x] $; that is, it is generated by the roots of $ f $.
\end{defn}

\begin{xmp}[source=Primary Source Material]
    The field $ \bQ[\sqrt[3]{2}]/\bQ $ is \textit{not} normal. \vsp
    %
    Note $ \sqrt[3]{2} $ is a root of $ x^{3}-2 $, but the \textit{other} roots
    $ \omega\sqrt[3]{2}, \omega^{2}\sqrt[3]{2} $ are not in $ \bQ[\sqrt[3]{2}]
    \subseteq \bR $. \vsp
    %
    The normal extension is $ \bQ[\sqrt[3]{2}, \sqrt{-3}] $.
\end{xmp}

Question: if $ f(x) \in \bF[x] $, can it have two different splitting fields?
(should really ask - can it have two \textit{non-isomorphic} splitting fields?)
Answer: unsurprisingly, no.

Suppose $ \vphi: \bF \goesto \bF' $ is an isomorphism, and $ f(x) \in \bF[x] $.
It induces a map $ \bF[x] \goesto \bF'[x] $.

If $ \bE/\bF, \bE'/\bF' $ are splitting fields for $ f, f' $ respectively, then
$ \vphi $ extends to an isomorphism $ \bE \goesto \bE' $.
The proof from here follows by a similar induction argument as above.

Thus, we can indeed talk about \textit{the} splitting field of $ f $ over
$ \bF $.

Observe that the splitting field of $ f $ is $ \bF[\alpha_{1}, \alpha_{2},
\dots, \alpha_{n}] $, where the $ \alpha_{i} $'s are the roots of $ f $.
Since $ \bF[\alpha_{1}, \dots, \alpha_{n}] = \bF[\alpha_{1}][\alpha_{2}]\dots
[\alpha_{n}] $, note that:
\begin{equation*}
    [\bF[\alpha_{1}]:\bF] \leq n = \deg(f)
\end{equation*}
In this field, $ f(x) = (x-\alpha_{1})f_{1}(x) $, where $ \deg(f_{1}) = n-1 $.
Then:
\begin{equation*}
    [\bF[\alpha_{2}]:\bF[\alpha_{1}]] \leq n-1
\end{equation*}
As we keep going, we see that:
\begin{align*}
    [\bF[\alpha_{1}, \dots, \alpha_{n}]: \bF] & \ = \
    [\bF[\alpha_{1}]:\bF]\cdot[\bF[\alpha_{1}\alpha_{2}]:\bF[\alpha_{1}]] \cdots
    [\bF[\alpha_{1}, \dots, \alpha_{n}]:\bF[\alpha_{1},\dots,\alpha_{n-1}]] \\
        & \leq n(n-1)(n-2)\dots 2\cdot 1 = n!
\end{align*}

\begin{xmp}[source=Primary Source Material]
    Notice $ [\bF[\sqrt{\alpha}]:\bF] $ is equal to:
    $ \begin{cases} 2 & d \trm{ not a square} \\ 1 & d \trm{ square}
    \end{cases} $.
    And indeed, $ 2 = 2! = \deg(x^{2}-\alpha) $. \vsp
    %
    Notice $ [\bQ[\sqrt[3]{2}], \sqrt{-3}] = 6 = 3! = \deg(x^{3}-2) $. \vsp
    %
    For $ x^{4}-4, 4! = 24 $. Then:
    \begin{equation*}
        [\trm{splitting field}:\bQ] \ = \ [\bQ[i]:\bQ] = 2 \neq 24
    \end{equation*}
    So ``most" irreducible polynomials have splitting fields of degree
    $ = (\deg(f))! $.
\end{xmp}

