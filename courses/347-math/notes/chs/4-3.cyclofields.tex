\subsection{Cyclotomic Fields (Roots of Unity)}
Consider $ x^{n}-1 \in \bQ[x] $. Clearly:
\begin{equation*}
    (x^{n}-1) = (x-1)(x^{n-1}+x^{n-2}+\dots+x+1)
\end{equation*}
As complex numbers, we know what these roots look like:

[roots]
%uhh e^2i\pi k/n, k in [0, n-1]
%eg: n = 9
%``evenly spaced pts arnd unit circle in C"

If $ n = p $ is prime, we have previously shown that $ x^{p-1}+\dots+x+1 $ is
irreducible, meaning:
\begin{equation*}
    [\bQ[e^{2\pi i/p}]:\bQ] = p-1
\end{equation*}
Note: $ \bQ[e^{2\pi i/p}] $ contains every power $ e^{2\pi ik/p} $, so it is
the splitting field of $ x^{p}-1 $. In particular:
\begin{equation*}
    [\bQ[e^{2\pi i/p}]:\bQ] = p-1 < (p-1)! \qquad (\trm{unless }p=2,3)
\end{equation*}

[doubleroots]
%n=3, n=6
%$ [\bQ[\sqrt{-3}]:\bQ] = 2 = p-1 = (p-1)! $
%1/2 + sqrt3/2 i

When we add $ n=6 $, we get the mirror image. Notice that the $ \sqrt{3} $ is
the same in the expression. When adding the 6th roots of unity, the splitting
field does not increase.

\begin{defn}
    A root $ \zeta $ of $ x^{n}-1 $ is a \textbf{primitive $ n $th root of unity}
    if it generates the multiplicative group of all $ n $th roots of unity.
\end{defn}

As it turns out, the number of primitive roots is $ \vphi(n) $. A little less
obvious is that this is also the degree of the splitting field.

\begin{xmp}[source=Primary Source Material]
    For $ n=4 $, we have for $ x^{4}-1 $:
    
    [runningoutofnames]

    Notice $ -1 $ is a root of $ x^{2}-1 $ (and so is $ 1 $). In particular:
    \begin{equation*}
        x^{2}-1 \mid x^{4}-1 \qquad x^{4}-1=(x^{2}-1)(x^{2}+1)
    \end{equation*}
    By factoring, we get the primitive roots of unity. \vsp
    %
    Let's try this with $ n=6 $:
    \begin{equation*}
        x^{6}-1 \ = \ (x^{3}-1)(x^{3}+1) \ = \ (x^{3}-1)(x+1)(x^{2}-x+1)
    \end{equation*}
    Indeed, $ x^{2}+x+1 $ gives the primitive 6th roots.
\end{xmp}
It's common to write $ \zeta_{n} $ for a primitive $ n $th root of unity.
We tend to think of $ \zeta_{n} = e^{2\pi i/n} $, but it could be any primitive
root.

If $ \vphi: \bQ[\zeta_{p}] \goesto \bQ[\zeta_{p}] $ is an automorphism, it fixes
the elements of $ \bQ $. What does it do to $ \zeta_{p} $?
Note $ \vphi(\zeta_{p}) $ has to be a $ p $th root of unity, and not $ 1 $.
That is: $ \vphi(\zeta_{p}) = \zeta_{p}^{a} $, for some $ a = 1, \dots, p-1 $.
So:
\begin{equation*}
    \vphi(\zeta_{p}^{k}) = (\zeta_{p}^{a})^{k} = (\zeta_{p}^{k})^{a}
\end{equation*}
The automorphisms of $ \bQ[\zeta_{p}] $ take each non-trivial $ p $th root of
unity to a power of itself. Furthermore, it is easy to see that the number of
automorphisms is the same as the degree of the extension.

\lecdate{Lec 44 - Mar 28 (Week 24)}
Recall that a primitive $ n $th root of unity is a generator for the group of
$ n $th roots; we denote the whole group by $ \mu_{n} $.
Clearly, if $ m \mid n $, then any $ m $th root of unity is also an $ n $th root.
Symbolically, $ \mu_{m} \subseteq \mu_{n} $.

\begin{defn}
    We denote by $ \Phi_{n}(x) $ the monic polynomial of degree $ \vphi(n) $
    whose roots are precisely the primitive $ n $th roots of unity.
\end{defn}

Since the primitive $ m $th roots are in $ \mu_{n} $ if $ m \mid n $, then we
have that:
\begin{equation*}
    \Phi_{n}(x) \mid (x^{n}-1)
\end{equation*}

\begin{crll}
    Every $ n $th root of unity is a primitive $ m $th root for some
    $ m \mid n $. Furthermore:
    \begin{equation*}
        x^{n}-1 \ = \ \prod_{m\mid n}\Phi_{m}(x)
    \end{equation*}
\end{crll}

\begin{xmp}[source=Primary Source Material]
    For the first few $ n $, we have:
    % TODO figure out this thing
    \begin{enumerate}
        \item $ \Phi_{1}(x) = x-1 $
        \item $ \Phi_{2}(x) = x+1 $
        \item $ \Phi_{3}(x) = (x-\omega)(x-\bar{\omega}) = x^{2}+x+1 $
        \item $ \Phi_{4}(x) = \frac{x^{4}-1}{\Phi_{1}(x)\Phi_{2}(x)}
            = \frac{x^{4}-1}{x^{2}-1} = x^{2}+1 $
        \item $ \Phi_{5}(x) = \frac{x^{5}-1}{\Phi_{1}(x)} = x^{4}+\dots+x+1 $
        \item $ \Phi_{6}(x) = \frac{x^{6}-1}{\Phi_{1}\Phi_{2}\Phi_{3}}
            = \cdots = x^{2}-x+1 $
    \end{enumerate}
    For any prime $ p $:
    \begin{equation*}
        \Phi_{p}(x) \ = \ \frac{x^{p}-1}{x-1} \ = \ x^{p-1}+\dots+x+1
    \end{equation*}
\end{xmp}
The cyclotomic fields and hence the cyclotomic polynomials are important for
various reasons; many interesting fields are contained within cyclotomic fields.

\begin{defn}
    Given a polynomial $ f(x) = (x-\alpha_{1})^{m_{1}}\cdots
    (x-\alpha_{n})^{m_{n}} $ where each $ \alpha $ is distinct, the $ m_{i} $'s
    are called the \textbf{multiplicity} of the corresponding root $ a_{i} $'s.
    \vsp
    If $ m_{i}=1 $ for all $ i $, we say that $ f $ is \textbf{multiplicity
    free}, or \textbf{separable}. Otherwise, $ f $ is \textbf{inseparable}.
\end{defn}

\begin{lm}
    Over a field $ \bF $ with $ \fchar(\bF) = 0 $, any irreducible polynomial is
    separable.
\end{lm}
Funny things happen in $ \fchar(\bF) = p > 0 $.

\begin{xmp}[source=Primary Source Material]
    Consider $ \bF_{2}[x] $, and let $ f(x) = x^{2}-t $ for some variable $ t $.
    Then, in some extension field:
    \begin{equation*}
        (x-\sqrt{t})(x+\sqrt{t}) = (x+\sqrt{t})^{2}
    \end{equation*}
    So, we see it is inseparable. But, in $ \bF_{2}(x) $, we see that $ f $ is
    indeed irreducible.
\end{xmp}

In a field of prime characteristic, notice that:
\begin{equation*}
    (a+b)^{p} \ = \ a^{p} + \underbrace{\qquad \qquad \qquad \qquad}_{\trm{
            coeff's here divisible by }p} + b^{p}
\end{equation*}
In other words, $ (a+b)^{p} = a^{p}+b^{p} $.
Since $ (ab)^{p}=a^{p}b^{p} $, then $ x \mto x^{p} $ is a field homomorphism.
If $ \bF $ is finite, it is injective, and therefore an automorphism. This is
called the \textbf{Frobenius automorphism}.
Since every $ x \in \bF $ is a $ p $th power, we can write $ f(x)=g(x^{p}) $ for
some $ g $.

Over $ \bR $, a polynomial $ f $ has $ a $ as a repeated root with multiplicity
$ k $ if:
\begin{equation*}
    f(a) \ = \ 0 \ = \ f'(a) \ = \ \dots \ = \ f^{(k-1)}(a)
\end{equation*}
Note that this also works with $ \fchar p $.

Consider $ f(x)=x^{p}-x $; the roots are things that equal their $ p $th power.
Then:
\begin{equation*}
    f'(x) = px^{p-1}-1 = 0 - 1 = -1 \neq 0
\end{equation*}
So we see that $ f $ has no repeated roots, and thus is separable.
Over $ \bF_{p} $, consider $ f(x)=x^{p^{k}} - x $. Again, $ f'(x) = -1 \neq 0 $.
So $ f(x) $ is separable, and thus has $ p^{k} $ roots in some extension field.
The set of roots of $ f $ is a field $ \bF_{p^{k}} $ of order $ p^{k} $.

\begin{thm}
    Every finite field is $ \bF_{p^{k}} $ for some prime $ p $ and integer
    $ k \geq 1 $.
\end{thm}
Of course $ \bF_{p} = \bZ/p\bZ $, but $ \bF_{p^{2}} \neq \bZ/p^{2}\bZ $, etc.
For each $ p $, we get a tower of fields:

[fieldtree]

Given a field $ \bF $, we can consider $ \Aut(\bF) $.
For any $ \vphi \in \Aut(\bF) $, note that:
\begin{equation*}
    \vphi(1) \ = \ 1 \qquad \vphi(2) \ = \ \qquad \vphi(n) = n \quad n \in \bZ
\end{equation*}
So $ \vphi $ fixes $ \bZ $ and hence fixes $ \bQ $, if $ \fchar(\bF) = 0 $.
Otherwise, it fixes $ \bZ/p\bZ $ if $ \fchar(\bF) = p $.

If $ \bE/\bF $ is an extension, we write $ \Aut(\bE/\bF) $ for the automorphisms
of $ \bE $ which fix $ \bF $ pointwise. For instance:
\begin{gather*}
    \Aut(\bC/\bR) = \set{\trm{id}, z \mto \bar{z}} \\
    \trm{But } \Aut(\bC) \trm{ is enormous.}
\end{gather*}

