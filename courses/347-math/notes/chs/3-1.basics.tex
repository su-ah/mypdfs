\subsection{Basics}

Suppose $ R $ is a ring. We want to generalize the idea of a vector space over a field, replacing
the field with our ring $ R $.

\begin{defn}
    A \textbf{left} $ R $\textbf{-module} is an abelian group $ M $ with operation $ + $,
    together with a left action of $ R $ on $ M $ satisfying:
    \begin{itemize}
        \item $ r(m+m') = rm + rm' $ for all $ r \in R, m, m' \in M $
        \item $ (r+s)m = rm + sm $ for all $ r, s \in R, m \in M $
        \item $ (rsm) = r(sm) $ for all $ r, s \in R, m \in M $
    \end{itemize}
    Usually $ R $ has a unit, and if so, we also require that $ 1m = m $ for all $ m $.
\end{defn}

\begin{crll}
    If $ R $ is a field, then this is a vector space. :)
\end{crll}

\begin{xmp}[source=Primary Source Material]
    Any ring $ R $ is a left $ R $-module:
    \begin{equation*}
        r \cdot r' = rr'
    \end{equation*}
    Similarly, $ R^{n} $ is a left $ R $-module with componentwise operations.
\end{xmp}

If $ M $ is any abelian group and $ r \in \bb{Z} $, we write:
\begin{equation*}
    r \cdot m \ = \ \underbrace{m + m + \dots + m}_{r \trm{ times}} \qquad
    (-r)m = r(-m)
\end{equation*}
Thus, any abelian group is a $ \bb{Z} $-module.
In fact, $ \bb{Z} $-modules are the same things as abelian groups.

\lecdate{Lec 35 - Feb 14 (Week 18)}

How are modules different from vector spaces?
A main difference is that we can't always find a basis.

If $ R $ is a ring and $ M_{1}, \dots, M_{k} $ are (left) $ R $-modules, then
we have already discussed the direct product:
\begin{equation*}
    M_{1} \times \cdots \times M_{k} = \set{(m_{1}, \dots, m_{k}) : m_{i} \in M_{i}}
\end{equation*}
with component-wise operation.
Note this definition also works with infiniteley many factors $ M_{i}, i \in I $.

For $ R $-modules $ A, B $, we also defined $ A + B $ as:
\begin{equation*}
    A + B = \set{a + b : a \in A, b \in B}
\end{equation*}
This is also an $ R $-submodule of $ M $.
We can similarly extend the definition to finitely many summands.

Given $ R $-submodules $ A_{1}, \dots, A_{k} \subseteq M $, we can define a map:
\begin{gather*}
    A_{1} \times \cdots \times A_{k} \rightarrow A_{1} + \dots + A_{k} \\
    (a_{1}, \dots, a_{k}) \rightarrow a_{1} + \dots + a_{k}
\end{gather*}

\begin{thm}
    The following are equivalent:
    \begin{itemize}
        \item $ A_{i} \cap (A_{1} + \dots + \hat{A_{i}} + \dots + A_{k}) = \set{0} $ for all $ i $.
            (The hat means exclude that summand.)
        \item For any $ a \in A_{1} + \dots + A_{k} $, there exists a unique $ a_{i} \in A_{i} $
            such that $ a = a_{1}+\dots+a_{k} $.
    \end{itemize}
\end{thm}
\begin{pf}
    easy to prove
\end{pf}
In the above situation, we'll say $ A_{1} + \dots + A_{k} $ is a \textbf{direct sum},
notated by $ A_{1} \oplus \cdots \oplus A_{k} $.
It follows that $ A_{1} \oplus \cdots \oplus A_{k} \simeq A_{1} \times \cdots \times A_{k} $.

Note that if there are infinitely many $ A_{i} $'s, the above will still work, but the direct sum
are \textit{not} isomorphic to the direct product.

If $ G \subseteq M $ is a subset of an $ R $-module $ M $, we write:
\begin{equation*}
    RG = \set{\sum_{i=1}^{k}r_{i}g_{i} : k \in \bb{N}, r_{i} \in R, g_{i} \in G}
\end{equation*}
We say that $ RG $ is the $ R $-submodule generated by $ G $, and $ G $ is a set of generators
for $ RG $. Note that there may be many different sets of generators.
If $ G = \set{g} $, then we write $ Rg $ for $ RG $, and $ Rg $ is called a \textbf{cyclic module}.

\begin{xmp}[source=Primary Source Material]
    Let $ R = 2\bb{Z} $. Clearly this is a nonunital ring. \vsp
    %
    Notice that $ M = \bb{Z} $ is an $ R $-module, and:
    \begin{equation*}
        R1 = \set{2m \cdot 1 : m \in \bb{Z}} = 2\bb{Z}
    \end{equation*}
    So $ R1 $ is a submodule of $ M $, which does not contain $ 1 $, its generator.
    This happened because $ R $ has no unit, so \textit{usually} we will work with unital rings.
\end{xmp}

Suppose an $ R $-module $ M $ satisfies:
\begin{equation*}
    M = R_{m_{1}} + \dots + R_{m_{k}}
\end{equation*}
Need some notion of independence of the $ m_{i} $'s to get an analogue of a basis in a vector space.
How about:
\begin{equation*}
    M = R_{m_{1}} \oplus \dots \oplus R_{m_{k}}
\end{equation*}
This comes closer, but is still not good enough.
\begin{xmp}[source=Primary Source Material]
    Consider $ R = \bb{Z}, M = \bb{Z}/3\bb{Z} $. \vsp
    %
    Clearly, $ M $ is a cyclic module, and so it is just one summand.
    But $ 3 \cdot \bar{1} = 0 $. Furthermore:
    \begin{equation*}
        2 \cdot \bar{1} \ = \ 5 \cdot \bar{1} \ = \ 8 \cdot \bar{1} \ = \ \dots
    \end{equation*}
    So we see two different reasons why the above will not work.
\end{xmp}
As a group, we can think of the above being represented with a relation $ x^{3} = 1 $, or in
additive notation, $ 3x = 0 $. Thus, we want to avoid having relations amongst the generators.

\begin{defn}
    We say an $ R $-module $ M $ is a \textbf{free module on a set of generators} $ A $ iff
    every $ m \in M $ can be written as:
    \begin{equation*}
        a = r_{1}a_{1} + \dots + r_{k}a_{k}
    \end{equation*}
    for finitely many $ a_{i} \in A $, such that the nonzero $ a_{i} $'s and corresponding
    $ r_{i} $'s are uniquely determined. \vsp
    %
    The number of generators of a free module $ M $ is called the \textbf{rank} of $ M $.
\end{defn}
Free modules are a lot like vector spaces, but special.
In the proof that all vector spaces have bases, we require that we can divide by our field elements;
we can't expect to do the same in an arbitrary ring.

\begin{thm}
    Given a set $ A $ and a ring $ R $, there is a module $ F(A) $ (the free module on $ A $)
    which has the following defining property: \vsp
    %
    Suppose $ M $ is an $ R $-module, and there is a map $ \vphi:A \rightarrow M $.
    Then, $ A $ is a subset of $ F(A) $ and there exists $ \Phi: F(A) \rightarrow M $ such
    that $ \vphi = \Phi \circ \iota $. In other words, the following diagram commutes:
    
    \centering
    \scalebox{.95}{\incfig{not-fit}}
    \flushleft
    
\end{thm}

\begin{pf}[source=Primary Source Material]
    Sketch of proof:
    \begin{equation*}
        F(A) = \set{f: A \rightarrow R : f(a) \neq 0 \trm{ for finitely many } a \in A}
    \end{equation*}
    This is an $ R $-module by multiplication by $ R $. \vsp
    %
    If $ f(a_{i}) = r_{i} $ for $ i = 1, \dots, k $ and $ f(a) = 0 $ otherwise,
    then $ \Phi(f) = \sum_{i=1}^{k} r_{i}\vphi(a_{i}) $.
\end{pf}
