\subsection{Actions}

\lecdate{Lec 3 - Sept 11 (Week 2)}

\begin{defn}
    A group $ G $ is said ``to have an action on $ X $" or ``to act on $ X $",
    and $ X $ is said ``to be acted on by $ G $",
    if there is a map $ \vphi: G \times X \rightarrow X $ which satisfies:
    \begin{itemize}
        \item $ \vphi(e, x) = x $
        \item $ \vphi(gh, x) = \vphi(g, hx) $
    \end{itemize}
    for all $ x \in X $, $ g, h \in G $.
\end{defn}

Note that the left action maps $ (g, x) \rightarrow gx $.

\begin{xmp}[source=Primary Source Material]
    \begin{itemize}
        \item $ \GL_{3}(\bb{R}) $ acts on $ \bb{R}^{3} $
        \item $ \SL_{3}(\bb{R}) $ acts on $ \bb{R}^{3} $ by rotations and reflections
        \item $ \SO_{3}(\bb{R}) $ acts on $ \bb{R}^{3} $ by rotations,
            but also acts on the sphere of radius $ r $ about the origin
    \end{itemize}
\end{xmp}

Note the definition above is specifically for \textit{left} actions -
right group actions \textit{do not work}.
This is because it fails the associativity axiom!
We can fix it by mapping $ (g, x) \rightarrow xg^{-1} $ instead.

\begin{xmp}[source=Primary Source Material]
    We can think of $ \SO_{2} < \SO_{3} $ as rotations around the earth's poles. \vsp
    %
    Indeed, if you pick a point and draw it after being acted on, you get the latitude lines.
    The two exceptions are the points at the poles themselves.
\end{xmp}

\begin{defn}
    The \textbf{orbit} of $ x \in X $ under $ G $ is $ Gx = \textrm{orb}(x) = \set{gx : g \in G} $.
\end{defn}

\newpage
\lecdate{Lec 4 - Sept 13 (Week 2)}

``get rotated" me to a square today

\begin{crll}
    Suppose $ G $ is a group.
    Then $ G $ acts on itself if $ X = G $. \\
    The group action then becomes a function $ G \times G \rightarrow G $. \vsp
    %
    If the action is a left action, it is called a \textbf{left translation}, sometimes denoted as
    \begin{equation*}
        L_{g}x = L_{g}(x) = gx
    \end{equation*}
    For fixed $ g $, we get a map $ G \rightarrow G $ as $ x \rightarrow gx $.
    We can denote a \textbf{right translation} similarly.
\end{crll}

We consider one more action:
\begin{equation*}
    C_{g}(x) = L_{g}R_{g} = R_{g}L_{g}(x) = gxg^{-1}
\end{equation*}
This is called \textit{conjugation} by $ g $.

\begin{defn}
    Suppose $ G $ acts on $ X $.
    For fixed $ x \in X $, the \textbf{stabilizer} of $ x $ is:
    \begin{equation*}
        \stab_{G}(x) = \set{g \in G : gx = x}
    \end{equation*}
    Note that $ \stab (x) \leq G $.
\end{defn}

Consider the left coset of $ \stab(x) $.
Then $ g\cdot\stab (x) $ consists of all $ k \in G $ such that $ kx = gx $.

\lecdate{Lec 5 - Sept 18 (Week 3)}

Suppose that $ \sigma_{g} $ represents conjugation by $ g $.
Then, $ \sigma(ab) = \sigma(a)\sigma(b) $, but note that this doesn't hold for
left/right group actions in general.
Also note that for any subgroup $ H $, we have that $ gHg^{-1} $ is also a subgroup of $ G $.

Now, consider the stabilizer of some element $ h \in G $.
Clearly, $ g \in \stab(h) \iff ghg^{-1} = h \iff gh = hg $.
In other words, the stabilizer of the conjugation action is called the centralizer.

Some examples of centers:
\begin{itemize}
    \item The center of $ D_{3} $ is trivial, but the center of $ D_{4} = \set{e, \rho^{2}} $.
    \item The center of $ Q_{8} $ is given as $ \set{1, -1} $.
    \item The center of $ \GL_{n}(\bb{F}) = \set{cI} \forall \,  c \in \bb{F} $.
    \item The center of $ \SL_{n}(\bb{C}) = \set{cI : c^{n} = 1} \forall \, c \in \bb{C} $.
        Notice that $ c $ is given by the $ n $-th roots of unity.
\end{itemize}
Clearly, centers can be subtle.

\newpage
yaaaay quotient groups :)

okay heres something i should try and prove cause its just. weird to me
\begin{lm}
    Suppose $ H \leq G $, and $ x, y \in G $. \vsp
    %
    Show that $ xH = yH $ if and only if $ y \in xH $.
\end{lm}

\begin{pf}
    ($ \Longleftarrow $) Suppose that $ y \in xH $.
    We want to show that $ yH = xH $. \vsp
    %
    Since $ y \in xH $, then $ y = xh $ for some $ h \in H $.
    Notice that $ yH = xhH $. We see that:
    \begin{align*}
        a & \in xhH \\
        \implies a & = xhh' \\
        \implies a & \in xH
    \end{align*}
    So we see that $ yH \subseteq xH $.
    We also see that $ y = xh \implies yh^{-1} = x \implies x \in yH $.
    So by a similar argument, we also have that $ xH \subseteq yH $,
    so they indeed must be equal. \vsp
    %
    ($ \implies $) Suppose $ yH = xH $.
    To see that $ y \in xH $, notice that for some $ h, h' \in H $:
    \begin{align*}
        yh & = xh' \\
        \implies \ y & = xh'h^{-1} \\
        \implies \ y & \in xH
    \end{align*}
    So clearly $ y \in xH $ as needed.
\end{pf}
