\subsection{Terminology and Examples}

\begin{thm}[type=Remark,num=1.19]
    $M$ noeth and artin iff $M$ has a comp series.
\end{thm} \

\begin{pf}
	Suppose $M$ is both noetherian and artinian.
	Then, $M$ has a maximal submodule $N_{1}$, by considering all
	proper submodules of $M$ under inclusion.
	Similarly, $N_{1}$ has such a maximal submodule $N_{2}$.

	Indeed, for any $k$, let $N_{k}$ be the maximal submodule of $N_{k-1}$.
	Since any submodule if $M$ is noetherian, this is well-defined.
	Thus, we have a chain of submodules:
	\begin{equation*}
	    \cdots < N_{3} < N_{2} < N_{1} < M
	\end{equation*}
	Since $M$ is artinian, there is some $k$ such that $N_{k}=N_{k+1}=\dots$.
	In other words, the chain terminates:
	\begin{equation*}
		\set{0}=N_{k}<N_{k-1}<\dots<N_{2}<N_{1}<M
	\end{equation*}
	Since each $N_{k}$ is maximal in $N_{k-1}$ by construction, then
	$N_{k}/N_{k-1}$ is simple for all $k$.
	Thus, $M$ has a composition series as needed.
\end{pf} \

\newpage
\begin{thm}[type=Remark,num=1.20]
    $N$ submod of $M$.
	then $M$ noeth iff $N,M/N$ noeth
\end{thm} \

\begin{pf}
	First, notice that if $N$ and $A/N$ are f.g., then so is $A$.
	Indeed, suppose:
	\begin{equation*}
	    N=\ang{x_{1},\dots,x_{n}} \qquad
		A/N=\ang{\bar{y}_{1},\dots,\bar{y}_{m}}
	\end{equation*}
	Let $a\in A$. Then:
	\begin{equation*}
	    \bar{a} \ = \ \sum_{i=1}^{m}s_{i}\bar{y}_{i} \ = \
		\bar{\sum_{i=1}^{m}s_{i}y_{i}} \ \implies \
		a - \sum_{i=1}^{m}s_{i}y_{i} \ = \ \sum_{j=1}^{n}r_{j}x_{j}
	\end{equation*}
	Thus, we have that $A=\ang{x_{1},\dots,x_{n},y_{1},\dots,y_{m}}$.

	Now, suppose $N$ and $M/N$ are noetherian.
	Let $A\leq M$ be any submodule.
	By the Second Isomorphism Theorem, we have:
	\begin{equation*}
		\frac{A}{A\cap N} \simeq \frac{A+N}{N} \leq \frac{M}{N}
	\end{equation*}
	Since $A\cap N \leq N$ is f.g. and $A/(A\cap N)\leq M/N$ is f.g.,
	by the above, it follows that $A$ is f.g., and thus $M$ is noetherian.

	Next, suppose $M$ is noetherian.
	Then $N$ is clearly noetherian, so let $A$ be any submodule of $M/N$.
	By the Fourth(*) Isomorphism Theorem, $A\simeq L/N$ for some submodule
	$N\leq L\leq M$.
	
	But since $M$ noetherian, $N$ and $L$ are finitely generated, so
	it follows that $A$ is finitely generated.
\end{pf} \

\begin{prop}[num=1.21(b)]
    $R$ right noeth, $M_{R}$ f.g.. then $M_{R}$ noeth.
\end{prop} \

\begin{pf}
    Suppose $M=\ang{m_{1},\dots,m_{n}}$ for some $n$. Then:
	\begin{equation*}
	    m \ = \ \sum_{i=1}^{n}r_{i}m_{i}
	\end{equation*}
	for all $m \in M$.

	Consider $R^{n}=R\times R\times \cdots \times R$.
	We have a map $\vphi:R^{n}\gto M$ given by:
	\begin{equation*}
	    \vphi(r_{1},\dots,r_{n}) \ = \ \sum_{i=1}^{n}r_{i}m_{i}
	\end{equation*}
	Clearly, $\vphi$ is surjective.
	Furthermore, since $R$ is noetherian, $R^{n}$ is as well.
	Therefore, by the First Isomorphism Theorem, we have that
	\begin{equation*}
		\frac{R^{n}}{\ker(\vphi)} \ \simeq \ M
	\end{equation*}
	is noetherian as needed.
\end{pf} \

\newpage
\begin{exr}[num=1.12a,source=Primary Source Material]
    $_{R}M$ is called \textit{Hopfian} if every surj endo is inj.
	show every noeth $_{R}M$ is hopfian.
\end{exr} \

\begin{pf}
	Let $\vphi$ be a surjective endomorphism of $M$.
	Recall that $\ker(\vphi)$ is a submodule of $M$.
	Consider $\vphi^{(2)}:=\vphi\circ\vphi$.
	Since $\vphi$ is surjective, then there exists $m \in M$ such that
	$\vphi(m)\in\ker(\vphi)$.
	Thus, we have that $\ker(\vphi)\leq\ker(\vphi^{(2)})$.

	In general, we can consider $\ker(\vphi^{(k)})$ in the same way.
	Since $\vphi$ is surjective, we have that $im(\vphi^{(k)})=M$,
	so we can construct the following chain:
	\begin{equation*}
	    \ker(\vphi) \leq \ker(\vphi^{(2)}) \leq \ker(\vphi^{(3)}) \leq \cdots
	\end{equation*}
	Since $M$ is noetherian, this chain must terminate;
	if $\ker(\vphi)$ is non-trivial, then the above construction holds.
	Therefore $\ker(\vphi)=\set{0}$, or in other words, $\vphi$ is injective.
\end{pf} \

\begin{exr}[num=1.12b,source=Primary Source Material]
    prove $_{R}R$ is hopfian iff dedekind-finite
\end{exr} \

\begin{pf}[source=Primary Source Material]
	First, suppose $_{R}R$ is Hopfian and fix $x\in R$ such that $xy=1$
	for some $y\in R$.
	It suffices to show that $x$ is left-invertible.

	For any $r\in R$, let $\vphi_{r}(s)=rs$. This is evidently an endomorphism.
	Consider the map $\vphi_{x}$; note that this is surjective, as it has
	a right-inverse:
	\begin{equation*}
		(\vphi_{x}\circ\vphi_{y})(s) \ = \ (xy)s \ = \ s
		\ \implies \
		\vphi_{x}\circ\vphi_{y} \ = \ \vphi_{1} \ = \ \trm{id}
	\end{equation*}
	Since $R$ is Hopfian, then $\vphi_{x}$ is injective.
	By uniqueness of inverses, we have:
	\begin{equation*}
		(\vphi_{y}\circ\vphi_{x})(s) \ = \ (yx)s \ = \ s
	\end{equation*}
	Since this is true of all $s \in R$, then $yx=1$ as needed.

	Now, suppose $R$ is Dedekind-finite.
	Let $\vphi:R\gto R$ be a surjective endomorphism. Notice:
	\begin{equation*}
	    \vphi(r) \ = \ \vphi(1r) \ = \ \vphi(1)r
	\end{equation*}
	Thus $\vphi=\vphi_{\vphi(1)}$ is given by scaling.
	Fix $\vphi(1)=r$.
	Since $\vphi$ is surjective, then there exists some $s\in R$ such that:
	\begin{equation*}
	    \vphi(s)=rs = 1
	\end{equation*}
	Similarly to above, we therefore have that:
	\begin{equation*}
	    \vphi\circ\vphi_{s} \ = \ \vphi_{r}\circ\vphi_{s}
		\ = \ \vphi_{1} \ = \ \trm{id}
	\end{equation*}
	Since $R$ is Dedekind-finite, then:
	\begin{equation*}
	    \vphi_{s}\circ\vphi \ = \ \vphi_{s}\circ\vphi_{r}
		\ = \ \vphi_{1} \ = \ \trm{id}
	\end{equation*}
	It follows that $\vphi$ is injective, and so $R$ is indeed Hopfian.
\end{pf}



