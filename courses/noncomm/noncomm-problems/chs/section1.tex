\subsection{Terminology and Examples}

\begin{thm}[type=Remark,num=1.19]
    $M$ noeth and artin iff $M$ has a comp series.
\end{thm} \

\begin{pf}
	Suppose $M$ is both noetherian and artinian.
	Then, $M$ has a maximal submodule $N_{1}$, by considering all
	proper submodules of $M$ under inclusion.
	Similarly, $N_{1}$ has such a maximal submodule $N_{2}$.

	Indeed, for any $k$, let $N_{k}$ be the maximal submodule of $N_{k-1}$.
	Since any submodule if $M$ is noetherian, this is well-defined.
	Thus, we have a chain of submodules:
	\begin{equation*}
	    \cdots < N_{3} < N_{2} < N_{1} < M
	\end{equation*}
	Since $M$ is artinian, there is some $k$ such that $N_{k}=N_{k+1}=\dots$.
	In other words, the chain terminates:
	\begin{equation*}
		\set{0}=N_{k}<N_{k-1}<\dots<N_{2}<N_{1}<M
	\end{equation*}
	Since each $N_{k}$ is maximal in $N_{k-1}$ by construction, then
	$N_{k}/N_{k-1}$ is simple for all $k$.
	Thus, $M$ has a composition series as needed.
\end{pf} \

\newpage
\begin{thm}[type=Remark,num=1.20]
    $N$ submod of $M$.
	then $M$ noeth iff $N,M/N$ noeth
\end{thm} \

\begin{pf}
	First, notice that if $N$ and $A/N$ are f.g., then so is $A$.
	Indeed, suppose:
	\begin{equation*}
	    N=\ang{x_{1},\dots,x_{n}} \qquad
		A/N=\ang{\bar{y}_{1},\dots,\bar{y}_{m}}
	\end{equation*}
	Let $a\in A$. Then:
	\begin{equation*}
	    \bar{a} \ = \ \sum_{i=1}^{m}s_{i}\bar{y}_{i} \ = \
		\bar{\sum_{i=1}^{m}s_{i}y_{i}} \ \implies \
		a - \sum_{i=1}^{m}s_{i}y_{i} \ = \ \sum_{j=1}^{n}r_{j}x_{j}
	\end{equation*}
	Thus, we have that $A=\ang{x_{1},\dots,x_{n},y_{1},\dots,y_{m}}$.

	Now, suppose $N$ and $M/N$ are noetherian.
	Let $A\leq M$ be any submodule.
	By the Second Isomorphism Theorem, we have:
	\begin{equation*}
		\frac{A}{A\cap N} \simeq \frac{A+N}{N} \leq \frac{M}{N}
	\end{equation*}
	Since $A\cap N \leq N$ is f.g. and $A/(A\cap N)\leq M/N$ is f.g.,
	by the above, it follows that $A$ is f.g., and thus $M$ is noetherian.

	Next, suppose $M$ is noetherian.
	Then $N$ is clearly noetherian, so let $A$ be any submodule of $M/N$.
	By the Fourth(*) Isomorphism Theorem, $A\simeq L/N$ for some submodule
	$N\leq L\leq M$.
	
	But since $M$ noetherian, $N$ and $L$ are finitely generated, so
	it follows that $A$ is finitely generated.
\end{pf} \

\begin{prop}[num=1.21(b)]
    $R$ right noeth, $M_{R}$ f.g.. then $M_{R}$ noeth.
\end{prop} \

\begin{pf}
    Suppose $M=\ang{m_{1},\dots,m_{n}}$ for some $n$. Then:
	\begin{equation*}
	    m \ = \ \sum_{i=1}^{n}r_{i}m_{i}
	\end{equation*}
	for all $m \in M$.

	Consider $R^{n}=R\times R\times \cdots \times R$.
	We have a map $\vphi:R^{n}\gto M$ given by:
	\begin{equation*}
	    \vphi(r_{1},\dots,r_{n}) \ = \ \sum_{i=1}^{n}r_{i}m_{i}
	\end{equation*}
	Clearly, $\vphi$ is surjective.
	Furthermore, since $R$ is noetherian, $R^{n}$ is as well.
	Therefore, by the First Isomorphism Theorem, we have that
	\begin{equation*}
		\frac{R^{n}}{\ker(\vphi)} \ \simeq \ M
	\end{equation*}
	is noetherian as needed.
\end{pf} \

\newpage
\begin{exr}[num=1.7a,source=Primary Source Material]
	sps $R=I_{1}\oplus\cdots\oplus I_{n}$ a dirsum of left ideals.
	show that $I_{k}=Re_{k}$ for $e_{k}$ idempotent elems of $R$ satisfying
	$e_{1}+ \dots + e_{n}=1$ and $e_{i}e_{j}=0$ for $i\neq j$.
\end{exr} \

\begin{pf}
	Since we can write $R=I_{1}\oplus \cdots \oplus I_{n}$, then for all
	$r\in R$, there exists a unique linear combination such that:
	\begin{equation*}
	    r \ = \ \sum_{k=1}^{n}r_{k} \qquad r_{k}\in I_{k}
	\end{equation*}
	In particular, there is a unique linear combination to write
	$1=e_{1}+\dots +e_{n}$.
	Now, note that each $r_{k}\in I_{k}$ can be written as:
	\begin{equation*}
	    r_{k} \ = \ \sum_{j=1}^{n}r_{k}\delta_{jk}
	\end{equation*}
	where $\delta_{jk}$ is the Kronecker delta.
	In particular, this holds for each $e_{k}$; it immediately follows that
	they are idempotent.
	We also see that for any $r_{k}\in I_{k}$:
	\begin{equation*}
	    r_{k} \ = \ r_{k}1 \ = \
		\left(\sum_{j=1}^{n}r_{k}\delta_{jk}\right)
		\left(\sum_{k=1}^{n}e_{k}\right)
		\ = \ \sum_{j=1}^{n}r_{k}\delta_{jk}e_{k}
		\ = \ r_{k}e_{k}
	\end{equation*}
	So $I_{k}=Re_{k}$ as needed.
	Finally, we have that $e_{j}e_{k}=0$ for $j\neq k$, since:
	\begin{equation*}
		e_{j}e_{k} \ = \
		\left(\sum_{i=1}^{n}e_{j}\delta_{ij}\right)
		\left(\sum_{i=1}^{n}e_{k}\delta_{ik}\right) \ = \
		\sum_{i=1}^{n}e_{j}\delta_{ij}e_{k}\delta_{ik} \ = \ 0
	\end{equation*}
	since $j\neq k$.
\end{pf}

\newpage
\begin{exr}[num=1.7b,source=Primary Source Material]
    show $I_{k}$ is a ring w identity $e_{k}$,
	and $R\simeq I_{1}\times \cdots \times I_{n}$.
\end{exr} \

\begin{pf}
	From part a), we saw that $r_{k}=r_{k}e_{k}$ for all $r_{k}\in I_{k}$.
	The proof that $r_{k}=e_{k}r_{k}$ is symmetric, so it remains to show that
	$R\simeq I_{1}\times \cdots \times I_{n}$.

	Indeed, suppose $S$ is any ring with $\vphi_{k}:S\gto I_{k}$
	ring homomorphisms. Define:
	\begin{equation*}
	    \vphi:S\gto I_{1}\oplus\cdots\oplus I_{n} \qquad
		\vphi(s) \ := \ \vphi_{1}(s) + \dots + \vphi_{n}(s)
	\end{equation*}
	We have $\vphi(1)=1$ since necessarily $\vphi_{k}(1)=e_{k}$;
	clearly $\vphi(a+b)=\vphi(a)+\vphi(b)$. Now:
	\begin{equation*}
	    \vphi_{i}(a)\vphi_{j}(b) \ = \
		\left(\sum_{k=1}^{n}\vphi_{i}(a)\delta_{ik}\right)
		\left(\sum_{k=1}^{n}\vphi_{j}(b)\delta_{jk}\right)
		\ = \ \vphi_{i}(a)\vphi_{j}(b)\delta_{ij}
	\end{equation*}
	where the Kronecker delta commutes, since it is either 0 or 1.
	This implies:
	\begin{align*}
		\vphi(ab) \ = \ \sum_{k=1}^{n}\vphi_{k}(ab)
		\ = \ \sum_{k=1}^{n}\vphi_{k}(a)\vphi_{k}(b)
		& \ = \ \sum_{i=1}^{n}\sum_{j=1}^{n}\vphi_{i}(a)\vphi_{j}(b) \\
		& \ = \ \left(\sum_{i=1}^{n}\vphi_{i}(a)\right)
		\left(\sum_{j=1}^{n}\vphi_{j}(b)\right) \\
		& \ = \ \vphi(a)\vphi(b)
	\end{align*}
	Therefore, $\vphi$ is indeed a ring homomorphism.
	Define $\pi_{k}:I_{1}\oplus\cdots\oplus I_{n}\gto I_{k}$ as
	$\pi_{k}(r) \ = \ r_{k}$, where $r = \sum_{k=1}^{n}r_{k}$.
	This is well-defined by the uniqueness of the decomposition of $r$,
	and is clearly a ring homomorphism.

	Finally, notice $\vphi_{k}=\pi_{k}\circ\vphi$ for all $k$.
	Thus, by the universal property of products in the category $\msf{Ring}$,
	we are done.
\end{pf} \

\begin{exr}[num=1.7c,source=Primary Source Material]
    show if $I_{k}$ are two-sided ideals, then $e_{k}\in Z(R)$.
\end{exr} \

\begin{pf}
	First, recall that if $j\neq k$, then $I_{j}\cap I_{k}=\set{0}$ must be
	trivial by uniqueness of decomposition.
	Then, it simply suffices to note that for any $r\in R$:
	\begin{equation*}
	    re_{k} \ = \ r_{k}e_{k} \ = \ r_{k} \ = \ e_{k}r_{k} \ = \ e_{k}r
	\end{equation*}
	by the above fact and work shown in previous parts.
\end{pf} \

\begin{exr}[num=1.8,source=Primary Source Material]
    sps $R=I\oplus J$ for ideals $I,J$.
	show every ideal of $R$ is of the form $I'\oplus J'$, where $I',J'$ ideals
	in $I,J$ resp.
\end{exr} \

\begin{pf}
	Suppose $S\subseteq R$ is an ideal.
	By Exercise 1.7, there exists $e_{i}\in I$ and $e_{j}\in J$ such that
	$I=Re_{i},J=Re_{j},e_{i}+e_{j}=1$, and $e_{i},e_{j}$ is the identity of
	$I,J$ resp. as a ring.
	Define $I':=Se_{i}$ and $J':=Se_{j}$.
	We want to show that $S=I'\oplus J'$.

	Clearly $I',J'\subseteq S$ since $S$ is an ideal, thus $I'+J'\subseteq S$.
	On the other hand, for any $s \in S$, let:
	\begin{equation*}
	    s_{i}\in I', s_{j}\in J' \qquad
		s_{i} \ := \ se_{i} \qquad s_{j} \ := \ se_{j}
	\end{equation*}
	Then $s_{i}+s_{j} = se_{i}+se_{j} = s(e_{i}+e_{j}) = s$ , so
	we see that $S=I'+J'$.
	Finally, since $s_{i}\in I$ and $s_{j}\in J$,
	then $s = s_{i}+s_{j}$ is the unique decomposition of $s$,
	and it follows that $S=I'\oplus J'$ as needed.
\end{pf} \

\newpage
\begin{exr}[num=1.12a,source=Primary Source Material]
    $_{R}M$ is called \textit{Hopfian} if every surj endo is inj.
	show every noeth $_{R}M$ is hopfian.
\end{exr} \

\begin{pf}
	Let $\vphi$ be a surjective endomorphism of $M$.
	Recall that $\ker(\vphi)$ is a submodule of $M$.
	Consider $\vphi^{(2)}:=\vphi\circ\vphi$.
	Since $\vphi$ is surjective, then there exists $m \in M$ such that
	$\vphi(m)\in\ker(\vphi)$.
	Thus, we have that $\ker(\vphi)\leq\ker(\vphi^{(2)})$.

	In general, we can consider $\ker(\vphi^{(k)})$ in the same way.
	Since $\vphi$ is surjective, we have that $im(\vphi^{(k)})=M$,
	so we can construct the following chain:
	\begin{equation*}
	    \ker(\vphi) \leq \ker(\vphi^{(2)}) \leq \ker(\vphi^{(3)}) \leq \cdots
	\end{equation*}
	Since $M$ is noetherian, this chain must terminate;
	if $\ker(\vphi)$ is non-trivial, then the above construction holds.
	Therefore $\ker(\vphi)=\set{0}$, or in other words, $\vphi$ is injective.
\end{pf} \

\begin{exr}[num=1.12b,source=Primary Source Material]
    prove $_{R}R$ is hopfian iff dedekind-finite
\end{exr} \

\begin{pf}[source=Primary Source Material]
	First, suppose $_{R}R$ is Hopfian and fix $x\in R$ such that $xy=1$
	for some $y\in R$.
	It suffices to show that $x$ is left-invertible.

	For any $r\in R$, let $\vphi_{r}(s)=rs$. This is evidently an endomorphism.
	Consider the map $\vphi_{x}$; note that this is surjective, as it has
	a right-inverse:
	\begin{equation*}
		(\vphi_{x}\circ\vphi_{y})(s) \ = \ (xy)s \ = \ s
		\ \implies \
		\vphi_{x}\circ\vphi_{y} \ = \ \vphi_{1} \ = \ \trm{id}
	\end{equation*}
	Since $R$ is Hopfian, then $\vphi_{x}$ is injective.
	By uniqueness of inverses, we have:
	\begin{equation*}
		(\vphi_{y}\circ\vphi_{x})(s) \ = \ (yx)s \ = \ s
	\end{equation*}
	Since this is true of all $s \in R$, then $yx=1$ as needed.

	Now, suppose $R$ is Dedekind-finite.
	Let $\vphi:R\gto R$ be a surjective endomorphism. Notice:
	\begin{equation*}
	    \vphi(r) \ = \ \vphi(1r) \ = \ \vphi(1)r
	\end{equation*}
	Thus $\vphi=\vphi_{\vphi(1)}$ is given by scaling.
	Fix $\vphi(1)=r$.
	Since $\vphi$ is surjective, then there exists some $s\in R$ such that:
	\begin{equation*}
	    \vphi(s)=rs = 1
	\end{equation*}
	Similarly to above, we therefore have that:
	\begin{equation*}
	    \vphi\circ\vphi_{s} \ = \ \vphi_{r}\circ\vphi_{s}
		\ = \ \vphi_{1} \ = \ \trm{id}
	\end{equation*}
	Since $R$ is Dedekind-finite, then:
	\begin{equation*}
	    \vphi_{s}\circ\vphi \ = \ \vphi_{s}\circ\vphi_{r}
		\ = \ \vphi_{1} \ = \ \trm{id}
	\end{equation*}
	It follows that $\vphi$ is injective, and so $R$ is indeed Hopfian.
\end{pf} \

\begin{exr}[num=1.13a,source=Primary Source Material]
	sps $k$ a field, $R$ a $k$-alg.
	an elem $r\in R$ is \textbf{algebraic} over $k$ if $\ex p\in k[x]$
	non-zero s.t. $p(r)=0$.
	if every elem is algebraic, $R$ is called an \textbf{algebraic}
	$\bm{k}$\textbf{-algebra}.

	show that if $R$ fd as a $k$-vec space, $R$ is algebraic.
	is the converse true?
\end{exr} \

\begin{pf}
	Suppose $R$ is finite-dimensional as a $k$-vector space, and fix $r\in R$.
	Let $s \in R$ be any element, and consider the endomorphism $\vphi_{r}$
	given as $\vphi_{r}(s)=rs$.

	Since $R$ is a $k$-algebra and $\vphi_{r}$ is an endomorphism, we must
	have that $\vphi_{r}$ is given by some square matrix $A$.
	Let $p\in k[x]$ be the characteristic polynomial of $A$.
	We claim that $p(r)=0$.

	Write the characteristic polynomial $p(x)$ of $A$ as:
	\begin{equation*}
		p(x) \ := \ a_{n}x^{n}+ \dots + a_{1}x + a_{0}
	\end{equation*}
	where each $a_{i}\in k$. Then:
	\begin{align*}
	    p(r) \ = \ p(r)\cdot1 \ = \ (a_{n}r^{n}+\dots+a_{1}r+a_{0})\cdot1
		& \ = \ a_{n}r^{n}\cdot1+\dots+a_{1}r\cdot1+a_{0}\cdot1 \\
		& \ = \ a_{n}A^{n}\cdot1+\dots+a_{1}A\cdot1+a_{0}I_{n}\cdot1 \\
		& \ = \ a_{n}A^{n}+\dots+a_{1}A+a_{0}I_{n} \\
		& \ = \ 0
	\end{align*}
	by the Cayley-Hamilton theorem.
	Therefore we have that $p(r)=0$ as needed.

	NOTE: idk for converse
\end{pf} \

\begin{exr}[num=1.13b,source=Primary Source Material]
	henceforth, sps $R$ an alg $k$-alg.
	show $R$ dedekind-finite.
\end{exr} \

\begin{pf}
	Suppose $r,s \in R$ such that $rs =1$.
	Since $R$ is algebraic, there exists $p\in k[x]$ such that $p(r)=0$. Write:
	\begin{equation*}
	    p(x) \ := \ a_{n}x^{n}+\dots +a_{1}x+a_{0}
	\end{equation*}
	Suppose $m$ is the lowest index such that $a_{m}\neq 0$. Then:
	\begin{equation*}
		p(r)\cdot s^{m} = (a_{n}r^{n}+\dots +a_{m}r^{m})s^{m}
		= a_{n}r^{n-m}+\dots +a_{m+1}r+a_{m}=0
	\end{equation*}
	Thus, suppose WLOG that $a_{0}\neq0$. Then:
	\begin{align*}
	    a_{n}r^{n}+\dots +a_{1}r+a_{0} = 0
		& \ \implies \ a_{n}r^{n}+\dots +a_{1}r = -a_{0} \\
		& \ \implies \ (a_{n}r^{n-1}+\dots +a_{2}r+a_{1})r=-a_{0} \\
		& \ \implies \ -(a_{0}^{-1})
		\left(a_{n}r^{n-1}+\dots +a_{2}r+a_{1}\right)r = 1
	\end{align*}
	since $a_{0}\in k$, and $k$ is a field.
	Thus, $r$ has a left-inverse, so by uniqueness of inverses, we have
	$sr=1$.
\end{pf} \

\begin{exr}[num=1.13c,source=Primary Source Material]
	show every left 0-div is a right 0-div.
\end{exr} \

\begin{pf}
    TODO
\end{pf} \

\begin{exr}[num=1.13d,source=Primary Source Material]
	sps $0\neq r\in R$. show $r$ a 0-div iff $r$ not a unit.
\end{exr} \

\begin{pf}
    TODO
\end{pf} \

\begin{exr}[num=1.13e,source=Primary Source Material]
	sps $B$ a subalg of $R$, $b\in B$.
	show that if $b$ a unit in $R$, $b^{-1}\in B$.
\end{exr} \

\begin{pf}
    TODO
\end{pf} \



