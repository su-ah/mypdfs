\subsection{Semisimplicity}

\lecdate{Week 2}

throughout this chapter (and possibly the book), we write $_{R}M$ or $M_{R}$
to indicate whether $M$ is a left or right $R$-mod.
if $M$ is an $(R,S)$-bimod, we may write $M$ as $_{R}M_{S}$.

\begin{defn}
    let $R$ be a ring, $M$ a (left) $R$-mod.

	$M$ is called \textbf{simple} (or \textbf{irreducible}) if
	$M\neq0$ and has no $R$-submods other than $(0)$ and $M$.

	$M$ is called \textbf{semisimple} (or \textbf{completely reducible}) if
	every $R$-submod is an $R$-mod direct summand of $M$; that is, for any
	submod $N$, there exists a complement $P$ s.t. $M=N\oplus P$.
\end{defn}

note the zero module is semisimple, but not simple.
furthermore, evident from the following is:

\begin{crll}
    any submod (resp. qmod)(?) of a semisimp $R$-mod is semisimp.
\end{crll}

we also clearly have that simple $\implies$ semisimp.
we examine the relationship between the two more closely,
but first prove an intermediate fact.

\begin{lm}
    any semisimp $_{R}M\neq0$ contains a simple submod.
\end{lm}

\begin{pf}[source=Primary Source Material]
	fix nonzero $m \in M$. by 2.2, it suffices to consider when $M=Rm$.

	by zorns, there exists submod $N$ of $M$ maximal wrt the property
	$m\notin N$.
	take a (necessarily nonzero) submod $N'$ s.t. $M=N\oplus N'$.
	we finish by showing $N'$ simple.

	indeed, if $N''\neq0$ a submod of $N'$, $m \in N\oplus N''$ by maximality,
	so $N\oplus N''=M$.
\end{pf}

as a result, we get two other characterizations of semisimp mods, which can
be useful as alt defns.

\begin{thm}
    tfae: \vspace{-0.2in}
	\begin{enumerate}
	    \item $_{R}M$ semisimp.
		\item $_{R}M$ a dirsum of a family of simple submods.
		\item $_{R}M$ a sum of a family of simple submods.
	\end{enumerate}
\end{thm}
note that by convention, we take the empty (dir)sum to be the zero module.

\begin{pf}[source=Primary Source Material]
	$(1\implies3)$
	let $M_{1}$ be the sum of all simple submods, and
	write $M=M_{1}\oplus M_{2}$.
	if $M_{2}\neq0$, by 2.3, $M_{2}$ contains a simple submod, a contradiction.

	$(3\implies1)$
	write $M=\sum_{i\in I}M_{i}$ and let $N\subseteq M$ be a given submod.
	consider subsets $J\subseteq I$ with the following properties:
	\vspace{-0.2in}
	\begin{itemize}
	    \item $\sum_{j\in J}M_{j}$ is a \textit{direct} sum.
		\item $N\cap\sum_{j}M_{j} = 0$.
	\end{itemize}
	an easy check shows zorns applies to the family of all such $J$'s under
	set inclusion. pick a maximal $J$ and let:
	\begin{equation*}
		M' \ := \ N+\sum_{j}M_{j} \ = \ N \oplus \bigoplus_{j}M_{j}
	\end{equation*}
	we finish by showing $M'=M$, for which it suffices to show
	$M_{i}\subseteq M' \ \forall \, i$.
	if some $M_{i}\nsubseteq M'$, simplicity of $M_{i}$ implies
	$M'\cap M_{i}=0$, so:
	\begin{equation*}
	    M' + M_{i} = N \oplus \left( \bigoplus_{j}M_{j} \right) \oplus M_{i}
	\end{equation*}
	but this contradicts maximality of $J$.

	$(3\implies2)$ follows from the above on $N=0$, and $(2\implies3)$
	is a tautology.
\end{pf}

we can now define a (left) semisimp ring.

\begin{defn}
    a ring $R$ is (left) \textbf{semisimple} if $_{R}R$ is semisimp.
\end{defn} \

\begin{thm}
    tfae: \vspace{-0.2in}
	\begin{enumerate}
	    \item all short exact seqs of left $R$-mods split.
		\item all left $R$-mods are semisimp.
		\item all finitegen left $R$-mods are semisimp.
		\item all cyclic left $R$-mods are semisimp.
		\item $_{R}R$ is semisimp.
	\end{enumerate}
\end{thm}

\begin{pf}[source=Primary Source Material]
	note that clearly(?):
	\begin{equation*}
		(1) \iff (2) \implies (3) \implies (4) \implies (5)
	\end{equation*}
	thus it suffices to prove $(5)\implies(2)$.

	fix $_{R}M$ where $R$ satisfies $(5)$.
	by 2.2, $(5)$ implies any cyclic submod $Rm$ of $M$ is semisimp.
	since $M=\sum_{m \in M}Rm$, it follows from 2.4(3) that $M$ is semisimp.
\end{pf}

let $R$ be left semisimp.
using 2.6(5), we have $R=\bigoplus_{i}U_{i}$ for simple left $R$-mods $U_{i}$,
which are just minimal left ideals in $R$.
since $1\in R$, this dirsum is in fact \textit{finite}.
thus, we can write a composition series for $_{R}R$ w comp factors
$\set{_{R}U_{i}}$. by 1.12(3), $_{R}R$ satisfies ACC and DCC for $R$-submods.

\begin{crll}
    a left semisimp ring is both left noeth and left artin.
\end{crll}

the characterization 2.6(1) gives us a homological interpretation of
left semisimplicity; this is done using the notion of a
\textit{projective module}.

\begin{defn}
    $_{R}P$ is called $\mbf{R}$\textbf{-projective} (or \textbf{projective})
	if for any surj $R$-hom $f:{_{R}A}\gto{_{R}B}$ and any $R$-hom
	$g:{_{R}P}\gto{_{R}B}$ there exists an $R$-hom $h:{_{R}P}\gto{_{R}A}$
	s.t. $f\circ h=g$.
\end{defn}

the following propn from homalg offers two alt characterizations of proj mods.

\begin{thm}
    a (left) $R$-mod $P$ is proj iff $P$ is (iso to) a dirsum of a left free
	$R$-mod, iff any surj $R$-hom from any left $R$-mod onto $P$ splits.
\end{thm} \

\begin{pf}
    exercise :)
\end{pf}

we can now state the homological characterization of the class of (left)
semisimp rings.

\begin{thm}
    tfae: \vspace{-0.2in}
	\begin{enumerate}
	    \item $R$ is left semisimp.
		\item all left $R$-mods are proj.
		\item all finitegen left $R$-mods are proj.
		\item all cyclic left $R$-mods are proj.
	\end{enumerate}
\end{thm}

\begin{pf}[source=Primary Source Material]
    we have:
	\begin{equation*}
		(1) \iff (2) \implies (3) \implies (4)
	\end{equation*}
	we show $(4)\implies(1)$ by verifying $_{R}R$ semisimp.

	consider any left ideal $U\subseteq R$.
	by our assumption, the left $R$-mod $R/U$ is proj, so the following
	s.e.seq splits:
	\begin{equation*}
	    0 \gto U \gto R \gto R/U \gto 0
	\end{equation*}
	this implies $U$ an $R$-mod direct summand of $_{R}R$ as needed.
\end{pf}

there is also the injective mod, directly dual to the proj mod.
literally u can figure out the defn, its a dual.
the 2nd part of 2.9 (?) admits the following dual:
$I$ is inj iff any inj $R$-hom to any left $R$-mod splits.
we can thus deduce an analogous characterization:
\begin{enumerate}
    \item $R$ is left semisimp.
	\item all left $R$-mods are inj.
	\item all finitegen left $R$-mods are inj.
	\item all cyclic left $R$-mods are inj.
\end{enumerate}
note that despite its dual nature, the characterization $(4)\implies(1)$ is
much harder, and is due to B. Osofsky.

there are also many more characterizations that have not been included here.
Vol I of Rowen pp. 496 has an exhaustive(x) list of 23 characterizations.


