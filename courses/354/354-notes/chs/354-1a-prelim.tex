\subsection{Preliminaries}
\lecdate{Lec 1 - Sep 3 (Week 1)}

i think sth abt the discriminant?

cardanos soln to simplified cubics $x^{3}+cx+d=0$ (circa 1535):

\begin{equation*}
    x=\sqrt[3]{\frac{d}{2}+\sqrt{\frac{d^{2}}{4}+\frac{c^{3}}{27}}}
    + \sqrt[3]{\frac{d}{2}-\sqrt{\frac{d^{2}}{4}+\frac{c^{3}}{27}}}
\end{equation*}
tartaglia (circa 1539) said: let $x=t-\frac{b}{3a}$ in $ax^{3}+bx^{2}+cx+d=0$.
this yields a cubic $At^{3}+Bt+C=0$, letting us use cardanos soln.

but this means each cubic has exactly one real solution - that doesnt make sense!
for instance, the roots of $x^{3}-3x=0$ are $x=0,\pm\sqrt{3}$.

lets try cardanos formula for $x^{3}-3x=0$:
given $c=-3,d=0$, we have:
\begin{align*}
    x &= \sqrt[3]{0+\sqrt{0+\frac{(-3)^{3}}{27}}}
    +\sqrt[3]{0-\sqrt{0+\frac{(-3)^{3}}{27}}} \\
      &= \sqrt[3]{\sqrt{-1}}+\sqrt[3]{-\sqrt{-1}} \\
      &= \sqrt[3]{\sqrt{-1}} - \sqrt[3]{\sqrt{-1}} \\
      &= \sqrt[3]{i} - \sqrt[3]{i}
\end{align*}
note that $i$ has 3 cubic roots - $x^{3}-i=0$ has 3 complex roots, given
by $-i, \sqrt{\frac{3}{2}}+\frac{1}{2}i$, and $-\sqrt{\frac{3}{2}+\frac{1}{2}i}$.

i lwk not paying attn lol

complex analysis is \textit{NOT} a generalization of real analysis!
for instance, consider:
\begin{equation*}
    f(x)=
    \begin{cases}
        x^{2} & x\geq0 \\ -x^{2} & x<0
    \end{cases} \qquad
    f'(x)=
    \begin{cases}
        2x & x\geq0 \\ -2x & x<0
    \end{cases}
\end{equation*}
note $f$ is differentiable at $x=0$ - this can be seen using the limit defn.
however, $f''(x)$ is clearly not defined at $x=0$.

this doesnt happen in complex functions - differentiable functions have
differentiable derivatives!

recall in real analysis, $f:\bR\sto\bR$ is diff'ble at $x_{0}$ if there is $L$
such that:
\begin{equation*}
    \lim_{x\sto x_{0}}\frac{\abs{f(x)-f(x_{0})}}{\abs{x-x_{0}}}=L \qquad
    0<\abs{x-x_{0}}<\delta \implies
    \abs{\frac{f(x)-f(x_{0})}{x-x_{0}}-L}<\ep
\end{equation*}
a complex $F:\bC\sto\bC$ is diff'ble at $z=z_{0}$ if there is $L$ such that:
\begin{equation*}
    0<\abs{z-z_{0}}<\delta \implies \abs{\frac{F(z)-F(z_{0})}{z-z_{0}}-L}<\ep
\end{equation*}
whats different? well... in $\bC$, the ``absolute value" is in fact a norm, and
that means we approach within a \textit{ball}, rather than only from two
directions.
as a result, this is a \textit{strictly stronger condition} than real
differentiability.

wasnt looking.
notice $a:=\Re z = \frac{z+\bar{z}}{2}$ and $b:=\Im z = \frac{z-\bar{z}}{2i}$.

\begin{xmp}[source=Primary Source Material]
    write equation of a line $ax+by=c$ with $a^{2}+b^{2}\neq0$ in terms
    of $z$ and $\bar{z}$:
    \begin{equation*}
        a \left( \frac{z+\bar{z}}{2} \right) +
        b \left( \frac{z-\bar{z}}{2i} \right) = c \ \implies \
        \left( \frac{a}{2}+\frac{b}{2i} \right)z +
        \left( \frac{a}{2}-\frac{b}{2i} \right)\bar{z} = c
    \end{equation*}
\end{xmp}

still not looking

consider:
\begin{equation*}
    f(x)=
    \begin{cases}
        e^{-1/x^{2}} & x\neq0 \\ 0 & x = 0
    \end{cases}
\end{equation*}
it can be shown that $f^{(n)}(0)=0$ for all $n$, so the taylor series at $x=0$
is $0$. yet the taylor series of $f$ does not approach $f$ on a nbhd of $0$.

thus, a real function can be $C^{\infty}$ but still not analytic
(ie no taylor series exact expansion), but this is not the case for complex
functions - differentiable functions are always analytic.

some properties:
\begin{equation*}
    \frac{1}{z}\cdot\frac{\bar{z}}{\bar{z}} = \frac{\bar{z}}{\abs{z}^{2}} \qquad
    \abs{z}=\abs{\bar{z}} \qquad -\abs{z}\leq\Re z\leq\abs{z} \qquad
    -\abs{z}\leq\Im z\leq\abs{z}
\end{equation*}



