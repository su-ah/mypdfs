\subsection{Compactness}

\begin{prop}
    if $x_{i} \sto x$ in $X$, then $\set{x_{i}}\cup\set{x}$ is cpt
\end{prop}

\begin{pf}[source=Primary Source Material]
    $x$ and tail of $x_{i}$ in some $U$,
    then take one set for remaining pts of seq
\end{pf}

\begin{prop}[type=Theorem]
    cts img of cpt is cpt
\end{prop}

\begin{pf}[source=Primary Source Material]
    preimg of cover covers $X$, img of finite subcover covers $f(X)$
\end{pf} \

\begin{lm}
    $X$ hausdorff, $A,B\subseteq X$ disj cpt, then
    $\exists \, $ disj open $U,V\subseteq X$ s.t. $A\subseteq U,B\subseteq V$
\end{lm}

\begin{pf}[source=Primary Source Material]
    first, sps $B=\set{q}$.
    for each $p\in A$, there exist disj open $U_{p},V_{p}$ s.t.
    $p\in U_{p},q\in V_{p}$, so $\set{U_{p}}$ covers $A$.
    set:
    \begin{equation*}
        U_{0}=\bigcup_{i}^{k}U_{p_{i}} \qquad
        V_{0}=\bigcap_{i}^{k}V_{p_{i}}
    \end{equation*}
    then $U_{0},V_{0}$ disj open with $A\subseteq U_{0},B\subseteq V_{0}$.

    for general $B$, by the above, there exist disj open $U_{q},V_{q}$
    for all $q$ s.t. $A\subseteq U_{q}$ and $q\in V_{q}$.
    by cpt, set $U=\bigcap_{i}^{m}U_{q_{i}},V=\bigcup_{i}^{m}V_{q_{i}}$,
    and we are done.
\end{pf}

\begin{lm}[title=Tube Lemma]
    $X$ topsp, $Y$ cpt, $x\in X, U\subseteq X\times Y$ open s.t.
    $\set{x}\times Y\subseteq U$.
    then $\exists$ nbhd $x\in V$ s.t. $V\times Y\subseteq U$.
\end{lm}

\begin{pf}[source=Primary Source Material]
    $\forall \, y\in Y \exists \, V_{0}\times W\subseteq X\times Y$ open s.t.
    $(x,y)\in V_{0}\times W\subseteq U$.
    the slice $\set{x}\times Y\simeq Y$, so finitely many $V_{i}\times W_{i}$
    cover it.
    set $V=\bigcap_{i}^{k}V_{i_{k}}$; then $V\times Y\subseteq U$.
\end{pf}

\begin{prop} \vspace{-0.275in}
    \begin{itemize}
        \item cpt subset of hausdorff is closed
        \item cpt subset of metric is bdd
        \item finite prod of cpt is cpt
        \item quot of cpt is cpt
    \end{itemize}
\end{prop}

\begin{pf}[source=Primary Source Material]
    for 1, if $A$ cpt subset of hausdorff, any $p\in A^{c}$ has a nbhd by 16.3.
    for 2, take a cover of open balls centered at each pt.
    for 3, apply the tube lemma to each slice in $X$.
    4 follows from 16.2.
\end{pf}
3 is in fact true in the more general setting of inf prods;
in its general form, it is known as Tychonoff's Theorem.

\begin{defn}
    $X$ is \textbf{limit point compact} if every inf subset has a lim pt in $X$.
\end{defn} \

\begin{lm}
    cpt implies lim pt cpt
\end{lm}

\begin{pf}[source=Primary Source Material]
    sps $S\subseteq X$ inf.
    if $S$ has no lim pt in $X$, then every pt has a nbhd $U$ s.t. $U\cap S$ is
    either empty or singleton.
    $X$ cpt, so finitely many such nbhds cover.
    but then $S$ necessarily finite, contradiction.
\end{pf}
converse is not generally true; see exr ??

\begin{prop}
    in 1st ctbl hausdorff, lim pt cpt implies seq cpt
\end{prop}

\begin{pf}[source=Primary Source Material]
    let $(p_{n})$ be a seq in $X$; trivially sps it takes inf many values.
    by hypothesis, the set $\set{p_{n}}$ has a lim pt $p$.
    if infinitely many $p_{n}=p$, we are done;
    thus, we can discard finitely many pts at the start s.t. $p_{n}\neq p$ for
    all $n$.
    since $X$ 1st ctbl, there is a nested nbhd basis $(B_{n})$ at $p$.
    since $p$ lim pt, $\exists \,p_{n_{1}}\in B_{1}$.
    by ind, we can choose $p_{n_{k+1}}\in B_{k+1}$.
    then, this is a subseq cvging to $p$.
\end{pf}

\begin{prop}
    for metric or 2nd ctbl spaces, seq cpt implies cpt
\end{prop}

\begin{pf}[source=Primary Source Material]
    sps $X$ 2nd ctbl.
    then $X$ lindelof, so open cvr has ctbl subcvr.
    sps no finite subcvrs;
    then $\forall \, i\exists \, q_{i}$ s.t. $q_{i} \notin\bigcup_{s}^{i}U_{s}$.
    then, $q_{i_{k}}\sto q\in U_{m}$ for some $m$.
    then a tail is necessarily in $U_{m}$, but this contradicts the construction.

    sps $X$ seq cpt metric space.
    we show $X$ 2nd ctbl by showing $X$ separable.
    first, open cvr of $\ep$-balls has finite subcvr $\forall \, \ep$.
    sps otw; fix $q_{1}$. then $\exists \, q_{2}\in(B_{\ep}(q_{1}))^{c}$;
    taking a seq, replace by a cvg subseq $q_{n}\sto q\in M$.
    thus cauchy; a contradiction.

    for each $n$, take $F_{n}$ finite set of pts s.t. their $1/n$ balls cvr $X$.
    then $\bigcup F_{n}$ ctbl and clearly dense, so $X$ separable as needed.
\end{pf}
we summarize the above.

\begin{thm}
    cpt implies lim cpt, 1st ctbl hausdorff lim pt cpt implies seq cpt,
    metric or 2nd ctbl seq cpt implies cpt.

    [diagram]

    in particular, for metric or 2nd ctbl hausdorff, the three are equivalent.
\end{thm}
the next two (and bolzano-weierstrass) are consequences of the above.

\begin{thm}
    subs of $\bR^{n}$ complete iff closed (in std metric)
\end{thm}

\begin{thm}
    cpt metric implies complete
\end{thm}
see exr ?? for pfs.

lastly, we have a simple but useful result. \
\begin{lm}[title=Closed Map Lemma]
    sps $f:X\sto Y$ cts with $X$ cpt, $Y$ hausdorff. \vspace{-0.275in}
    \begin{itemize}[parsep=3pt]
        \item $f$ closed
        \item $f$ surj implies qmap
        \item $f$ inj implies embedding
        \item $f$ bij implies homeo
    \end{itemize}
\end{lm}

\begin{pf}[source=Primary Source Material]
    if $A\subseteq X$ closed, then cpt, so $f(A)$ cpt and thus closed.
    the other follow from ??.
\end{pf}

\begin{prop}
    sps $X$ cpt hausdorff, $q:X\sto Y$ qmap. tfae: \vspace{-0.2in}
    \begin{enumerate}[(\alph*),parsep=3pt]
        \item $Y$ hausdorff
        \item $q$ closed
        \item $R=\set{(x_{1},x_{2}):q(x_{1})=q(x_{2})}$ closed in $X^{2}$
    \end{enumerate}
\end{prop}

\begin{pf}[source=Primary Source Material]
    (a) imp (b) follows from closed map lemma, (a) imp (c) from prop 9.7.

    for (b) imp (a), every $y=q(x)$ for all $y\in Y$.
    $\set{x}$ closed so $\set{y}=q(\set{x})$ closed, so the fiber $q^{-1}(y)$
    closed and thus cpt. now, sps $y_{1}\neq y_{2}$.
    by 15.3, $q^{-1}(y_{i})$ have disj nbhds $U_{i}$.
    set $W_{i}=\set{y:q^{-1}(y)\subseteq U_{i}}$.
    then each $y_{i}\in W_{i}$, and $W_{i}$ disj bc $U_{i}$ disj.
    finally, note $W_{i}=Y\sm q(X\sm U_{i})$;
    $q$ closed implies $W_{i}$ open as needed.

    for (c) imp (a), given $y\in Y,x \notin q^{-1}(y)$, fix $x_{1}\in q^{-1}(y)$.
    $x_{1}$ exists since $q$ surj.
    $R$ closed and $(x_{1},x) \notin R$ so there exists nbhd $U_{1}\times U_{2}$
    of $(x_{1},x)$ in $X^{2}$ disj from $R$.
    then $U_{2}$ disj from $q^{-1}(y)$, so $q^{-1}(y)$ closed and thus cpt.
    fix $y_{1}\neq y_{2}\in Y$; we define $W_{i}$ as above.
    since $q$ qmap, $W_{i}$ open iff $X\sm q^{-1}(W_{i})$ closed. note:
    \begin{equation*}
        X\sm q^{-1}(W_{i}) =
        \set{x:\exists \, x'\in x\sm U_{i} \trm{ s.t. } q(x)=q(x')}
        = \pi_{1}(R\cap(X\times(X\sm U_{i})))
    \end{equation*}
    by closed map lemma, $\pi_{1}$ closed, and the result follows.
\end{pf}

\newpage
\begin{xmp}[source=Primary Source Material]
    we mentioned before that $\bar{B^{n}}/S^{n-1}\simeq S^{n}$ by collapsing
    the bdry.
    to see this, consider $q:\bar{B^{n}}\sto S^{n}$ given by:
    \begin{equation*}
        q(x) \ = \ \left(2x\sqrt{1-\abs{x}^{2}},2\abs{x}^{2}-1\right)
    \end{equation*}
    this is surj cts and has the same identifications,
    so by closed map lemma we're done.
\end{xmp}

lastly, we use the lemma to improve the result of exercise ??.

\begin{defn}
    given mfld $M$, a coord ball $B\subseteq M$ is
    a \textbf{regular coordinate ball} if $\exists \, $ nbhd $B'$ of $\bar{B}$
    and homeo $\vphi:B'\sto B_{r'}(x)\subseteq\bR^{n}$ s.t.
    $B\mto B_{r}(x)$ and $\bar{B}\mto\bar{B}_{r}(x)$ for some
    $0<r<r',x\in\bR^{n}$.
\end{defn} \ %rly need 2 fix spacing

\begin{lm}
    given $M$ mfld, $B'$ coord ball, $\vphi:B'\sto B_{r'}(x)\subseteq\bR^{n}$
    homeo, $\vphi^{-1}(B_{r}(x))$ is reg coord ball if $0<r<r'$.
\end{lm}

\begin{pf}[source=Primary Source Material]
    clearly $\vphi\rvert_{B}:\vphi^{-1}(B_{r}(x))\sto B_{r}(x)$ homeo.
    we check $\bar{B}=\vphi^{-1}(\bar{B}_{r}(x))$; indeed,
    closed map lemma says $\vphi^{-1}\rvert_{\bar{B}_{r}(x)}$ closed as needed.
\end{pf}

\newpage
\begin{prop}
    every mfld has ctbl basis of reg coord balls.
\end{prop}

\begin{pf}[source=Primary Source Material]
    all $p\in M$ have euclidean nbhd; since $M$ 2nd ctbl,
    $\set{U_{i}}$ ctbly covers $M$.
    for each $U_{i}$, choose $\vphi_{i}:U_{i}\sto\hat{U}_{i}\subseteq\bR^{n}$.
    then for each $x\in\hat{U_{i}}$, there is some $r(x)>0$ s.t.
    $B_{r(x)}(x)\subseteq\hat{U}_{i}$.
    let $\cl{B}$ be the collection of all subsets of the form
    $\vphi^{-1}(B_{r}(x))$ where $x\in\hat{U}_{i}$ has rational coords and
    $0<r<r(x)$ is rational.
    by 15.17, each is a reg coord ball, and $\cl{B}$ is ctbl;
    checking that $\cl{B}$ is a basis is an exercise.
\end{pf}
there is also a version for mflds w bdry;
we can define a \textbf{regular coordinate half-ball} $B\subseteq M$ if there is
open $B'$ containing $\bar{B}$ and homeo from $B'$ to $B_{r'}(0)\cap\bb{H}^{n}$
which takes $B$ to $B_{r}(0)\cap\bb{H}^{n}$ and $\bar{B}$ to
$\bar{B}_{r}(0)\cap\bb{H}^{n}$ for some $0<r<r'$.

