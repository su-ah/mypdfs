\subsection{Disjoint Unions}
given a family of spaces, we want a canonical construction of a space which
contains each space as a subspace. recall the disjoint union $ \coprod_{\alpha
\in A}X_{a} $ given by all pairs $ (x, \alpha) $ with $ x \in X_{\alpha} $ and
$ \alpha \in A $. this induces canonical injections $ \iota_{\alpha}:X_{\alpha}
\sto \coprod_{\alpha}X_{\alpha} $, identifying each $ X_{\alpha} $ with the img
of the corresponding $ \iota_{\alpha} $.

\begin{defn}
    We define the \textbf{disjoint union topology} by declaring a subset to be
    open iff its intersection with each $ X_{\alpha} $ (as a subset of the union)
    is open in $ X_{\alpha} $.
\end{defn}

\begin{prop}[type=Theorem,title=Characteristic Property of Disjoint Unions]
    a map $ f:\coprod X_{\alpha}\sto Y $ is cts iff the restriction to each
    $ X_{\alpha} $ is cts. furthermore, the disjoint union topology is the unique
    topology up to homeo with this property.
\end{prop}
See Exercise ?? for proof.

\begin{prop}
    properties of disjoint union topo
    \begin{itemize}
        \item replace open w/ closed in the topo definition
        \item the canonical injections $ \iota_{\alpha} $ are open and closed
            embeddings
        \item if each $ X_{\alpha} $ hausdorff/first ctbl, so is the union
        \item if $ A $ ctbl and each $ X_{\alpha} $ 2nd ctbl, so is the union
    \end{itemize}
\end{prop}
See Exercise ?? for proof.

