\subsection{Hausdorff Spaces}

\begin{defn}
    A space $ X $ is \textbf{Hausdorff} if for any pair of distinct points $ p,q
    \in X $, there are neighbourhoods $ U, V $ of $ p, q $ respectively such that
    $ U \cap V = \eset $.
\end{defn}

\begin{prop}
    Let $ X $ be Hausdorff. Then:
    \begin{itemize}
        \item Every finite subset of $ X $ is closed.
        \item Limits of convergent sequences are unique.
    \end{itemize}
\end{prop}

\begin{pf}[source=Primary Source Material]
    Fix $ p_{0} $, and let $ p \neq p_{0} $. Then, there are disjoint
    neighbourhoods $ U, V $ of $ p, p_{0} $ respectively. It follows that $ U $
    is open in $ X \setminus \set{p_{0}} $, and so $ \set{p_{0}} $ is closed.
    The result then follows.

    To see that limits are unique, suppose a sequence has two limits $ p, p' $.
    Then there exist disjoint neighbourhoods $ U, U' $ respectively. But since
    these are disjoint, this contradicts the definition of convergence.
\end{pf} \

\newpage
\begin{exr}[source=Primary Source Material]
    Show that the only Hausdorff topology on a finite set is the discrete
    topology.
\end{exr} \

\begin{soln}
    Suppose  there exists some subset which is not open. Denote this subset
    $ A \subseteq X $, with $ A = \set{p_{1}, \dots, p_{n}} $. Then, there must
    exist some $ \set{p_{i}} $ which is not open. But then any neighbourhood
    containing $ p_{i} $ necessarily contains another point. Since there can only
    be finitely many, then their intersection must also be open, but the
    intersection is precisely $ \set{p_{i}} $. Thus, $ A $ must be open.
\end{soln}

\begin{prop}
    Let $ X $ be Hausdorff, and $ A \subseteq X $. If $ p $ is a limit point of
    $ A $, then any neighbourhood of $ p $ contains infinitely many points of
    $ A $.
\end{prop}

See Exercise ?? for proof.

we're familiar w bases n stuff. lets skip that too

