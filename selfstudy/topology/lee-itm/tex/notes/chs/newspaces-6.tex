\subsection{Adjunction Spaces}
qspaces let us construct topospaces by ``attaching" them to each other. sps
$X,Y$ spaces, $A\subseteq Y$ closed, and $f:A\sto X$ cts. let $\sim$ be the equiv
rel defined on $X\coprod Y$ generated by $a\sim f(a)$ for all $a \in A$. denote
the resulting space as:
\begin{equation*}
    X \cup_{f} Y \ = \ (X \coprod Y)/\sim
\end{equation*}
this is called an \textbf{adjunction space}, and is said to be formed by
\textbf{attaching} $\bm{Y}$ \textbf{to} $\bm{X}$ \textbf{along} $\bm{f}$. the map
$f$ is called the \textbf{attaching map}. note any $x\in X$ is related to any pts
in $f^{-1}(x) \subseteq A$, if any. if $A=\eset$, then $U\cup_{f} Y=X\coprod Y$.

\begin{prop}
    $X \cup_{f} Y$ adj space, $q:X\coprod Y\sto X\cup_{f}Y$ the associated qmap.
    \begin{itemize}
        \item $q\rvert_{X}$ is an embedding, and $q(X)$ closed in $X\cup_{f}Y$.
        \item $q\rvert_{Y\setminus A}$ is an embedding, and $q(Y\setminus A)$
            open in $X\cup_{f}Y$.
        \item $X\cup_{f}Y \ = \ q(X) \sqcup q(Y\setminus A)$.
    \end{itemize}
\end{prop}

\begin{pf}[source=Primary Source Material]
    sps $B\subseteq X$ closed. note $q^{-1}(q(B))\cap X = B$ is closed in $X$,
    and $q^{-1}(q(B))\cap Y = f^{-1}(B)$ is closed in $A$ since $f$ cts, and thus
    closed in $Y$. thus $q(B)$ closed, so $q$ closed, and so $q(X)$ closed.
    clearly, $q$ also inj, and so it is an embedding.

    note $Y\setminus A$ is saturated open in $X\coprod Y$, so the restriction is
    a qmap. since $q$ bij, it is homeo, and the img is open by defn of qtopo.

    follows from defn of equiv rel.
\end{pf}

adjunction spaces are particularly useful for constructing manifolds. given
$n$-manifolds $M, N$ with non-empty boundaries and $\p M, \p N$ homeomorphic, let
$h:\p N \sto \p M$ be such a homeo. assuming invariance of bdry, we have that
$\p N$ is closed in $N$, thus we can define $M\cup_{h}N$. this space is said to
be formed by \textbf{attaching} $\bm{M}$ \textbf{and} $\bm{N}$ \textbf{together
along their boundaries}.

\begin{prop}[type=Theorem]
    in the above, $M\cup_{h}N$ is an $n$-manifold w/o bdry. there are embeddings
    $e:M\sto M\cup_{h}N$ and $f:N\sto M\cup_{h}N$ whose imgs r closed in
    $M\cup_{h}N$, and satisfy:
    \begin{equation*}
        e(M) \cup f(N) = M\cup_{h}N \qquad
        e(M) \cap f(N) = e(\p M) = f(\p N)
    \end{equation*}
\end{prop} \

\begin{pf}[source=Primary Source Material]
    first, we show $M\cup_{h}N$ is locally euclidean of dim $n$.
    let $q:M\coprod N\sto M\cup_{h}N$ be the qmap, and $S=q(\p M\cup\p N)$. note
    $\sint(M)\coprod\sint(N)$ saturated open in $M\coprod N$, so $q$ restricts to
    a qmap $\sint(M)\coprod\sint(N)\sto(M\cup_{h}N\setminus S)$. this is inj so
    homeo, so we only need to consider pts in $S$.

    let $s \in S,y\in\p N, x=h(y)\in\p M$ where $x,y$ are the two pts in the
    fiber $q^{-1}(s)$. we can choose charts $(U,\vphi)$ for $M$ and $(V,\psi)$
    for $N$ s.t. $x\in U,y\in V$. set:
    \begin{equation*}
      \hat{U}=\vphi(U),\hat{V}=\psi(V)\subseteq \bb{H}^{n}  
    \end{equation*}
    then, by shrinking $U,V$ as necessary, we assume that:
    \begin{equation*}
      h(V\cap\p N) = U\cap \p M \qquad \hat{U}=U_{0}\times[0,\ep) \qquad
      \hat{V}=V_{0}\times[0,\ep)  
    \end{equation*}
    for some $\ep > 0$ and $U_{0},V_{0} \subseteq \bR^{n-1}$. then, we can write
    $\vphi(x)=(\vphi_{0}(x),\vphi_{1}(x))$ where:
    \begin{equation*}
        \vphi_{0}:U\sto U_{0} \qquad \vphi_{1}:U\sto[0,\ep)
    \end{equation*}
    we do the same for $\psi$. $x,y$ being bdry pts means that
    $\vphi_{1}(x)=\psi_{1}(x)=0$.

    we want to assemble these charts into a map whose img is open in $\bR^{n}$ by
    matching their bdries. note that the restrictions given by
    \begin{equation*}
        \vphi_{0}\rvert_{U\cap\p M}:U\cap\p M \sto U_{0} \qquad
        \psi_{0}\rvert_{V\cap\p N}:V \cap\p N \sto V_{0}
    \end{equation*}
    are homeos. define a homeo $\beta:V_{0}\sto U_{0}$ and map $B:\hat{V}\sto
    \bR^{n}$ by:
    \begin{equation*}
        \beta = (\vphi_{0}\rvert_{U\cap\p M})\circ h \circ
        (\psi_{0}\rvert_{V\cap\p N})^{-1} \qquad
        B(x_{1},\dots,x_{n}) = (\beta(x_{1},\dots,x_{n-1}),-x_{n})
    \end{equation*}
    geometrically, this applies $\beta$ then vertically ``flips" pts. notice for
    $y\in V\cap\p N$:
    \begin{equation*}
        B\circ\psi(y)=(\beta\circ\psi_{0}(y),0)=(\vphi_{0}\circ h(y),0)
        = \vphi \circ h(y)
    \end{equation*}
    now, define $\tilde{\Phi}:U\coprod V\sto \bR^{n}$ by:
    \begin{equation*}
        \tilde{\Phi}(y)=
        \begin{cases}
            \vphi(y) & y \in U \\
            B\circ\psi(y) & y \in V
        \end{cases}
    \end{equation*}
    since $U\coprod V$ saturated open, the restriction of $q$ is a qmap to the
    nbhd $q(U\coprod V)$ of $s$, so $\tilde{\Phi}$ passes to the quotient and
    induces an inj cts $\Phi:q(U\coprod V)\sto\bR^{n}$. we can define an inverse
    as:
    \begin{equation*}
        \Phi^{-1}(y)=
        \begin{cases}
            q\circ \vphi^{-1}(y) & y_{n}\geq0 \\
            q\circ\psi^{-1}\circ B^{-1}(y) & y_{n}\leq 0
        \end{cases}
    \end{equation*}
    (why did he use $y^{n}$) these defns agree on their overlap, so the result is
    cts by gluing lemma. thus $\Phi$ homeo, so $M\cup_{h}N$ locally euclidean of
    dim $n$.

    the space $M\cup_{h}N$ is 2nd ctbl by 9.2. to show hausdorff, we show fibers
    of $q$ can be separated by saturated open subsets. straightforward to check
    that preimgs of sufficiently small balls works.

    by 11.1, $q$ restricts to an embedding of $N$ into $M\cup_{h}N$ with closed
    img. since $h$ homeo, it is easy to see that $M\cup_{h}N=N\cup_{h^{-1}}M$, so
    $q$ also restricts to an embedding of $M$ w/ closed img. the union of all
    imgs of these embeddings is $M\cup_{h}N$, and their intersection is $S$ as
    above, which is exactly the img of either bdry.
\end{pf} \

\begin{xmp}[title=Double of a Manifold with Boundary,source=Primary Source Material]
    sps $M$ an $n$-dim mfld w/ bdry. if $h:\p M\sto\p M$ id, the qspace
    $M\cup_{h}M$ denoted by $D(M)$ is called the \textbf{double of} $\bm{M}$. it
    is visualized by gluing two copies of $M$ along their bdries. if
    $\p M = \eset$, then $D(M)$ is the disjoint union of two copies of $M$.
\end{xmp}
the following is an immediate consequence.

\begin{prop}
    every $n$-mfld w/ bdry is homeo to a closed subset of an $n$-mfld w/o bdry.
\end{prop}


