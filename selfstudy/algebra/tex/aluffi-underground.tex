\documentclass{article}
\usepackage{preamble}
\usepackage{env}
\usepackage{configure}

% available environments:
% theorem: thm
% definition: defn
% proof: pf
% corollary: crll
% lemma: lm
% question: qu
% solution: soln
% example: xmp
% exercise: exr
%
% options: title=<title>   {all}
%          source=<source> {pf, qu, soln, xmp, exr}  Note: if content is taken directly from the main resource, cite the main resource as ``Primary source material"


% define these variables!
\def\coursecode{MAT347...?}
\def\coursename{Introduction to Abstract Algebra} % use \relax for non-course stuff
\def\studytype{1} % 1: Personal Self-Study Notes / 2: Course Lecture Notes / 3: Revised Notes / 4: Exercise Solution Sheet
\def\author{\me}
\def\createdate{November 19, 2024}
\def\updatedate{\today}
\def\source{Algebra - Notes from the Underground} % name, ed. of textbook, or `Class Lectures` for class notes
\def\sourceauthor{Paolo Aluffi} % for class notes, put lecturer
% \def\leftmark{} % set text in header; should only be necessary in assignments etc.
% \pagenumbering{arabic} % force revert numbering to default; should only be necessary in assignments etc.

\makeatletter
% settings for toc alignment
%
% Configuration
% -------------
% Horizonal alignment in \numberline:
%   l: left-aligned
%   c: centered
%   r: right-aligned
% \nl@align@: Default setting
% \nl@align@<levelname>: Setting for specific level

\def\nl@align@{l}% default
\def\nl@align@section{r}

\makeatother

\begin{document}

\cover
\toc
\blurb

% start here

\section{Rings}
\subsection{Before we start}

We'll be using some mildly unconventional defintions here and there.

\begin{defn}
    We say an integer $ p $ is \textbf{irreducible} if $ p \neq \pm 1 $ and the
    only divisors of $ p $ are $ \pm 1 $ and $ \pm p $.
\end{defn}

If you don't know what's coming, the above might seem strange.
If you do, like me, then it doesn't.

\begin{defn}
    An integer $ p $ is \textbf{prime} if $ p \neq \pm1 $ and:
    \begin{equation*}
        p \mid bc \ \implies \ p \mid b \trm{ or } p \mid c
    \end{equation*}
\end{defn}

\begin{crll}
    Suppose $ p \neq 0 $. Then $ p $ is prime iff $ p $ is irreducible.
\end{crll}

The key observation here is that \textit{we consider} $ 0 $ \textit{to be prime!}

\begin{thm}
    Let $ n > 1 $. Then the following are equivalent:
    \begin{enumerate}
        \item $ n $ is prime.
        \item If $ a \neq 0 $, then $ a^{-1} \mod n $ exists.
        \item For all $ a, b \in \bb{Z}_{n} $, $ ab = 0 \ \implies \ a = 0 \trm{ or } b = 0 $.
    \end{enumerate}
\end{thm}

\begin{pf}[source=Primary Source Material]
    its kinda lengthy and im lazy.
\end{pf}

\newpage
\subsection{Rings for real}

\begin{defn}
    you know what a ring is.
\end{defn}

Note that for the purposes of this text (and these notes),
a ring requires the existence of a multiplicative identity.
We may or may not also consider $ 0 $ a natural, not sure on this one tho...

\begin{lm}
    Let $ R $ be a ring. If $ 0 = 1 $, then $ R $ is the trivial ring.
\end{lm}

\begin{pf}
    Let $ r \in R $. Then:
    \begin{equation*}
        0 = 0r = 1r = r
    \end{equation*}
    So clearly $ r = 0 $, and so $ r $ is equal to the only element known to be in the ring.
\end{pf}

\begin{defn}
    Let $ R $ be a ring. \vsp
    %
    We say that an element $ a \in R $ is a \textbf{zero-divisor} if
    there exists a \textit{non-zero} $ b \in R $ such that $ ab = 0 $ or $ ba = 0 $. \vsp
    %
    If this is not the case, we say that $ a $ is a non-zero-divisor,
    sometimes abbreviated as nzd.
\end{defn}

This tells us that $ a \in R $ is a nzd precisely when $ ab = 0 $ implies $ b = 0 $ and $ ba = 0 $
implies $ b = 0 $. There is a special name for rings in which every nonzero element has such a
property.

\begin{defn}
    Let $ R $ be a commutative ring. Then $ R $ is called an \textbf{integral domain}
    if $ 0 \neq 1 $ and:
    \begin{equation*}
        \forall \, a, b \in R \qquad ab = 0 \implies a = 0 \trm{ or } b = 0
    \end{equation*}
\end{defn}

Notice that we can say that $ \bb{Z}_{p} $ is an integral domain iff $ p $ is prime
(recall that $ 0 $ is prime in this text!).

We see that integral domains are ``nicer" to some degree than even commutative rings.
Here is one such reason why.

\newpage
\begin{thm}
    Let $ R $ be a ring, and $ a \in R $ a nzd.
    Then, multiplicative cancellation holds for $ a $; that is:
    \begin{equation*}
        ab = ac \ \implies \ b = c \qquad ba = ca \ \implies \ b = c
    \end{equation*}
\end{thm}

\vspace{-0.1in}
\begin{pf}[source=Primary Source Material]
    We see that:
    \begin{equation*}
        ab = ac \ \implies \ a(b-c) = 0 \ \implies \ b-c = 0 \ \implies \ b = c
    \end{equation*}
    The proof for right-side cancellation is analogous.
\end{pf}

So commutative rings are special kinds of rings, and integral domains are special kinds of
commutative rings. There are other ways in which an integral domain may be even more special,
which we will see in the future. One particular property of interest are rings which produce
multiplicative inverses.

\begin{defn}
    If an element $ a \in R $ has a multiplicative inverse, it is said to be \textbf{invertible} or
    a \textbf{unit}. \vsp
    %
    The set of all units in $ R $ is denoted by $ R^{*} $ or $ R^{\times} $.
\end{defn}

Note that $ 0 $ is never invertible in any non-trivial ring.
There are, however, rings in which \textit{every} non-trivial element is invertible.

theyre fields. fields are integral domains. no surprise there.

We see that the concepts of integral domains and fields coincide greatly for $ \bb{Z}_{n} $ when
$ n > 1 $. We can keep going with this.

\begin{thm}
    Let $ R $ be a finite integral domain. Then, $ R $ is a field.
\end{thm}

\vspace{-0.1in}
\begin{pf}[source=Primary Source Material]
    Suppose $ R $ is a finite integral domain, and $ a \in R $ a nonzero element.
    It suffices to show that $ a $ has a multiplicative inverse. \vsp
    %
    Consider the elements of $ R $, all without repetition and their products:
    \begin{gather*}
        r_{1} \quad r_{2} \quad \dots \quad r_{m} \\
        ar_{1} \quad ar_{2} \quad \dots \quad ar_{m}
    \end{gather*}
    We show that these are \textit{also} the elements of $ R $ without repetition.
    Indeed, since $ a \neq 0 $ and $ R $ is an integral domain, then we see that:
    \begin{equation*}
        ar_{i} = ar_{j} \ \implies \ r_{i} = r_{j} \ \implies \ i = j
    \end{equation*}
    Therefore, there cannot be any repeats in the list.
    Since $ ar_{i} = 1 $ for some $ i $, then this implies that $ a $ has a multiplicative
    inverse, as needed.
\end{pf}

\newpage
\subsection{Exercises}
\begin{exr}[source=Primary Source Material]
    Show that if $ R $ has characteristic $ n $, then $ na = 0 $ for all $ a \in R $.
\end{exr}

\begin{exr}[source=Primary Source Material]
    Prove that the characteristic of an integral domain is prime.
\end{exr}

\begin{exr}[source=Primary Source Material]
    Prove that if $ R $ is an integral domain, then so is $ R[x] $.
\end{exr}

\end{document}
