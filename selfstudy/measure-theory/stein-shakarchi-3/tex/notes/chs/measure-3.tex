\subsection{``sigma"-algebras and Borel sets}

\begin{defn}
    a $\bm{\sigma}$\textbf{-algebra} is a collection of subsets closed under
    ctbl unions, intersections, and complements.
\end{defn}

in particular, the collection of all measurable subsets of $\bR^{d}$ forms a
$\sigma$-algebra.
another key example is the \textbf{Borel} $\bm{\sigma}$\textbf{-algebra} in
$\bR^{d}$, denoted $\cl{B}_{\bR^{d}}$, defined as the smallest $\sigma$-alg which
contains all open sets.
elements are called \textbf{Borel sets}.

of course, we first have to define ``smallest" $\sigma$-alg,
then prove existence and uniqueness to justify the definition.
to do this, note that an arbitrary intersection of $\sigma$-algs is a
$\sigma$-alg, and we define $\cl{B}_{\bR^{d}}$ as the intersection of all
$\sigma$-algs which contain all open sets.
smallest then means that it is a subset of any $\sigma$-alg containing all open
sets.

since open sets measurable, then the Borel $\sigma$-alg is contained in the
$\sigma$-alg of all measurable sets.
from the pov of borel sets, lebesgue sets are the \textit{completion} of
$\cl{B}_{\bR^{d}}$, i.e. by adjoining subsets of borel sets of measure 0.

starting with open/closed sets, we can try to list borel sets by complexity.
next would be ctbl intersections of open sets and their complements;
these sets have special names.

\begin{defn}
    a ctbl intersection of open sets is called a $\bm{G_{\delta}}$ set.
    a ctbl union of closed sets is called a $\bm{F_{\sigma}}$ set.
\end{defn}

\begin{prop}
    $E\subseteq\bR^{d}$ measurable iff it differs from a $G_{\delta}$
    set by a measure zero set, likewise for $F_{\sigma}$.
\end{prop}

\begin{pf}[source=Primary Source Material]
    reverse dir is trivial.
    sps $E$ measurable. then for all $n\geq1$ there exists
    open $E\subseteq O_{n}$ with $m(O_{n}\sm E)<1/n$.
    then $S=\bigcap_{n}O_{n}$ is $G_{\delta}$ and $S\sm E \subseteq O_{n}\sm E$
    for all $n$.
    thus $m(S\sm E)<1/n$ for all $n$, thus has measure 0.

    for $F_{\sigma}$, repeat with closed increasing subsets and take the union
    (prop 2.7(ii)).
\end{pf}

we now construct a non-measurable subset of $\bR$; this uses choice.
first, define an equiv rel $x\sim y \iff x-y\in\bQ$.
next, note $[0,1] = \bigcup_{\alpha}\cl{E}_{\alpha}$ is the disjoint union of
all equiv classes.
we construct our non-measurable set $\cl{N}$ by choosing one $x_{\alpha}$ from
each $\cl{E}_{\alpha}$,
and setting $\cl{N}=\set{x_{\alpha}}$ as the collection of all chosen elements.
we prove it is indeed non-measurable.

by contradiction, sps it is.
let $\set{r_{k}}$ enumerate $\bQ\cap[-1,1]$, and consider the translations:
\begin{equation*}
    \cl{N}_{k} = \cl{N}+r_{k}
\end{equation*}
we claim each $\cl{N}_{k}$ is disjoint, and:
\begin{equation*}
    [0,1]\subseteq\bigcup_{k}\cl{N}_{k}\subseteq[-1,2]
\end{equation*}
to see they are disjoint, sps $\cl{N}_{k}\cap\cl{N}_{k'}\neq\eset$.
then for some $\alpha,\beta$, there exist $r_{k}\neq r_{k'}$ with:
\begin{equation*}
    x_{\alpha}+r_{k} = x_{\beta}+r_{k'} \ \implies \
    x_{\alpha}-x_{\beta}=r_{k'}-r_{k}
\end{equation*}
this contradicts the fact that they are from distinct equiv classes.
the inclusions are straightforward.

now, if $\cl{N}$ measurable, then each $\cl{N}_{k}$ measurable, and since they
are disjoint:
\begin{equation*}
    1\leq\sum_{k}m(\cl{N}_{k})\leq3
\end{equation*}
since $\cl{N_{k}}$ is a translate of $\cl{N}$, then $m(\cl{N}_{k})=m(\cl{N})$
for all $k$.
but then no matter what, no value of $m(\cl{N})$ would satisfy the equation, so
it cannot be measurable as needed.

