\subsection{Measurable Functions}
we now turn to the core objects of integration theory: measurable functions.

\begin{defn}
    given a set $E$, the \textbf{characteristic function} is defined as:
    \begin{equation*}
        \chi_{E}(x) =
        \begin{cases}
            1 & x \in E \\ 0 & x \notin E
        \end{cases}
    \end{equation*}
    these serve as our starting point.
\end{defn}
next, we want to pass to the ``building blocks" of functions.
for riemann integrals, these are step functions;
however, we will need a more general notion.

\begin{defn}
    a \textbf{simple function} is a finite sum:
    \begin{equation*}
        f = \sum_{k=1}^{N}a_{k}\chi_{E_{k}}
    \end{equation*}
    where each $a_{k}$ is a constant, $E_{k}$ msrbl of finite measure
\end{defn}
in the case that $E_{k}$ are rectangles, these are step functions.

we begin by considering real-valued funcs, which we allow to
take on inf values $\pm \infty$, so that $f(x)$ belongs to the extended reals:
\begin{equation*}
    -\infty \leq f(x) \leq \infty
\end{equation*}
we will say $f$ is \textbf{finite-valued} if $f(x)$ finite for all $x$.
in our studies, we will almost always find ourselves in situations where
funcs are inf valued on at most a set of msr 0.

\newpage
\begin{defn}
    a func $f$ defd on a msrbl $E\subseteq\bR^{d}$ is \textbf{measurable} if:
    \begin{equation*}
        f^{-1}([-\infty,a))=\set{x\in E:f(x)<a}
    \end{equation*}
    is msrbl for all $a\in\bR$.
    we will denote this set $\set{f<a}$ when it will be clear.
\end{defn}
there are many equiv defns.
for ex, we may want the preim of closed intervals to be msrbl.
to prove equiv, we first note:
\begin{equation*}
    \set{f\leq a} \ = \ \bigcap_{k=1}^{\infty}\set{f<a+1/k}
\end{equation*}
and recall that ctbl intersection of msrbl is msrbl. similarly:
\begin{equation*}
    \set{f<a} \ = \ \bigcup_{k=1}^{\infty}\set{f\leq a-1/k}
\end{equation*}
similar rsning shows that using $\set{f\geq a}$ or $\set{f>a}$ also suffice.
as a simple consequence, $f$ msrbl iff $-f$ msrbl.
we can also similarly show that finite-valued $f$ msrbl iff all the sets
$\set{a<f<b}$ are, regardless of strict or weak inequalities.
by the same argument:

\begin{crll}
    finite-valued $f$ msrbl iff $f^{-1}(O)$ msrbl $\forall \, $ open $O$,
    iff $f^{-1}(F)$ msrbl $\forall \, $ closed $F$.
\end{crll}

note this holds for extended-valued funcs as well, if we also assume
$f^{-1}(\pm \infty)$ are msrbl.

\begin{crll}
    $f$ cts imp $f$ msrbl.
    $f$ msrbl fin-val and $\Phi$ cts imp $\Phi\circ f$ msrbl.
\end{crll}

topology beam

\begin{crll}
    let $\set{f_{n}}$ be a seq of msrbl funcs. then:
    \begin{equation*}
        \sup_{n}f_{n} \qquad \inf_{n}f_{n} \qquad
        \limsup_{n\sto\infty}f_{n} \qquad \liminf_{n\sto\infty}f_{n}
    \end{equation*}
    are all msrbl.
\end{crll}

*sidenote: sup/infs of funcs can be defd ptwise.
def order by $f\leq g$ if $f(x)\leq g(x)$ for all $x$; sup/inf are then the same.
for limsups/liminfs:
\begin{defn}
    given $\set{x_{n}}$, we write:
    \begin{gather*}
        \limsup_{n\sto\infty}x_{n}=\lim_{n\sto\infty}\sup_{m\geq n}x_{m}=
        \inf\set{\sup\set{x_{m}:m\geq n}:n\geq0} \\
        \liminf_{n\sto\infty}x_{n}=\lim_{n\sto\infty}\inf_{m\geq n}x_{m}=
        \sup\set{\inf\set{x_{m}:m\geq n}:n\geq0}
    \end{gather*}
    for $\set{f_{n}}$, the limsup of $f_{n}$ is a function $f'$ where:
    \begin{equation*}
        f'(x) = \limsup_{n\sto\infty}\set{f_{n}(x)}
    \end{equation*}
    liminf of $f_{n}$ is def'd equivalently.
\end{defn}

to see that $\sup_{n}f_{n}$ msrbl, note that
$\set{\sup_{n}f_{n}>a}=\bigcup_{n}\set{f_{n}>a}$.
since $\inf_{n}f_{n}=-\sup_{n}-f_{n}$, this also yields the inf case.
the limsup/inf follows from [the above defn].

\begin{crll}
    if $\set{f_{n}}$ a collection of msrbl funcs and $\lim_{n\sto\infty}f_{n}=f$,
    then $f$ msrbl.
\end{crll}

since $f=\limsup_{n}f_{n}=\liminf_{n}f_{n}$, this follows from 1.6.

\begin{crll}
    if $f$ msrbl then $f^{k},k\geq1$ msrbl.
    if $f,g$ both fin-val, then $f+g,fg$ msrbl.
\end{crll}

if $k$ odd, then $\set{f^{k}>a}=\set{f>a^{1/k}}$;
if $k$ even, then $\set{f^{k}>a}=\set{f>a^{1/k}}\cup\set{f<-a^{1/k}}$.
for the 2nd, we have that:
\begin{equation*}
    \set{f+g>a}=\bigcup_{r\in\bQ}\set{f>a-r}\cap\set{g>r} \qquad
    fg=\frac{(f+g)^{2}-(f-g)^{2}}{4}
\end{equation*}

\begin{defn}
    $f,g$ are equal \textbf{almost everywhere} (a.e.) if
    $\set{x\in E:f(x)\neq g(x)}$ has msr 0.
\end{defn}
more generally, a property is said to hold a.e. if it is true except on a set
of msr 0.

it is easy to see that if $f=g$ a.e. and $f$ msrbl, then $g$ msrbl.
similarly; most of the above properties can be relaxed to hold a.e..
in particular, the 2nd of 1.9 holds when $f,g$ are fin-val a.e..

note if $f,g$ def'd a.e. on msrbl $E\subseteq\bR^{d}$, then $f+g,fg$ can only
be def'd on the intersection of the domains.
since union of two msr 0 is msr 0, $f+g,fg$ are thus def'd a.e. as well.

we now examine the structure of msrbl funcs.
we begin by approximating ptwise, non-neg msrbl w simple funcs.

\newpage
\begin{prop}
    $f$ non-neg msrbl on $\bR^{d}$.
    then there exists inc seq of non-neg simple funcs $\set{\vphi_{k}}$ s.t.
    $\vphi_{k}\sto f$ ptwise.
\end{prop}

\begin{pf}[source=Primary Source Material]
    for $n\geq1$, let $Q_{n}$ be the cube centered at the origin of side len $n$.
    define:
    \begin{equation*}
        F_{n}(x) =
        \begin{cases}
            f(x) & x\in Q_{n},f(x)\leq n\\
            n & x\in Q_{n},f(x)>n \\
            0 & \trm{otw}
        \end{cases}
    \end{equation*}
    clearly, $F_{n}\sto f$;
    we then partition $[0,n]$, the range of $F_{n}$. for $0\leq\ell<nm$, define:
    \begin{gather*}
        E_{\ell,m}=\set{x\in Q_{n}:\frac{\ell}{m}<F_{n}(x)\leq\frac{\ell+1}{m}}
        \\
        F_{n,m}(x)=\sum_{\ell}\frac{\ell}{m}\chi_{E_{\ell,m}}(x)
    \end{gather*}
    each $F_{n,m}$ is simple and satisfies $0\leq F_{n}-F_{n,m}\leq1/m$.
    now, choose $n=m=2^{k}$, and let $\vphi_{k}=F_{2^{k},2^{k}}$.
    then $0\leq F_{m}-\vphi_{k}\leq1/2^{k}$, $\set{\vphi_{k}}$ inc, and the seq
    satisfies all desired properties.
\end{pf}
note this holds for non-neg ext-val if the limit $\infty$ is allowed.
now, we drop the assumption of non-neg, and allow the ext limit $-\infty$.

\newpage
\begin{prop}
    sps $f$ msrbl on $\bR^{d}$.
    then there exists simple $\set{\vphi_{k}}$ s.t. for all $x$:
    \begin{equation*}
        \abs{\vphi_{k}(x)}\leq\abs{\vphi_{k+1}(x)} \qquad
        \lim_{k\sto\infty}\vphi_{k}(x)=f(x)
    \end{equation*}
    in particular, $\abs{\vphi_{k}(x)}\leq\abs{f(x)}$.
\end{prop}




