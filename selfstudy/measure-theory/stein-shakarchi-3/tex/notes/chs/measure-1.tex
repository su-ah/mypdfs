\subsection{Exterior Measure}

we use standard topology notions and definitions, including distance of sets
we make the additional definition that a closed set is \textbf{perfect} if
it does not have any isolated pts (pts which are not limit pts).

we call boxes rectangles, and they are closed by default unless specified.
we denote the measure of a set $S$ as $\abs{S}$. which is kinda stupid imo.

\begin{defn}
    a union of rects is \textbf{almost disjoint} if the ints are disjoint
\end{defn}

\begin{prop}
    any open set in $\bR$ can be written uniquely as a ctbl union of disjoint
    open intervals
\end{prop}

\begin{pf}[source=Primary Source Material]
    let $O$ be open.
    for each $x\in O$, let $I_{x}$ be the largest open interval in $O$ containing
    $x$. clearly:
    \begin{equation*}
        O = \bigcup_{x\in O}I_{x}
    \end{equation*}
    sps $I_{x},I_{y}$ intersect.
    then their union contains $x$ and is contained in $O$.
    since $I_{x}$ maximal:
    \begin{equation*}
        I_{x}\cup I_{y}=I_{x} \qquad I_{x}\cup I_{y}=I_{y}
    \end{equation*}
    thus $I_{x}=I_{y}$.
    therefore any distinct intervals in $\cl{I}=\set{I_{x}}$ must be disjoint.
    finally, $\cl{I}$ is ctbl as each interval contains a rational.
    thus distinct intervals contain distinct rationals, so $\cl{I}$ must be ctbl.
\end{pf}

this allows us to naturally define the measure of open sets in $\bR$.
since the above construction is unique, then with $O=\bigcup I_{j}$ with disjoint
$I_{j}$'s, we can set $\abs{O}=\sum\abs{I_{j}}$.

\begin{prop}
    any open set in $\bR^{d}$ can be written as a ctbl union of almost disjoint
    closed rects
\end{prop}

pf: its like the same but way more tedious. whatever. topology blast

we can thus define measure for open sets in $\bR^{d}$ the natural way, however
we note that the above construction (lol) is not unique.
thus, it is not immediately obvious that the (same) sum is independent of
decomposition.

the cantor set will serve as motivation for defining a notion of measure.
\begin{defn}
    let $E\subseteq\bR$. the \textbf{exterior} or \textbf{outer measure} of $E$
    is:
    \begin{equation*}
        m_{*}(E) = \inf\sum_{j=1}^{\infty}\abs{Q_{j}}
    \end{equation*}
    taken over all ctbl coverings $E\subseteq\bigcup_{j=1}^{\infty}Q_{j}$ by
    closed cubes.
\end{defn} \

\begin{crll}
    for all $\ep>0$, there exists a covering $E\subseteq\bigcup_{j=1}^{\infty}
    Q_{j}$ with
    \begin{equation*}
        \sum_{j=1}^{m}m_{*}(Q_{j})\leq m_{*}(E)+\ep
    \end{equation*}
\end{crll}

we note some properties of exterior measure:
\begin{enumerate}
    \item $E_{1}\subseteq E_{2}\ \implies \ m_{*}(E_{1})\leq m_{*}(E_{2})$
    \item $E=\bigcup_{j=1}^{\infty}E_{j}\ \implies \
        m_{*}(E)\leq\sum_{j}m_{*}(E_{j})$
    \item $E\subseteq\bR^{d}\ \implies \ m_{*}(E)=\inf m_{*}(O)$ for all open
        $E\subseteq O$
    \item $E=E_{1}\cup E_{2},d(E_{1},E_{2})>0\ \implies \
        m_{*}(E)=m_{*}(E_{1})+m_{*}(E_{2})$
    \item $E=\bigcup_{j=1}^{\infty}Q_{j}$ almost disjoint $\ \implies \
        m_{*}(E)=\sum_{j}\abs{Q_{j}}$
\end{enumerate}
dont feel like including pfs cz they kinda suck but whatever

